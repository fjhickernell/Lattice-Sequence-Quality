
%% First Draft March 2002


%\documentclass[jcom]{apjrnl}
\documentclass{article}
\usepackage{amsmath,amsfonts,graphicx}


\renewcommand{\vec}[1]{\ensuremath{\mathbf{#1}}}
\newcommand{\vecsym}[1]{\ensuremath{\boldsymbol{#1}}}
\def\bbl{\text{\boldmath$\{$}}
\def\bbr{\text{\boldmath$\}$}}
\newcommand{\bbrace}[1]{\bbl #1 \bbr}
\newcommand{\bbbrace}[1]{\mathopen{\pmb{\bigg\{}}#1\mathclose{\pmb{\bigg\}}}}
%\def\bbl{\boldsymbol{\left \{}}
%\def\bbr{\boldsymbol{\right \}}}
\def\betahat{\hat\beta}
\def\e{\text{e}}
\def\E{\text{E}}

\newlength{\overwdth}
\def\overstrike#1{ 
\settowidth{\overwdth}{#1}\makebox[0pt][l]{\rule[0.5ex]{\overwdth}{0.1ex}}#1}

\def\abs#1{\ensuremath{\left \lvert #1 \right \rvert}}
\def\norm#1{\ensuremath{\left \lVert #1 \right \rVert}}

\DeclareMathOperator*{\argmin}{argmin}
\DeclareMathOperator{\sign}{sign}
\DeclareMathOperator{\cond}{cond}
%\DeclareMathOperator{\diag}{diag}
\DeclareMathOperator{\col}{col}
\DeclareMathOperator{\disc}{disc}
\DeclareMathOperator{\nullspace}{null}
\DeclareMathOperator{\Order}{O}
%\DeclareMathOperator{\rank}{rank}

\newcommand{\bfgam}{\vecsym \gamma}

\newcommand{\base}{\text{base}}
%\DeclareMathOperator{\base}{base}
\DeclareMathOperator*{\rms}{rms}
\DeclareMathOperator*{\argmax}{argmax}
\DeclareMathOperator{\rank}{rank}
\DeclareMathOperator*{\esssup}{esssup}

\newcommand{\til}[1]{\tilde{#1}}
\newcommand{\la}{\langle}
\newcommand{\ra}{\rangle}
\newcommand{\lfrf}[1]{\lfloor{#1}\rfloor}
\newcommand{\crno}{\cr\noalign{\vskip3mm}}
\newcommand{\ld}{\ldots}
\newcommand{\comb}[2]{{#1}\choose{#2}}

\newcommand{\one}{\vecsym 1}
\newcommand{\bfinfty}{\vecsym \infty}
\newcommand{\bfDelta}{\vecsym \Delta}
\newcommand{\bba}{\vec a}
\newcommand{\bbc}{\vec c}
\newcommand{\bbd}{\vec d}
\newcommand{\bfe}{\vec e}
\newcommand{\bfi}{\vec i}
\newcommand{\bbj}{\vec j}
\newcommand{\bbx}{\vec x}
\newcommand{\bfeta}{\vecsym \eta}
\newcommand{\bfsigma}{\vecsym \sigma}
\newcommand{\bftau}{\vecsym \tau}
\newcommand{\bfpsi}{\vecsym \psi}
\newcommand{\bfphi}{\vecsym \phi}
\newcommand{\bfPhi}{\vecsym \Phi}
\newcommand{\bfxi}{\vecsym \xi}
\newcommand{\bby}{\vec y}
\newcommand{\bbz}{\vec z}
\newcommand{\bbC}{\vec C}
\newcommand{\Linf}{{\mathcal L}_{\infty}}
\newcommand{\Ltwo}{{\mathcal L}_{2}}
\newcommand{\Lp}{{\mathcal L}_{p}}
\newcommand{\Lq}{{\mathcal L}_{q}}
\newcommand{\calF}{\mathcal F}
\newcommand{\calH}{\cal H}
\newcommand{\calW}{\mathcal W}
\newcommand{\Fb}{F_{b}}
\newcommand{\bfL}{\vec L}
\newcommand{\bfI}{\vec I}
\newcommand{\zero}{{\bf 0}}
\newcommand{\Cinf}{[0,1)^{\infty}}
\newcommand{\Cs}{[0,1)^{s}}
\newcommand{\Ctwos}{[0,1)^{2s}}
\newcommand{\ewo}{\text{err}^{w}}
\newcommand{\era}{\text{err}^{r}}
\newcommand{\Ksc}{K_{\Oscr}}
\newcommand{\tKsc}{\Tilde{K}_{\Oscr}}

\newcommand{\scr}{\text{sc}}
\newcommand{\Oscr}{\text{O}}
\newcommand{\OFTscr}{\text{OFT}}
\newcommand{\FTscr}{\text{FT}}
\newcommand{\g}{\vec g}
\newcommand{\B}{\text{B}}
\newcommand{\T}{\mbox{\scriptsize T}}
\newcommand{\bbell}{\vecsym \ell}
%\newcommand{\bbg}{\vec g}
\renewcommand{\k}{\vec k}
\newcommand{\h}{\vec h}
\renewcommand{\l}{\vec l}
\newcommand{\x}{\vec x}
\newcommand{\y}{\vec y}
\newcommand{\z}{\vec z}
\newcommand{\bfnu}{\vecsym \nu}
\newcommand{\bflam}{\vecsym \lambda}
\newcommand{\A}{\vec A}
\newcommand{\Zb}{\mathbb{Z}_{b}}
\newcommand{\Z}{\mathbb{Z}}
\newcommand{\Zinf}{\mathbb{Z}^{\infty}}
\newcommand{\tZs}{\tilde{\Z}^{s}}
\newcommand{\tZsn}{\tilde{\Z}^{s}_{n}}
\newcommand{\tZu}{\tilde{\Z}^{u}}
\newcommand{\tZun}{\tilde{\Z}^{u}_{n}}
\newcommand{\R}{\mathbb{R}}


\newcommand{\dkv}{{\bbd}(\kappa,v)}

%%********************************************************

%%%Define theorem environments %%%%%%%%%%%%
%   %%%----Theorem type environments
%   Authors: customize as desired but please do not change theoremstyle commands.
%   Please see comments regarding numbering schemes for each environment
%
%   %%%h4a heads - Environments usually associated with proofs.
% {\theoremstyle{plain}%
  % \newtheorem{theorem}{Theorem}[]%           %theorem numbers will
%                                                     %include section numbers
% %
%   \newtheorem{corollary}[theorem]{Corollary}%       %Corollaries will number
%                                                     %with theorem numbers
% %
%   \newtheorem{proposition}[theorem]{Proposition}%    %Proposition numbers will include
%                                                         %subsection numbers
% %
%   \newtheorem{lemma}[theorem]{Lemma}%                    %Lemmas will number in a single
%                                                 %sequence throughout the article.
% }
% %   %%%h4 heads - Environments not associated with proofs
% {\theoremstyle{remark}
% \newtheorem{fact}{Fact}
% \newtheorem{remark}{Remark}
% }
% %   %%%h4b heads - Definitions and examples
% {\theoremstyle{definition}
% \newtheorem{definition}[theorem]{Definition}
% \newtheorem{example}{Example}
% }
%%%%%%%%%%%%%%%%%%%%%%%%%%%%%%%%%%%%%%%%%%%
\begin{document}

                          
\title{The Existence of Good Extensible \\  Rank-1 Lattices
\thanks{This work was partially
supported by a Hong Kong Research Grants Council grant HKBU/2030/99P.}} 

\ifruninauthors
\else
%%Stacked style author template
\author{Fred J. Hickernell \\
Department of Mathematics, Hong Kong Baptist University \\ Kowloon
Tong, Hong Kong SAR, China
\\ E-mail: fred@hkbu.edu.hk
\and         % please use the \and command between authors
 Harald Niederreiter \\
Department of Mathematics,
National University of Singapore \\ 
2 Science Drive 2,
Singapore 117543,
Republic of Singapore\\
E-mail: nied@math.nus.edu.sg
}
\fi



\date{April 2, 2002}

\maketitle

\begin{abstract}
{Extensible integration lattices have the attractive property that the
number of points in the node set may be increased while retaining the
existing points.  It is shown here that there exist generating
vectors, $\h$, for extensible rank-1 lattices such that for
$n=b, b^{2}, \ldots$ points and dimensions $s=1, 2, \ldots$ the
figures of merit $R_{\alpha}$, $P_{\alpha}$ and discrepancy are all
small.  The upper bounds obtained on these figures of merit for
extensible lattices are some power of $\log n$ worse than the best
upper bounds for lattices where $\h$ is allowed to vary with $n$ and
$s$.}
\end{abstract}

\keywords{discrepancy, figures of merit, extensible lattices, lattice
rules, quasi-Monte Carlo integration}

%\begin{article}


\section{Introduction}

Multidimensional integrals over the unit cube are often approximated 
by taking the average over a well-chosen set of points:
$$
\int_{\Cs} f(\x) \ d\x \approx \frac 1n \sum_{i=0}^{n-1} f(\z_{i}).
$$
Quasi-Monte Carlo methods \cite{Nie92} choose the set $\{\z_{i}\}$ to
be evenly distributed over the unit cube.  One popular choice of
points are the node sets of rank-1 integration lattices (see
\cite{HuaWan81}, \cite[Chapter 5]{Nie92}, and \cite{SloJoe94}).  Such
sets may be expressed as 
\begin{equation} \label{shlat}
    \{\bbrace{i \h/n + \bfDelta} : i=0,1, \ldots, n-1 \},
\end{equation}
where $\h$ is an $s$-dimensional integer generating vector, $\bfDelta$
is an $s$-dimensional shift, and $\bbrace{\cdot}$ denotes the
fractional part, i.e., $\bbrace{\x} = \x \pmod 1$.  The quality of the
set defined in \eqref{shlat} depends mainly on the choice of $\h$ and
somewhat on the choice of $\bfDelta$.  There are several quality
measures, including $R_{\alpha}$, $P_{\alpha}$, and the discrepancy, which will
be defined and discussed later in this article.

One disadvantage of lattice rules for numerical integration has been that the
generating vector depends on $n$ and $s$.  If one changes either of
these, then one must also change $\h$.  Recently, lattice node sets
have been proposed that can be extended in $n$ (for this reason they are
called {\it extensible lattices}).  For a fixed integer base $b
\ge 2$, let  any integer $i \ge 0$ be expressed as $i=i_{1}+i_{2}b +i_{3} 
b^{2} + \cdots$, where the digits $i_j$ are in $\{0,1,\ldots,b-1 \}$.  Then define the radical inverse function as
$$
\Phi_{b}(i) = i_{1}b^{-1} + i_{2}b^{-2} + i_{3}b^{-3} + \cdots .
$$
The sequence 
$$
\{ \z_{i}=\bbrace{\h \Phi_{b}(i) + \bfDelta} : i=0, 1, \ldots \}
$$
has the property that any finite piece with $n=b^{m}$ points,
\begin{multline} \label{nodeset}
\{\z_{i}=\bbrace{\h \Phi_{b}(i+lb^{m}) + \bfDelta} =\bbrace{\h 
\Phi_{b}(i) + \Phi_b(l)\h b^{-m} + \bfDelta}: \\
 i=0,1, \ldots,
b^{m}-1 \}, \quad l=0, 1, \ldots, \ m=1,2, \ldots
\end{multline}
is the node set of a shifted rank-1 lattice \cite{HicHon97a}, 
\cite{HicEtal00}, just
like the pieces of a $(t,s)$-sequence are $(t,m,s)$-nets \cite[Chapter
4]{Nie92}.  Good generating vectors $\h$ that work
simultaneously for a range of $m$
and $s$ have been found experimentally \cite{HicEtal00}.  The purpose
of this article is to prove that there exist generating vectors $\h$,
dependent on $b$, but independent of $m$ and $s$, that give node sets
of the form \eqref{shlat} and \eqref{nodeset} with figures of merit that are nearly as
good as the best known upper bounds for node sets where the generating
vector is allowed to depend on $m$ and $s$.

The remainder of this introduction reviews several figures of merit
for lattice rules.  The next section gives an upper bound on the value
of $R_{\alpha}$ for extensible lattices.  This is used to give upper
bounds on $P_{\alpha}$ and discrepancy in the following two sections.

In this article it is assumed that the integrand, $f$, is a function
of $\x=(x_{1}, x_{2}, \ldots) \in \Cinf$, and that the integration
domain is to be $\Cinf$.  Integration problems over $\Cs$ just assume
that $f$ depends only on the first $s$ variables.  Let $1:s$ denote
the set $\{1, \ldots, s\}$, and for any $u \subset 1:\infty$, let
$\x_{u}$ denote the vector indexed by the elements of $u$.  Let
$\abs{u}$ denote the cardinality of $u$.  Throughout this article it
is assumed that $u$ is a finite set.

Suppose that the integrand can be written as an absolutely summable
Fourier series,
$$
f(\x) = \sum_{\k \in \Z^{\infty}} \tilde{f}(\k) e^{2 \pi \imath 
\k^{T} \x},
$$
where $\imath=\sqrt{-1}$.  For any parameter $\bfgam \in
[0,\infty)^{\infty}$ and $\k \in \Z^{\infty}$ let
$$
\tilde{r}(\k,\bfgam) = \prod_{j=1}^{\infty} r(k_{j},\gamma_{j}),
\qquad \text{where} \qquad r(k_{j},\gamma_{j})=\begin{cases} 1, &
k_{j}=0, \\ \gamma_{j}^{-1} \abs{k_{j}}, & k_{j} \ne 0.  \end{cases}
$$ 
For any $\alpha > 1$ this quantity can be used to define a Banach
space of functions:
$$
\calF_{\alpha} = \{ f \in \Ltwo(\Cinf) : 
\norm{f}_{\calF_{\alpha}} < \infty \},
$$
where
$$
\norm{f}_{\calF_{\alpha}} = \sup_{\k \in \Z^{\infty}}
\left(\tilde{r}(\k,\bfgam)^{\alpha} \abs{\tilde{f}(\k)} \right).
$$
If $\gamma_{j}=0$, then functions in $\calF_{\alpha}$ are assumed not
to depend on the coordinate $x_{j}$, i.e., $\tilde{f}(\k)=0$ for all
$\k$ with $k_{j} \ne 0$.  Let $\tZu = \{ \k \in \Z^{\infty} : k_{j}=0
\ \forall j \notin u \}$.  Then the subspace of $\calF_{\alpha}$
containing functions depending only on the coordinates indexed by $u$ is
$\calF_{u,\alpha} = \{ f \in \calF_{\alpha} : \tilde{f}(\k)=0 \
\forall \k \notin \tZu \}$.

For any positive integer $n$ and any $\h \in \Z^{\infty}$, the dual
lattice consists of all wave numbers $\k$ with $\k^{T}\h = 0 \mod n$.  The set $B(\h,n,u) = \{ \zero \ne \k \in \tZu : \k^{T}\h = 0 \mod n\}$ consists
of all nonzero wave numbers in the dual lattice whose nonzero
components are in the directions indexed by $u$.  Let
\begin{equation} \label{Pdef}
P_{\alpha}(\h,\bfgam,n,u) = \sum_{\k \in B(\h,n,u)} 
\tilde{r}(\k,\bfgam)^{-\alpha}.
\end{equation}
Then one may derive the following tight worst-case integration error bound 
for lattice rules using the node set \eqref{shlat} \cite{Hic98b}:
\begin{equation} \label{quaderr1}
    \abs{\int_{\Cinf} f(\x) \ d\x - \frac 1n \sum_{i=0}^{n-1} f(\z_{i})}
    \le P_{\alpha}(\h,\bfgam,n,u) \norm{f}_{\calF_{\alpha}} \quad
    \forall f \in \calF_{u,\alpha}.
\end{equation}

For band-limited integrands one may derive a similar error bound.  Let
$\tZun=\tZu \cap (-n/2, n/2]^{\infty}$ and $\calF_{n,u,\alpha} = \{ f
\in \calF_{u,\alpha} : \tilde{f}(\k)=0 \ \forall \k \notin \tZun
\}$.  Then the analogous error bound to \eqref{quaderr1} is
\begin{equation} \label{quaderr2}
    \abs{\int_{\Cinf} f(\x) \ d\x - \frac 1n \sum_{i=0}^{n-1} f(\z_{i})}
    \le R_{\alpha}(\h,\bfgam,n,u) \norm{f}_{\calF_{\alpha}} \quad
    \forall f \in \calF_{n,u,\alpha},
\end{equation}
where
\begin{equation} \label{Rdef}
R_{\alpha}(\h,\bfgam,n,u) = \sum_{\k \in \tilde{B}(\h,n,u)}
\tilde{r}(\k,\bfgam)^{-\alpha}
\end{equation}
and $\tilde{B}(\h,n,u) = B(\h,n,u)\cap (-n/2, n/2]^{\infty}$.  Note that
whereas one must require $\alpha>1$ for $P_{\alpha}(\h,\bfgam,n,u)$ to be
defined and \eqref{quaderr1} to make sense, the error bound
\eqref{quaderr2} is well defined for $\alpha \ge 0$.  Note also that 
both $P_{\alpha}(\h,\bfgam,n,u)$ and $R_{\alpha}(\h,\bfgam,n,u)$ do not depend on 
the shift vector $\bfDelta$.

Lattice rules are not only used for periodic integrands, but
non-periodic ones as well.  Let $\calW_{p}$ denote the Banach
space of functions that are absolutely continuous on $[0,1]^{\infty}$
with $\Lp$-integrable mixed partial derivatives of up to order one in
each coordinate direction.  Let $\gamma_{u} = \prod_{j \in u}
\gamma_{j}$.  The norm for this space is defined as
\begin{align*}
\norm{f}_{\calW_{p}} &= \left\{ \sum_{\substack {u \subseteq 1:\infty
\\ \abs{u} < \infty}}  \gamma_{u}^{-p} \int_{[0,1]^{u}} \abs{ \left
.\frac{\partial^{|u|} f}{\partial \x_{u}}\right
\rvert_{\x_{1:\infty \setminus u}=\one}}^{p} \ d \x_{u} \right\}^{1/p}, \quad 1
\le p < \infty, \\
\norm{f}_{\calW_{\infty}} &= \sup_{\substack {u \subseteq 1:\infty \\
\abs{u} < \infty}} \gamma_{u}^{-1} \sup_{\x_{u} \in
[0,1]^{u}} \abs{ \left .\frac{\partial^{|u|} f}{\partial \x_{u}}\right
\rvert_{\x_{1:\infty \setminus u}=\one}} .
\end{align*}
Again, if $\gamma_{j}=0$, then the functions in $\calW_{p}$ are
assumed not to depend on $x_{j}$.  As above let $\calW_{u,p}$ denote
the subspace of $\calW_{p}$ whose elements do not depend on
$x_{j}$ for $j \notin u$.  The integration error bound for this
space is \cite{WanHic00b}
\begin{multline} \label{quaderr3}
    \abs{\int_{\Cinf} f(\x) \ d\x - \frac 1n \sum_{i=0}^{n-1}
    f(\z_{i})} \le D_{q}^{*}(\{\z_{i}\},n,u)
    \norm{f}_{\calW_{p}} \\
    \forall f \in \calW_{u,p}, \ \frac 1p + \frac 1q =1,
\end{multline}
where the (weighted) $\Lq$-star discrepancy is defined as
\begin{subequations} \label{discdef}
\begin{align}
D_{q}^{*}(\{\z_{i}\},\bfgam,n,u) &= \left\{ \sum_{\emptyset \subset v
\subseteq u} \gamma_{v}^{q} \int_{[0,1]^{v}} \abs{
\disc_{v}(\zero, \x, \{\z_{i}\},n)}^{q} \ d \x_{v} \right\}^{1/q}, 
\nonumber \\
& \qquad \qquad \qquad \qquad 1 \le q < \infty, \\
D_{\infty}^{*}(\{\z_{i}\},\bfgam,n,u) &= \sup_{\emptyset \subset v \subseteq
u} \gamma_{v} \sup_{\x_{v} \in [0,1]^{v}} \abs{
\disc_{v}(\zero,\x, \{\z_{i}\},n)} .
\end{align}
The discrepancy function, $\disc_{u}(\y,\x, \{\z_{i}\},n)$, measures
the difference between the volume of the box $[\y_{u},\x_{u})$ and the
proportion of sample points inside it:
\begin{multline} \label{discfundef}
\disc_{u}(\y,\x, \{\z_{i}\},n) \\
= \prod_{j \in u} (x_{j} - y_{j}) -
\frac 1n \sum_{i=0}^{n-1} \prod_{j \in u} 
(1_{(z_{j},\infty)}(x_{j}) - 1_{(z_{j},\infty)}(y_{j})),
\end{multline}
\end{subequations}
where $y_{j} \le x_{j}$ for all $j \in u$.  Here $1_{\{\cdot\}}$
denotes the characteristic function.

One may also define an $\Lq$-unanchored discrepancy:
\begin{subequations} \label{undiscdef}
\begin{multline}
D_{q}(\{\z_{i}\},\bfgam,n,u) \\
= \left\{ \sum_{\emptyset \subset v \subseteq u} 
2^{\abs{v}}\gamma_{v}^{q} \int_{\substack{[0,1]^{2v} \\
\y_{v} \le \x_{v}}} \abs{
\disc_{v}(\y, \x, \{\z_{i}\},n)}^{q} \ d \y_{v} \ d \x_{v}  
\right\}^{1/q}, \\
 1 \le q < \infty, 
\end{multline}
\begin{equation}
D_{\infty}(\{\z_{i}\},\bfgam,n,u) = \sup_{\emptyset \subset v
\subseteq u} \gamma_{v} \sup_{\zero \le \y_{v} \le \x_{v} 
\le \one} \abs{ \disc_{v}(\y,\x, \{\z_{i}\},n)}.
\end{equation}
\end{subequations}
Both the star and the unanchored discrepancies may be bounded above as 
follows:
\begin{multline} \label{discbds}
D_{q}^{*}(\{\z_{i}\},\bfgam,n,u), D_{q}(\{\z_{i}\},\bfgam,n,u) \\
\le \sum_{\emptyset \subset v \subseteq u} \gamma_{v} \sup_{\zero
\le \y_{v} \le \x_{v} \le \one} \abs{ \disc_{v}(\y,\x, \{\z_{i}\},n)},
\quad 1 \le q \le \infty.
\end{multline}

\section{Existence of Extensible Lattices with Small $R_{\alpha}$}

The existence of lattices with small $R_{\alpha}(\h,\bfgam,n,1:s)$ for
fixed $n$ and $s$ has been shown using averaging arguments.  After
computing an upper bound on the average of $R_{\alpha}(\h,\bfgam,n,1:s)$
over some set of $\h$, one argues that there exist at least some $\h$
with an $R_{\alpha}(\h,\bfgam,n,1:s)$ at least as small as that upper
bound. In the following, an averaging argument is also used, but with a
difference.  For extensible lattice rules it will be argued that there
exist some $\h$, independent of $n$ and $s$, such that
$R_{\alpha}(\h,\bfgam,n,1:s)$ is not too much worse than the upper bound
on the average for all $n$ and $s$.

For extensible lattices one needs to allow generating vectors, $\h$,
whose coordinates are generalizations of integers.  For any integer $b \ge 2$
let $\Zb$ be the set of all $b$-adic integers
$i= \sum_{l=1}^{\infty}i_{l}b^{l-1}$, where $i_{l} \in \{0,1, \ldots, b-1 \}$
for all $l \ge 1$; see Mahler \cite{Mah73} for the theory of $b$-adic integers.
 Note that $\Zb
\supset \Z$.  The set $\Zb$ contains too many bad numbers, so define
$$
H_{b} = \{ i \in \Zb : \gcd(i_{1},b)=1\}.
$$
Then $H_{b}^{\infty} = H_{b} \times H_{b} \times \cdots$ is the set of
candidates for $\h$, an $\infty$-vector.  The set $\Zb$ has a
probability measure such that the set of all $i\in \Zb$ with specified
first $l$ digits has measure $b^{-l}$.  This probability measure
conditional on $H_{b}$ is denoted $\mu_{b}$.  The Cartesian product $H_{b}^{\infty}$
has the product probability measure $\mu_{b}^{\infty}$.  The following upper bound
on the average of $R_1(\h,\one,n,u)$ taken over $H_{b}^{\infty}$ is implied by the proof of \cite[Theorem 1]{Nie78b}. For $\h \in H_b^{\infty}$ we define
$R_{\alpha}(\h,\bfgam,n,u)$ with $n=b,b^2,\ldots$ in the obvious way, namely
by \eqref{Rdef} with $\h$ considered modulo $n$. 

\begin{lemma} \label{Haraldlem} For any finite $u \subset 1:\infty$ we have 
$$
\int_{H_{b}^{\infty}} R_{1}(\h,\one,n,u) \ d\mu_{b}^{\infty}(\h) \le n^{-1} 
(\beta_{1} + \beta_{2} \log n)^{\abs{u}}
$$
for some positive absolute constants $\beta_{1}$ and $\beta_{2}$ and $n=b, 
b^{2}, \ldots$.
\end{lemma}

\begin{theorem} \label{Rthm} Suppose we are given a fixed integer $b \ge
2$, a fixed $\bfgam \in [0,\infty)^{\infty}$, a fixed $\alpha \ge 1$,
and a fixed $\epsilon > 0$.  

i) There exist a $\mu_b^{\infty}$-measurable $\tilde{G}_{b} \subset H_{b}^{\infty}$ and some constant
$C_{R}(\alpha,\bfgam,\epsilon,s)$ such that for all $\h \in
H_{b}^{\infty} \setminus \tilde{G}_{b}$,
\begin{multline} \label{ntoinfty}
    R_{\alpha}(\h,\bfgam,n,1:s) \le C_{R}(\alpha,\bfgam,\epsilon,s)
    n^{-\alpha} (\log n)^{\alpha(s+1)} [\log \log (
    n+1)]^{\alpha(1+\epsilon)}, \\ n=b, b^{2},\ldots,\ s=1,2,\ldots.
\end{multline}
Furthermore, one may make $\mu_{b}^{\infty}(\tilde{G}_{b})$ arbitrarily close to 
zero by choosing  $C_{R}(\alpha,\bfgam,\epsilon,s)$ large enough.

ii)  If $\sum_{j=1}^{\infty} \gamma_{j}^{a} j [\log(j+1)]^{1+\epsilon} < \infty$ for some
$a \in [1,\alpha]$, then for any fixed $\delta>0$ there exists some
$\tilde{C}_{R}(\alpha,a,\bfgam,\delta)$ such that for all $\h \in
H_{b}^{\infty} \setminus \tilde{G}_{b}$ we have
\begin{multline}\label{nstoinfty}
R_{\alpha}(\h,\bfgam,n,1:s) \le \tilde{C}_{R}(\alpha,a,\bfgam,\delta) 
n^{-\alpha/a+\delta}, \\
n=b,b^{2},\ldots,\ s=1,2,\ldots.
\end{multline}
Again one may make $\mu_{b}^{\infty}(\tilde{G}_{b})$ arbitrarily close to 
zero by choosing the above leading constant large enough.

iii)  If $\sum_{j=1}^{\infty} \gamma_{j}^{a} < \infty$ for some $a \in
[1,\alpha]$, then for any fixed $\delta>0$ there exists a
$\mu_b^{\infty}$-measurable $G_{b}
\subset H_{b}^{\infty}$ such that for all $\h \in  H_{b}^{\infty} \setminus 
G_{b}$ \eqref{nstoinfty} is satisfied, but \eqref{ntoinfty} may not 
be.  Here too, $\mu_{b}^{\infty}(G_{b})$ can be made arbitrarily close to 
zero.
\end{theorem}

\begin{proof} Define the quantities
\begin{gather}
    \tilde{R}_{\alpha}(\h,\bfgam,n,s) = \sum_{\{s\} \subseteq u \subseteq
    1:s} \gamma_{u}^{\alpha} R_{\alpha}(\h,\one,n,u), \nonumber \\
    \label{Rhatdef}
    \hat{R}_{\alpha}(\h,\bfgam,n,s) = \sum_{\emptyset \subset u
    \subseteq 1:s} \gamma_{u}^{\alpha} R_{\alpha}(\h,\one,n,u) =
    \sum_{d=1}^{s} \tilde{R}_{\alpha}(\h,\bfgam,n,d).
\end{gather}
Note that $R_{\alpha}(\h,\bfgam,n,1:s) \le
\hat{R}_{\alpha}(\h,\bfgam,n,s)$.  The proof here actually shows the
above conclusions for $\hat{R}_{\alpha}(\h,\bfgam,n,s)$ rather than
$R_{\alpha}(\h,\bfgam,n,1:s)$ because these are needed for Theorem
\ref{Dthm} below.  

There are two important relations among 
$\hat{R}_{\alpha}(\h,\bfgam,n,s)$ for different parameters $\alpha$ 
that are used in this proof.  Because $R_{\alpha}(\h,\one,n,u)$ is 
non-increasing for $\alpha$ increasing, it follows that
\begin{align}
    \hat{R}_{\alpha}(\h,\bfgam,n,s) &  = \sum_{\emptyset \subset u
    \subseteq 1:s} \gamma_{u}^{\alpha} R_{\alpha}(\h,\one,n,u) 
    \le \sum_{\emptyset \subset u
    \subseteq 1:s} \gamma_{u}^{\alpha} R_{a}(\h,\one,n,u) \nonumber \\
    & = \hat{R}_{a}(\h,\bfgam^{\alpha/a},n,s) \label{mono}
\end{align}
for $\alpha \ge a$, where $\bfgam^{\alpha/a}$ denotes the vector
obtained by raising each component of $\bfgam$ to the power
$\alpha/a$.  Furthermore, the definitions of $R_{\alpha}$ and
$\hat{R}_{\alpha}$ together with Jensen's inequality imply that
\begin{align}
    \hat{R}_{\alpha}(\h,\bfgam,n,s) &  = \sum_{\emptyset \subset u
    \subseteq 1:s} \gamma_{u}^{\alpha} R_{\alpha}(\h,\one,n,u) 
    \le \sum_{\emptyset \subset u
    \subseteq 1:s} [\gamma_{u}^{a} R_{a}(\h,\one,n,u) ]^{\alpha/a} \nonumber \\
    &\le \left[\sum_{\emptyset \subset u
    \subseteq 1:s} \gamma_{u}^{a} R_{a}(\h,\one,n,u) 
    \right]^{\alpha/a} = [\hat{R}_{a}(\h,\bfgam,n,s)]^{\alpha/a} \label{Jens}
\end{align}
for $\alpha \ge a$.
    
Lemma \ref{Haraldlem} implies the
following upper bound on the average value of
$\tilde{R}_{1}(\h,\bfgam,n,s)$:
$$
\int_{H_{b}^{\infty}} \tilde{R}_{1}(\h,\bfgam,n,s) \ d\mu_{b}^{\infty}(\h)
\le \tilde{M}(\bfgam,n,s) \quad \text{for} \quad n=b, 
b^{2}, \ldots,
$$
where
$$
\tilde{M}(\bfgam,n,s) := n^{-1} \gamma_{s} (\beta_{1}
+ \beta_{2} \log n) \prod_{j=1}^{s-1}[1+\gamma_{j} (\beta_{1} +
\beta_{2} \log n)].
$$

For any $\epsilon > 0$ and $j=1,2,\ldots$ define $c_{j}:= c_j(\epsilon):= c_{0}(\epsilon) j
[\log(j+1)]^{1+\epsilon}$, where $c_{0}(\epsilon)$ is chosen to be
larger than $\sum_{j=1}^{\infty} j^{-1} [\log(j+1)]^{-1-\epsilon}$.  A
set of bad $\h$ for a particular $n=b^{m}$ and $s$ is then defined as
\begin{multline*}
\tilde{G}_{bms} = \{ \h \in H_{b}^{\infty} :
\tilde{R}_{1}(\h,\bfgam,b^{m},s) > c_{m}c_{s}
\tilde{M}(\bfgam,b^{m},s) \},\\
m=1, 2, \ldots, \ s=1,
2, \ldots.
\end{multline*}
These sets cannot be too large.  In fact, $\mu_{b}^{\infty}(\tilde{G}_{bms})
 \le
1/(c_{m}c_{s})$ since
\begin{multline*}
\mu_{b}^{\infty}(\tilde{G}_{bms}) c_{m}c_{s} \tilde{M}(\bfgam,b^{m},s) \le
\int_{\tilde{G}_{bms}} \tilde{R}_{1}(\h,\bfgam,b^{m},s) \ d\mu_{b}^{\infty}(\h)
\\
\le \int_{H_{b}^{\infty}} \tilde{R}_{1}(\h,\bfgam,b^{m},s) \
d\mu_{b}^{\infty}(\h) \le \tilde{M}(\bfgam,b^{m},s).
\end{multline*}
The set of all bad $\h$ is defined as the union of these sets:  
$$
\tilde{G}_{b} = \bigcup_{m=1}^{\infty} \bigcup_{s=1}^{\infty} \tilde{G}_{bms}. 
$$
It follows that this set is also not too large, namely,
$$
\mu_{b}^{\infty}(\tilde{G}_{b}) \le \sum_{m=1}^{\infty}
\sum_{s=1}^{\infty} \frac{1}{c_{m}c_{s}} = \left \{ \frac 1{c_{0}(\epsilon)}
\sum_{j=1}^{\infty} \frac{1}{ j [\log(j+1)]^{1+\epsilon}} \right 
\}^{2} < 1.
$$
Thus, $\mu_{b}^{\infty}(H_{b}^{\infty} \setminus \tilde{G}_{b}) >0$, i.e., there exist 
some $\h$ that are not in $\tilde{G}_{b}$.  In fact, 
$\mu_{b}^{\infty}(H_{b}^{\infty} \setminus \tilde{G}_{b})$ may be made arbitrarily close 
to $1$ by choosing $c_{0}(\epsilon)$ large enough.

For any $\h \in H_{b}^{\infty} \setminus \tilde{G}_{b}$ one may derive the following 
upper bound on $\hat{R}_1$:
\begin{align*}
    \hat{R}_{1}(\h,\bfgam,b^m,s) &= \sum_{j=1}^{s}
    \tilde{R}_{1}(\h,\bfgam,b^{m},j) \le c_{m} \sum_{j=1}^{s} c_{j}
    \tilde{M}(\bfgam,b^{m},j) \\
   & \le c_{0}(\epsilon) m [\log(m+1)]^{1+\epsilon} b^{-m} \\
   & \qquad \qquad \times \prod_{j=1}^{s} [1 + \gamma_{j}
   c_{0}(\epsilon) j \{\log(j+1)\}^{1+\epsilon} (\beta_{1} + \beta_{2} m \log b
   ) ]
\end{align*}
for $m=1,2, \ldots$ and $s=1,2, \ldots$.  Applying the inequality
$\hat{R}_{\alpha}(\h,\bfgam,n,s)\le
[\hat{R}_{1}(\h,\bfgam,n,s)]^{\alpha}$ from \eqref{Jens} implies part
i) of Theorem \ref{Rthm}.

If $\sum_{j=1}^{\infty} \gamma_{j}^{a} j [\log(j+1)]^{1+\epsilon} < \infty$, then by
\eqref{mono} above and Lemma \ref{boundinfprd} below it follows that for any
$\delta>0$ and any $\h \in H_{b}^{\infty} \setminus \tilde{G}_{b}$,
\begin{multline*}
   \hat{R}_{a}(\h,\bfgam,b^m,s) \le \hat{R}_{1}(\h,\bfgam^{a},b^m,s) \le
   \hat{C}_{R}(\alpha,a,\bfgam,\delta) b^{m(-1+\delta)}, \\
   m=1,2,\ldots,\ s=1,2,\ldots, \ a \ge 1,
\end{multline*}
for some constant $\hat{C}_{R}(\alpha,a,\bfgam,\delta)$.  Then
applying \eqref{Jens} completes the proof of part ii) of Theorem
\ref{Rthm}.

Lemma \ref{Haraldlem} also implies an upper bound on the
average value of $\hat{R}_{1}(\h,\bfgam,n,s)$:
$$
\int_{H_{b}^{\infty}} \hat{R}_{1}(\h,\bfgam,n,s) \ d\mu_{b}^{\infty}(\h)
\le M(\bfgam,n) \quad \text{for} \quad n=b, 
b^{2}, \ldots,
$$
where it is assumed that $\sum_{j=1}^{\infty} \gamma_{j} < \infty$ and
$M(\bfgam,n)$ is defined as
$$
M(\bfgam,n) := n^{-1} \prod_{j=1}^{\infty}[1+\gamma_{j} (\beta_{1} +
\beta_{2} \log n)].
$$
With $c_m$ as above, a set of bad $\h$ for a particular $n=b^{m}$ and $s$
is then defined as
\begin{multline*}
G_{bms} = \{ \h \in H_{b}^{\infty} : \hat{R}_{1}(\h,\bfgam,b^{m},s) >
c_{m} M(\bfgam,b^{m})\},\\
m=1, 2, \ldots, \ \ s=1, 2, \ldots .
\end{multline*}
By a similar argument to that above, $\mu_{b}^{\infty}(G_{bms}) \le
1/c_{m}$ for all $m$ and $s$.  Moreover, since
$\hat{R}_{1}(\h,\bfgam,b^{m},s)$ is a non-decreasing function of $s$,
it follows that $G_{bm1} \subseteq G_{bm2} \subseteq \cdots$.  Letting
$G_{bm} = \cup_{s=1}^{\infty} G_{bms}$, it follows that
$\mu_{b}^{\infty}(G_{bm}) \le 1/c_{m}$.  If $G_{b} =
\cup_{m=1}^{\infty} G_{bm}$, then $\mu_{b}^{\infty}(G_{b}) <
1$ by the choice of the $c_m$.  Thus, for all $\h \in H_{b}^{\infty} \setminus G_{b}$, it follows from
\eqref{mono} and by substituting $\bfgam^{a}$ for $\bfgam$ that
\begin{align*}
   \hat{R}_{a}(\h,\bfgam,b^m,s) & \le 
   \hat{R}_{1}(\h,\bfgam^{a},b^m,s) \le c_{m} M(\bfgam^{a},b^{m}) \\
   &= c_{0}(\epsilon) m
   [\log(m+1)]^{1+\epsilon} b^{-m} \prod_{j=1}^{\infty} [1 +
   \gamma_{j}^{a} (\beta_{1} + \beta_{2} m \log b ) ]
\end{align*}
for all $a \ge1$, $m \ge 1$, and $s \ge 1$.  Again using Lemma \ref{boundinfprd} and
\eqref{Jens}, it follows that if $\sum_{j=1}^{\infty} \gamma_{j}^{a} <
\infty$ for some $a \in [1,\alpha]$, then for any fixed $\delta>0$
there exists a $\hat{C}_{R}(\alpha,a,\bfgam,\delta)$ such that for
all $\h \in H_{b}^{\infty} \setminus G_{b}$,
\begin{multline*}
   \hat{R}_{\alpha}(\h,\bfgam,b^m,s) \le 
   [\hat{R}_{a}(\h,\bfgam,b^m,s)]^{\alpha/a} \le
   \hat{C}_{R}(\alpha,a,\bfgam,\delta) b^{m(-\alpha/a+\delta)}, \\
   m=1,2,\ldots,\ s=1,2,\ldots.
\end{multline*}
This completes the proof of part iii) of Theorem \ref{Rthm}.
\end{proof}

The following lemma was used in the above proof.  A similar result was 
proved in \cite{WanHic00b}.

\begin{lemma} \label{boundinfprd} Given a fixed $\tilde{\bfgam} \in
[0,\infty)^{\infty}$, let
$$
S(\tilde{\bfgam},m) = \prod_{j=1}^{\infty} [1 + \tilde{\gamma}_{j} m
], \quad m=1, 2, \ldots.
$$
If $\sum_{j=1}^{\infty} \tilde{\gamma}_{j} < \infty$, then for any
$\delta > 0$ it follows that $S(\tilde{\bfgam},m) \le
\tilde{C}(\tilde{\bfgam},\delta) b^{\delta m}$ for $m=1,2,\ldots$.
\end{lemma}
\begin{proof}
Let $\sigma_{d} = \sum_{j=d+1}^{\infty} \tilde{\gamma}_{j}, \ d=0, 1,
\ldots$.  Note that by increasing $d$ one may make $\sigma_{d}$
arbitrarily small. We can assume w.l.o.g. that all $\sigma_d > 0$.  Then applying some relatively elementary 
inequalities yields:
\begin{align*}
\log[S(\tilde{\bfgam},m)] & = 
\sum_{j=1}^{\infty} \log [1 + \tilde{\gamma}_{j} m ] \\
& \le \sum_{j=1}^{d} \log [1 + 
\sigma_{d}^{-1} + \tilde{\gamma}_{j}
m ] + \sum_{j=d+1}^{\infty} \log [1 + \tilde{\gamma}_{j}
m ] \\
& = d \log(1 + \sigma_{d}^{-1}) + \sum_{j=1}^{d}
\log [1 + \tilde{\gamma}_{j} m/(1 + \sigma_{d}^{-1}) ] \\
& \qquad \qquad +
\sum_{j=d+1}^{\infty} \log [1 + \tilde{\gamma}_{j} m] \\
& \le d \log(1 + \sigma_{d}^{-1}) + \sigma_{d}m 
\sum_{j=1}^{d}
\tilde{\gamma}_{j} +
m \sum_{j=d+1}^{\infty} \tilde{\gamma}_{j} \\
& \le d \log(1 + \sigma_{d}^{-1}) + \sigma_{d} 
(\sigma_{0}+1)m.\\
\end{align*}
Thus, it follows that
$$
S(\tilde{\bfgam},m) \le (1 + \sigma_{d}^{-1})^{d} b^{m\sigma_{d} 
(\sigma_{0}+1)/\log b}.
$$
Choosing $d$ large enough to make $\sigma_{d} \le \delta (\log
b)/(\sigma_{0}+1)$ completes the proof.
\end{proof}

Theorem \ref{Rthm} shows the existence of good extensible lattices
with $R_{1}(\h,\bfgam,n,1:s) = O(n^{-1} (\log n)^{s+1} [\log \log (n+1)]^{1+
\epsilon})$
for any $s$ with $n$ tending to infinity.  This is slightly worse
than the result of \cite[Theorem 5.10]{Nie92}, which shows the
existence of lattices with $R_{1}= O(n^{-1} (\log n)^{s})$ for fixed
$s$ and $n$, and the
lower bound of \cite{Lar87} of the same order.  Thus, the price for an
extensible lattice (in both $s$ and $n$) is an extra factor of the order $(\log n) [\log \log (n+1)]^{1+\epsilon}$.



\section{Existence of Extensible Lattices with Small $P_{\alpha}$}


Upper bounds on $P_{\alpha}$ follow from the upper bounds on
$R_{\alpha}$ derived in the previous section.  The theorem below
follows immediately from the following lemma, which is a minor
generalization of \cite[Theorem 5.5]{Nie92}.

\begin{lemma} \label{RPlem} For any integer $n \ge 2$, finite set $u
\subset 1:\infty$, $\bfgam \in [0,\infty)^{\infty}$, $\alpha > 1$, and $\h 
=(h_j)_{j=1}^{\infty} \in
\Z^{\infty}$ with $\gcd(h_{j},n)=1$ for all $j \in u$ we have
\begin{multline*}
    P_{\alpha}(\h,\bfgam,n,u) < R_{\alpha}(\h,\bfgam,n,u) -1 + \exp\left( 2 
    \zeta(\alpha)n^{-\alpha}  \sum_{j \in u} \gamma^{\alpha}_{j} 
    \right)  \\
    + n^{-1} \exp\left( 2 \zeta(\alpha) \sum_{j \in u}
    \gamma^{\alpha}_{j} \right) \\
    \times \left\{-1 + \exp\left[ 2^{\alpha}
    \zeta(\alpha)n^{1-\alpha} \sum_{j \in u} \gamma^{\alpha}_{j}/(1 +
    2 \gamma_{j}^{\alpha} \zeta(\alpha)) \right] \right\},
\end{multline*}
where $\zeta(\alpha) = \sum_{k=1}^{\infty} k^{-\alpha}$ is the 
Riemann zeta function.
\end{lemma}
\begin{proof} The proof here follows that in \cite[Theorem
5.5]{Nie92}.  Starting from \eqref{Pdef} the definition of
$P_{\alpha}$ may be written as a sum of several pieces:
\begin{align}
    P_{\alpha}(\h,\bfgam,n,u) & = -1 + \sum_{\l \in \tZu}
    \tilde{r}(n\l,\bfgam)^{-\alpha} + \sum_{\k \in
    \tilde{B}(\h,n,u)} \ \ \sum_{\l \in \tZu} \tilde{r}(\k+n
    \l,\bfgam)^{-\alpha} \nonumber \\
    & = -1 + \prod_{j \in u} \left\{ \sum_{l \in
    \Z} r(nl,\gamma_{j})^{-\alpha}  \right\} \nonumber \\
    & \qquad \qquad + \sum_{\k \in \tilde{B}(\h,n,u)} \ \
    \prod_{j \in u} \left\{ \sum_{l \in \Z}
    r(k_{j}+nl,\gamma_{j})^{-\alpha} \right\}. \label{Paderiv}
\end{align}
The second term in this equation may be simplified as
\begin{align*}
    \prod_{j \in u} \left\{ \sum_{l \in \Z}
    r(nl,\gamma_{j})^{-\alpha} \right\} &= \prod_{j \in u} [1 +
    2 \gamma_{j}^{\alpha} \zeta(\alpha) n^{-\alpha} ] \\
    &=\exp\left[ \sum_{j \in u} \log(1 + 2 \gamma_{j}^{\alpha} \zeta(\alpha)
    n^{-\alpha} ) \right ]\\
    & \le \exp\left( 2 \zeta(\alpha)n^{-\alpha}
    \sum_{j \in u} \gamma^{\alpha}_{j} \right).
\end{align*}

The inner sum in the last term in \eqref{Paderiv} may be written as
\begin{align*}
    \lefteqn{\sum_{l \in \Z} r(k_{j}+nl,\gamma_{j})^{-\alpha}} \\
    & = r(k_{j},\gamma_{j})^{-\alpha} + \gamma_{j}^{\alpha} \left[
    \sum_{l=1}^{\infty} (k_{j} + nl)^{-\alpha} + \sum_{l=1}^{\infty}
    (-k_{j} + nl)^{-\alpha} \right] \\
    & \le r(k_{j},\gamma_{j})^{-\alpha} + \gamma_{j}^{\alpha} \left[
    \sum_{l=1}^{\infty} (nl)^{-\alpha} + \sum_{l=1}^{\infty}
    (-n/2 + nl)^{-\alpha} \right] \\
    & = r(k_{j},\gamma_{j})^{-\alpha} + 2^{\alpha}\gamma_{j}^{\alpha} 
    n^{-\alpha} \left[
    \sum_{l=1}^{\infty} (2l)^{-\alpha} + \sum_{l=1}^{\infty}
    (2l-1)^{-\alpha} \right] \\
    & = r(k_{j},\gamma_{j})^{-\alpha} +
    2^{\alpha}\gamma_{j}^{\alpha} \zeta(\alpha) n^{-\alpha}.
\end{align*}
Thus, the last term in \eqref{Paderiv} satisfies 
\begin{align*}
    \lefteqn{\sum_{\k \in \tilde{B}(\h,n,u)} \ \
    \prod_{j \in u} \left\{ \sum_{l \in \Z}
    r(k_{j}+nl,\gamma_{j})^{-\alpha} \right\}} \\
    & \le \sum_{\k \in \tilde{B}(\h,n,u)} \ \ \prod_{j \in u} \left\{
    r(k_{j},\gamma_{j})^{-\alpha} + 2^{\alpha}\gamma_{j}^{\alpha}
    \zeta(\alpha) n^{-\alpha} \right\} \\
    & < R_{\alpha}(\h,\bfgam,n,u) \\
    & \qquad + \sum_{v \subset u} \left\{ \prod_{j \in u \setminus
    v}[2^{\alpha} \gamma_{j}^{\alpha}\zeta(\alpha)n^{-\alpha} ]
    \sum_{\substack{ \k \in \tZun \\ \k^{T} \h= 0 \mod
    n}} \ \ \prod_{j \in v} r(k_{j},\gamma_{j})^{-\alpha} \right\}.
\end{align*}
Using the same argument as in the proof of \cite[Theorem 5.5]{Nie92}, it follows that
$$
    \sum_{\substack{ \k \in \tZun \\ \k^{T} \h= 0 \mod
    n}} \ \ \prod_{j \in v} r(k_{j},\gamma_{j})^{-\alpha} <
    n^{\abs{u}-\abs{v}-1}\prod_{j \in v} [1 + 2 \gamma_{j}^{\alpha} \zeta(\alpha)].
$$
Thus, the last term in \eqref{Paderiv} has the following upper bound:
\begin{align*}
    \lefteqn{\sum_{\k \in \tilde{B}(\h,n,u)} \ \
    \prod_{j \in u} \left\{ \sum_{l \in \Z}
    r(k_{j}+nl,\gamma_{j})^{-\alpha} \right\}} \\
    & < R_{\alpha}(\h,\bfgam,n,u) + n^{-1}\sum_{v \subset u} \left\{
    \prod_{j \in u \setminus v}[2^{\alpha}
    \gamma_{j}^{\alpha}\zeta(\alpha)n^{1-\alpha} ] \prod_{j \in v} [1 + 2
    \gamma_{j}^{\alpha} \zeta(\alpha)] \right\} \\
    &= R_{\alpha}(\h,\bfgam,n,u) - n^{-1} \prod_{j \in u}[1 + 2
    \gamma_{j}^{\alpha} \zeta(\alpha)] \\
    & \qquad \qquad + n^{-1} \prod_{j \in u}[1 + 2
    \gamma_{j}^{\alpha} \zeta(\alpha) + 2^{\alpha}
    \gamma_{j}^{\alpha}\zeta(\alpha)n^{1-\alpha} ] \\
    &\le R_{\alpha}(\h,\bfgam,n,u) \\
    &\qquad \qquad + n^{-1} \exp\left( 2 \zeta(\alpha) \sum_{j \in u}
    \gamma^{\alpha}_{j} \right) \\
    & \qquad \qquad \times \left\{-1 + \exp\left[ 2^{\alpha}
    \zeta(\alpha)n^{1-\alpha} \sum_{j \in u} \gamma^{\alpha}_{j}/(1 +
    2 \gamma_{j}^{\alpha} \zeta(\alpha)) \right] \right\}.
\end{align*}
Combining this bound with that for the second term in \eqref{Paderiv} 
completes the proof of this lemma.
\end{proof}

The lemma above implies that $P_{\alpha}(\h,\bfgam,n,u) <
R_{\alpha}(\h,\bfgam,n,u) + O(n^{-\alpha})$, where the leading
constant in the $O$ term depends on $u$.  This relationship is uniform
in $u$ if $\sum_{j=1}^{\infty} \gamma^{\alpha}_{j} < \infty$.  Thus,
Theorem \ref{Rthm} and Lemma \ref{RPlem} imply the following existence
theorem for extensible lattices with good $P_{\alpha}$.

\begin{theorem} \label{Pthm} Theorem \ref{Rthm} also holds if one
replaces $R_{\alpha}$ by $P_{\alpha}$, allowing for a change of
constants as well.
\end{theorem}

For fixed $s$ the quantity $P_{\alpha}(\h,n,1:s)$ is known to have the
following lower and upper bounds if the generating vector is allowed
to depend on $n$ and $s$ \cite{Nie93a,Sha63}:
$$
C_{1}(\alpha,s) n^{-\alpha} (\log n)^{s-1} \le
P_{\alpha}(\h,n,1:s) \le C_{2}(\alpha,s,\epsilon) n^{-\alpha} (\log 
n)^{\alpha(s-1)+1 + \epsilon}.
$$
The lower bound holds for all $\h$ and the upper bound holds for suitable
$\h$. The price of an extensible lattice is that the upper bound has now
increased by roughly a factor of $(\log n)^{2\alpha -1}$.  Equation
\eqref{ntoinfty} applied to $P_{\alpha}$ improves upon \cite[Theorem
2.1]{SloRez01} because the right-hand side decays more quickly with $n$,
and the same $\h$ works for all $n=b^{m}$ as well as all $s$. 
Equation \eqref{nstoinfty} applied to $P_{\alpha}$ improves upon
\cite[Theorem 3]{SloWoz00a} because the same $\h$ works for all
$n=b^{m}$ and all $s$.


\section{Existence of Extensible Lattices with 
Small Discrepancy}

The unanchored discrepancy defined in \eqref{undiscdef} is related to
$R_{1}$ \cite[Chapter 5]{Nie92}.  The following lemma, a slight
generalization of \cite[Theorem 3.10 and Theorem 5.6]{Nie92}, makes
this relationship explicit.  

\begin{lemma} \label{RDlem} For any integer $n=b^{m}, \ m=1,2,\ldots$,
any finite subset $u \subset 1:\infty$, any $\h \in \Z^{\infty}$, and
any corresponding node set of a shifted lattice, $\{\z_{i} = \bbrace{\h
\Phi_{b}(i) + \bfDelta} : i=0,1, \ldots, b^{m}-1\}$, the unanchored
$\Linf$-discrepancy of these points projected into the coordinates
indexed by $u$ has the following upper bound:
\begin{equation*}
    \sup_{\zero \le \y_{u} \le \x_{u} \le \one} \abs{
    \disc_{u}(\y,\x, \{\z_{i}\},n)} \le 1 - (1-1/n)^{\abs{u}} + \frac 12
    R_{1}(\h,\one,n,u).
\end{equation*}
\end{lemma}
\begin{proof} It suffices to note that the proof of \cite[Theorem 3.10]{Nie92} works also for the discrepancy extended over all intervals modulo 1 and 
that this discrepancy is invariant under shifts of the node set modulo 1.
\end{proof}
%\begin{proof} For any $\l \in \tZu$, let $A(\l)$ be the number of
%$i=0, \ldots, n-1$ with $\z_{iu}=\bbrace{\l_{u}/n +\bfDelta_{u}}$. 
%This quantity may be written as
%\begin{align*}
%A(\l) &= \frac 1{n^{\abs{u}}} \sum_{i=0}^{n-1} \sum_{\k \in
%\tZun} e^{2 \pi \imath \k'(\z_{i} - \l/n - \bfDelta)} \\
%&= \frac{1}{n^{\abs{u}-1}} + \frac 1{n^{\abs{u}}} \sum_{i=0}^{n-1} \sum_{\k %\in
%\tZun - \{\zero\}} e^{2 \pi \imath \k'(\z_{i} - \l/n - \bfDelta)},
%\end{align*}
%since the inner sum in the first line is $n^{s}$ if $\z_{iu}=\bbrace{\l_{u}/%n
%+\bfDelta_{u}}$ and zero otherwise.  For any an interval
%$[\y_{u}, \x_{u})$, let $[\bflam_{u}/n+\bfDelta,\bfnu_{u}/n+\bfDelta]$
%be the largest interval contained in $[\y_{u}, \x_{u})$ where $\bflam 
%\le \bfnu$ are integer vectors.  If no such interval exists, then
%$$
%\abs{ \disc_{u}(\y,\x, \{\z_{i}\},n)} = \prod_{j \in u} (x_{j} - 
% y_{j}) < \frac 1n.
%$$
%
%Otherwise, by the definition of the discrepancy function, 
%\eqref{discfundef}
%\begin{align*}
%   \lefteqn{\disc_{u}(\y,\x, \{\z_{i}\},n)} \\
%   & = \prod_{j \in u} (x_{j} - y_{j}) - \frac 1n \sum_{i=1}^{n}
%   \prod_{j \in u} (1_{(z_{j},\infty)}(x_{j}) -
%   1_{(z_{j},\infty)}(y_{j})) \\
%& =   \prod_{j \in u} (x_{j} - y_{j}) - \prod_{j \in u} \left(\frac {\nu_{j}% - 
%\lambda_{j}+1}{n} \right )  - \sum_{\l_{u} \in 
%[\bflam_{u},\bfnu_{u}]} \left( \frac{A(\l)}{n} - \frac 1{n^{\abs{u}}} 
%\right) \\
%& =   \prod_{j \in u} (x_{j} - y_{j}) - \prod_{j \in u} \left(\frac {\nu_{j}% - 
%\lambda_{j}+1}{n} \right )    \\
%&\qquad \qquad - \frac 1{n^{\abs{u}}} \sum_{\k \in \tZun - \{\zero\}} \left[%\sum_{\l_{u} \in
%[\bflam_{u},\bfnu_{u}]} e^{-2 \pi \imath \k'\l/n} \right]  \left[\frac 1n 
%\sum_{i=0}^{n-1}  e^{2 \pi \imath \k'(\z_{i} - \bfDelta)}   \right].
%\end{align*}
%Since $\abs{(x_{j} - y_{j}) - (\nu_{j} - \lambda_{j}+1)/n} < 1/n$, the
%first two terms in the right hand side of the above equation are
%bounded by $1 - (1-1/n)^{\abs{u}}$ by Lemma \ref{diffprodlem} below.  Moreov%er, the sum over $i$ is unity if $\k \in
%\tilde{B}(\h,n,u)$ and zero otherwise.  The geometric sum over $\l_u$
%has absolute value
%$$
%\abs{\sum_{\l_{u} \in [\bflam_{u},\bfnu_{u}]} e^{-2 \pi \imath 
%\k'\l/n}}  =
%\prod_{j \in u} \abs{ \sum_{\lambda_{j} \le l_{j}  \le \nu_{j}} e^{-2 \pi
%\imath k_{j}l_{j}/n} }.
%$$
%For $k_{j} = 0$ the sum can be reduced to 
%$$
%\abs{ \sum_{\lambda_{j} \le l_{j}  \le \nu_{j}} e^{-2 \pi
%\imath k_{j}l_{j}/n} }
% = \nu_{j} - \lambda_{j} + 1 \le n = n/r(k_{j},1),
%$$
%and for $0 < \abs{k_{j}} \le n/2$ the sum can be reduced to
%\begin{align*}
%\abs{ \sum_{\lambda_{j} \le l_{j}  \le \nu_{j}} e^{-2 \pi
%\imath k_{j}l_{j}/n} }
%& = \abs{ \frac{e^{-2 \pi \imath
%k_{j}(\nu_{j}-\lambda_{j}+1)/n} - 1}{e^{-2 \pi \imath k_{j}/n} - 1} } 
%\\
%&= \abs{ \frac{\sin(\pi k_{j}(\nu_{j}-\lambda_{j}+1)/n)}{\sin(\pi
%k_{j}/n)}} \le \frac{1}{\sin(\pi \abs{k_{j}}/n)} \\
%& \le \frac {n}{2 r(k_{j},1)}
%\end{align*}
%since $\sin(\pi\theta) \ge 2\theta$ for $0 \le t \le 1/2$. 
%Substituting these bounds into the expression for the discrepancy
%function yields
%\begin{align*}
% \abs{\disc_{u}(\y,\x, \{\z_{i}\},n)}& \le 1 - (1-1/n)^{\abs{u}} +
% \frac 1{n^{\abs{u}}} \sum_{\k \in \tilde{B}(\h,n,u)}
% \frac{n^{\abs{u}}}{2 r(k_{j},1)} \\
%&= 1 - (1-1/n)^{\abs{u}}   + \frac 12 R_{1}(\h,\one,n,u).
%\end{align*}
%Since this bound holds for all $\y$ and $\x$, the theorem is proved.
%\end{proof}
%
%\begin{lemma}\label{diffprodlem} \cite[Lemma 3.9]{Nie92} If $\bfsigma,
%\bftau \in [0,1]^{s}$, then
%    $$
%    \abs{\prod_{j=1}^{s} \sigma_{j} - \prod_{j=1}^{s}\tau_{j}} \le 1 - (1- 
%    \sup_{j}\abs{\tau_{j}-\sigma_{j}}).
%    $$
%\end{lemma}

Lemma \ref{RDlem} implies the following theorem for the existence of
extensible lattice rules with small discrepancy.

\begin{theorem} \label{Dthm} Suppose we are given a fixed integer $b \ge
2$, a fixed $\bfgam \in [0,\infty)^{\infty}$, 
and a fixed $\epsilon > 0$.  

i) There exist a $\mu_b^{\infty}$-measurable $\tilde{G}_{b} \subset H_{b}^{\infty}$ and some constant
$C_{D}(\bfgam,\epsilon,s)$ such that for all $\h \in
H_{b}^{\infty} \setminus \tilde{G}_{b}$ and all $\bfDelta \in \Cinf$, the node set
of the shifted lattice given in \eqref{nodeset} satisfies
\begin{multline} \label{Dntoinfty}
    D^{*}_{q}(\{\z_{i}\},\bfgam,n,1:s), D_{q}(\{\z_{i}\},\bfgam,n,1:s) \\ \le
    C_{D}(\bfgam,\epsilon,s) n^{-1} (\log n)^{s+1} [\log \log (
    n+1)]^{1+\epsilon}, \\ n=b, b^{2},\ldots,\ s=1,2,\ldots,\ 1 \le q \le \infty.
\end{multline}
Furthermore, one may make $\mu_{b}^{\infty}(\tilde{G}_{b})$ arbitrarily close to 
zero by choosing  $C_{D}(\bfgam,\epsilon,s)$ large enough.

ii) If $\sum_{j=1}^{\infty} \gamma_{j} j [\log(j+1)]^{1+\epsilon} < \infty$, then for
any fixed $\delta>0$ there exists some
$\tilde{C}_{D}(\bfgam,\delta)$ such that for all $\h \in
H_{b}^{\infty} \setminus \tilde{G}_{b}$ and all $\bfDelta \in \Cinf$, the node set 
of the resulting shifted lattice satisfies
\begin{multline}\label{Dnstoinfty}
D^{*}_{q}(\{\z_{i}\},\bfgam,n,1:s), D_{q}(\{\z_{i}\},\bfgam,n,1:s) \le
\tilde{C}_{D}(\bfgam,\delta) n^{-1+\delta}, \\ 
n=b,b^{2},\ldots,\ s=1,2,\ldots,\ 1 \le q \le \infty.
\end{multline}
Again one may make $\mu_{b}^{\infty}(\tilde{G}_{b})$ arbitrarily close to 
zero by choosing the above leading constant large enough.

iii) If $\sum_{j=1}^{\infty} \gamma_{j} < \infty$, then for any fixed
$\delta>0$ there exists a $\mu_b^{\infty}$-measurable $G_{b} \subset H_{b}^{\infty}$ such that
for all $\h \in H_{b}^{\infty} \setminus G_{b}$ and all $\bfDelta \in
\Cinf$ \eqref{Dnstoinfty} is satisfied, but \eqref{Dntoinfty} may not
be.  Here too, $\mu_{b}^{\infty}(G_{b})$ can be made arbitrarily close
to zero.
\end{theorem}
\begin{proof} From \eqref{discbds} and Lemma \ref{RDlem} it follows 
that the star and unanchored discrepancies have the following upper bound:
\begin{align*}
     \lefteqn{D^{*}_{q}(\{\z_{i}\},\bfgam,n,1:s),
     D_{q}(\{\z_{i}\},\bfgam,n,1:s) } \\
     & \le \sum_{\emptyset \subset u
    \subseteq 1:s} \gamma_{u} \sup_{\zero \le \y_{u} \le \x_{u} \le
    \one} \abs{ \disc_{u}(\y,\x, \{\z_{i}\},n)} \\
    & \le \sum_{\emptyset \subset u \subseteq 1:s} \gamma_{u}\left[1 -
    (1-1/n)^{\abs{u}} + \frac 12 R_{1}(\h,\one,n,u) \right] \\
    &= \prod_{j=1}^{s} (1 + 
    \gamma_{j}) - \prod_{j=1}^{s} [1 + 
    \gamma_{j}(1-1/n)]+ \frac 12 \hat{R}_{1}(\h,\bfgam,n,s) \\
   &= \prod_{j=1}^{s} (1 + \gamma_{j})\left [ 1 - \prod_{j=1}^{s}
   \left(1 - \frac {\gamma_{j}}{n(1+\gamma_{j})}\right) \right] +
   \frac 12 \hat{R}_{1}(\h,\bfgam,n,s),
\end{align*}
where $\hat{R}$ was defined in \eqref{Rhatdef}.  Using the fact that
$\log(1-x) \ge x (\log(1- a))/a$ for $0 \le x \le a < 1$, we obtain a
lower bound on the second product:
\begin{align*}
   \log \left(\prod_{j=1}^{s} \left(1 - \frac
   {\gamma_{j}}{n(1+\gamma_{j})}\right) \right) &= \sum_{j=1}^{s} \log
   \left(1 - \frac {\gamma_{j}}{n(1+\gamma_{j})}\right) \\
   &\ge \log(1-1/n) \sum_{j=1}^{s} \frac {
   \gamma_{j}}{1+\gamma_{j}}.
\end{align*}
Thus,
\begin{multline*}
     D^{*}_{q}(\{\z_{i}\},\bfgam,n,1:s),
     D_{q}(\{\z_{i}\},\bfgam,n,1:s) \\
\le  \prod_{j=1}^{s} (1 +
     \gamma_{j}) \left [1 - \exp\left( \log(1-1/n)
     \sum_{j=1}^{s} \frac { \gamma_{j}}{1+\gamma_{j}} \right) \right] +
     \frac 12 \hat{R}_{1}(\h,\bfgam,n,s) \\
     =  \prod_{j=1}^{s} (1 +
     \gamma_{j}) \left [1 - \left( 1- \frac 1n \right)
     ^{\sum_{j=1}^{s} \gamma_{j}/(1+\gamma_{j})} \right] +
     \frac 12 \hat{R}_{1}(\h,\bfgam,n,s).
\end{multline*}

Since the star and unanchored discrepancies are both bounded above by
$\hat{R}_{1}(\h,\bfgam,n,s)/2$ plus a term that is $O(n^{-1})$
uniformly in $s$ assuming that the $\gamma_{j}$ are summable, the
conclusions of this theorem follow immediately from the proof of
Theorem \ref{Rthm}.
\end{proof}

\section*{Acknowledgements} The authors would like to thank the
participants in the 2001 Oberwolfach workshop on Numerical Integration
and Its Complexity for their helpful comments.  We specially thank Greg
Wasilkowski for useful discussions.

\bibliographystyle{amsplain}
\bibliography{FJH23,FJHown23}

\end{document}
