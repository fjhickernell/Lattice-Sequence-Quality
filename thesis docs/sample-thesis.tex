
%%%%%%%%%%%%%%%%%%%%%%%%%%%%%%%%%%%%%%%%%%%%%%%%%%%%%%%%%%%%%%%%%%%%%%%%%%%%%%%
%
% IIT Sample THESIS File,   Version 3, Updated by Babak Hamidian on 11/18/2003
%
%%%%%%%%%%%%%%%%%%%%%%%%%%%%%%%%%%%%%%%%%%%%%%%%%%%%%%%%%%%%%%%%%%%%%%%%%%%%%%%
% File: sample3.tex                                   %
% IIT Sample LaTeX File                               %
% by Ozlem Kalinli on 05/30/2003                      %
% Revised by Babak Hamidian on 11/18/2003             %
%%%%%%%%%%%%%%%%%%%%%%%%%%%%%%%%%%%%%%%%%%%%%%%%%%%%%%%
%                                                     %
% This is a sample thesis document created using      %
% iitthesis.cls style file. The PDF output is also    %
% available for your reference. In this file, it has  %
% been illustrated how to make table of contents,     %
% list of tables, list of figures, list of symbols,   %
% bibliography, equations, enumerations, etc.         %
% You can find detailed instructions                  %
% for using the style file in Help.doc,               %
% TableHelp.doc, FigureHelp.doc, and                  %
% Bibliography.doc files.                             %
%                                                     %
%%%%%%%%%%%%%%%%%%%%%%%%%%%%%%%%%%%%%%%%%%%%%%%%%%%%%%%
% Note: The texts that are used in this sample3.tex   %
% file are irrelevant. They are just used to show     %
% you the style created by iitthesis style file.      %
%%%%%%%%%%%%%%%%%%%%%%%%%%%%%%%%%%%%%%%%%%%%%%%%%%%%%%%
%%% Updates added by Jonathan Harmon, effective Dec. 15. 2021. 

\documentclass{iitthesis-au} %current document class

\usepackage{graphicx}    % This package is used for Figures
\usepackage{rotating}           % This package is used for landscape mode.
\usepackage{epsfig}
\usepackage{subfigure}          % These two packages, epsfig and subfigure, are used for creating subplots.
% Packages are explained in the Help document.


\usepackage{mathtools,upref,siunitx,upquote,fancyvrb,bbm,xspace,color,amsmath,amssymb, bm}
\usepackage[hyphens]{url}
\usepackage[utf8]{inputenc}
\usepackage{esdiff}
\usepackage{graphicx}
\usepackage{xcolor}

\input{FJHDef.tex}
\usepackage{algpseudocode}
\usepackage{algorithm, algorithmicx}
\algnewcommand\algorithmicparam{\textbf{Parameters:}}
\algnewcommand\PARAM{\item[\algorithmicparam]}
\algnewcommand\algorithmicinput{\textbf{Input:}}
\algnewcommand\INPUT{\item[\algorithmicinput]}
\algnewcommand\RETURN{\State \textbf{Return }}

\newcommand{\tr}{\widetilde{r}}
\newcommand{\appxintn}{\appxint_n}
\DeclareMathOperator{\appxint}{\hat{I}}
\DeclareMathOperator{\trun}{trunc}
\newcommand{\onetos}{1\!:\!s}

\newcommand{\FredNote}[1]{{\color{blue}#1}}

\newcommand{\LarysaNote}[1]{{\color{violet}#1}}
\begin{document}


%%% Declarations for Title Page %%%
\title{The Quality of Lattice Sequences}
\author{Larysa Matiukha}
\degree{Master of Science}
\dept{Applied Mathematics}
\date{May 2025}
\copyrightnoticetrue      % create copyright page or not
% \coadvisortrue           % add co-advisor. activate it by removing % symbol to add co-advisor
\maketitle                % create title and copyright pages


\prelimpages         % Settings of preliminary pages are done with \prelimpages command


%%%  Acknowledgement %%%
\begin{acknowledgment}     % acknowledgement environment, this is optional
%\vspace{10pt}	%These two lines will single space the ACK if needed
%\ssp
\par  This dissertation could not have been written without Dr. X
who not only served as my supervisor but also encouraged and
challenged me throughout my academic program. He and the other
faculty members, Dr. Y and Dr. Z, guided me through the
dissertation process, never accepting less than my best efforts. I
thank them all.\\ \\ (Don't copy this sample text. Write your own
acknowledgment.)

% or \input{acknowledgement.tex} % you need a separate acknowledgement.tex file to include it.

\end{acknowledgment}

\begin{authorship}
	\par I, Larysa Matiukha, attest that the work in this thesis is substantially my own.	
	
	In accordance with the disciplinary norm of [your major] (see IIT Faculty Handbook, Appendix S), the following collaborations occurred in the thesis: 
	
	
	%Be sure to include the parenthetical reference to the Faculty Handbook. The Handbook can be found here: https://web.iit.edu/general-counsel/faculty-handbook. Appendix S contains references for "disciplinary norms" of authorship by discipline.	
\end{authorship}

% Table of Contents
\tableofcontents
\clearpage

% List of Tables
\listoftables

\clearpage

%List of Figures
\listoffigures

\clearpage

%List of Symbols(optional)

 \clearpage



%%% Abstract %%%
\begin{abstract}           % abstract environment, this is optional
\par Your Abstract goes here! 
% or \input{abstract.tex}  %you need a separate abstract.tex file to include it.

\end{abstract}
%\clearpage




\textpages     % Settings of text-pages are done with \textpages command

% % Chapters are created with 
% \Chapter{title} command


% \maketitle
\Chapter{Introduction}
%\Section{Introduction}



Multidimensional integrals arise in various fields and practical applications such as quantitative finance \cite{CafMorOwe97}, statistics \cite{Gen92,GenBre99}, and physics \cite{PapTra97}. 


\FredNote{Often these arise as expectations of random variables.}  However, many such integrals cannot be computed analytically and, thus, numerical methods are often utilized. \FredNote{A common approach to approximating an integral that represents a population mean is by a sample mean:} 
\[
\mathbb{E}[f(\underbrace{\vX}_{\sim \mathcal{U}[0,1]^d})] : = \int_{[0,1)^s} f(\vx) \, \dif \vx \approx \frac{1}{n} \sum_{i=0}^{n-1} f(\vz_i) =:\appxintn(f),
\]
\cite{DicEtal22a,Nie92,SloJoe94}.
The nodeset, $\{\vz_0, \vz_1, \ldots \}$, is chosen as independent and identically distributed (IID), which corresponds to simple Monte Carlo. However, the nodeset may be chosen to be more evenly distributed, which corresponds to quasi-Monte Carlo methods. Lattices are a popular choice of nodes $\{\vz_i\}_{i=0}^{\infty} \in [0,1)^s$ for such approximations. \\

% what we are doing 
In this work we derive the upper bounds on the figure of merit $P_{\alpha}$ for extensible lattices with arbitrary cardinality $n$. We consider a case of an arbitrary positive $n$ and the case of $n = \lambda b^p$, where $\lambda$ is a odd integer. For arbitrary n, we found that $P_{\alpha}$ decays like $\Order(n^{-1})$, and in the case of $n = \lambda b^p$, $P_{\alpha}$ decays like $\Order(b^{-p \alpha})$, which can be faster when $\lambda$ is small. Hickernell and Niederreiter \cite{HicNie03a} established the existence of extensible lattices with good generating vectors for the preferred values $n = b, b^2, \ldots$.  Our purpose is to extend these results to all positive $n$, recognizing that the upper bounds on the figures of merit will be somewhat worse for general $n$ than for powers of the base. \\



   
In the next chapter, we give more background on extensible lattices. In chapter 3 we go over the figures of merit and the worst case error analysis for arbitrary $n$ for integrands that have absolutely summable Fourier series. In chapter 4 we derive upper bounds on the figure of merit % P_alpha 
for $n$ other than $b^m$, and in chapter 5 we generalize the results to other spaces of integrands and error measures. 


\begin{figure}[h]
\centering
\includegraphics[width=5cm]{lattice-plot.png}
\caption{Two-dimensional Shifted Lattice}
\label{fig:enter-label}
\end{figure}

\Chapter{Background}

Historically, lattice points were initially constructed as sets with fixed cardinality, $n$, and took the form
\begin{equation} \label{eq:lat}
    \{\vz_i = i \vh/n \bmod{\vone} : i=0,1, \ldots, n-1 \} \in [0,1)^s,
\end{equation}
where $\vh \in \{1, \ldots, n-1\}^s$ is the \emph{generating vector}.  A (random) shift, $\vDelta \in [0,1)^s$, is often added:
\begin{equation} \label{eq:shlat}
    \{\vz_i = i \vh/n + \vDelta \pmod{\vone} : i=0,1, \ldots, n-1 \} \in [0,1)^s.
\end{equation}



% \cite{HicNie03a} established the existence of extensible lattices with good generating vectors for the preferred values $n = b, b^2, \ldots$.  The purpose of this thesis is to extend these results to all positive $n$, recognizing that the upper bounds on the figures of merit will be somewhat worse for general $n$ than for powers of the base. 

% In the next Chapter, we give more background on extensible lattices. In Section 2 we go over the figures of merit and the worst case error analysis for arbitrary $n$ for integrands that have absolutely summable Fourier series. In Section 3 we derive upper bounds on the figure of merit % P_alpha 
% for $n$ other than $b^m$, and in Section 4 we generalize the results to other spaces of integrands and error measures. 














% Lattices are a popular choice of nodes $\{\vz_i\}_{i=0}^{\infty} \in [0,1)^s$ for approximating multidimensional integrals by a sample mean,
% \[
% \int_{[0,1)^s} f(\vx) \, \dif \vx \approx \frac{1}{n} \sum_{i=0}^{n-1} f(\vz_i) =:\appxintn(f),
% \]
% \citep{DicEtal22a,Nie92,SloJoe94}.



% \cite{HicNie03a} established the existence of extensible lattices with good generating vectors for the preferred values $n = b, b^2, \ldots$.  The purpose of this thesis is to extend these results to all positive $n$, recognizing that the upper bounds on the figures of merit will be somewhat worse for general $n$ than for powers of the base. 

% \begin{figure}[h]
% \centering
% \includegraphics[width=5cm,trim={0 0 0 7.5mm},clip]{shifted-lattice}
% \caption{Two-dimensional Shifted Lattice}
% \label{fig:enter-label}
% \end{figure}

% Historically, lattice points were initially constructed as sets with fixed cardinality, $n$, and took the form
% \begin{equation} \label{eq:lat}
%     \{\vz_i = i \vh/n \bmod{\vone} : i=0,1, \ldots, n-1 \} \in [0,1)^s,
% \end{equation}
% where $\vh \in \{1, \ldots, n-1\}^s$ is the \emph{generating vector}.  A (random) shift, $\vDelta \in [0,1)^s$, is often added:
% \begin{equation} \label{eq:shlat}
%     \{\vz_i = i \vh/n + \vDelta \pmod{\vone} : i=0,1, \ldots, n-1 \} \in [0,1)^s.
% \end{equation}




Extensible lattice sequences were proposed by \cite{HicEtal00,Mai81a} and take the form
\begin{equation} \label{eq:extlat}
    \{\vz_i = \vh\phi(i)+ \vDelta \pmod{\vone} : i=0,1, \ldots \} \in [0,1)^s.
\end{equation}
where $\{\phi(\cdot)\}_{i=0}^\infty$ is the van der Corput sequence in base $b$.  In this case $\vh$ must be a generalized integer as defined in \cite[Section 2]{HicNie03a}.

The van der Corput sequence is defined as: 
\[
\phi((\cdots i_2 i_1 i_0)_b) = {}_b0.i_0 i_1 i_2 \cdots.
\]
For example, for $b=2$,
\[
\phi(6) = \phi(110_2) = {}_20.011 = \frac 38.
\]
Note that the first ${b^m}$ points of the van der Corput sequence are just equally spaced points reordered. 
\begin{equation} \label{eq:phipropone}
\{ \phi(i) : i = 0, \ldots, b^m-1 \} = \{0, b^{-m}, 2\times b^{-m}, \ldots, 1 - b^{-m} \}.
\end{equation}
Also note that
\begin{multline} \label{eq:phiproptwo}
\{ \phi(i) : i = \lambda \times b^m , \ldots, (\lambda+1)b^m-1 \} \\
= \{\phi(\lambda \times b^m) + 0, \phi(\lambda \times b^m) + b^{-m}, \ldots, \phi(\lambda \times b^m) + 1 - b^{-m} \} , \\
\lambda \in \natzero.
\end{multline}


\LarysaNote{Hickernell et al.} showed that if a lattice sequence in base $b$ has the worst case error of $C(b^m)^{-\alpha}$ in a reproducing kernel Hilbert space, then the worst case error of using the first $n \geq 2$ points of the lattice sequence is $ C \frac{\log_bn}{n^{\min(\alpha,1)}}$, for some constants $C, \alpha > 0$ \cite [Corollary 1]{HicEtal10a}. 
We provide a refinement of this result (\LarysaNote{no extra log term}.)

\Chapter{Worst case error analysis for arbitrary $n$}
We first consider integrands, $f$, that have an absolutely summable Fourier series:
\begin{equation} \label{eq:fseries}
    f(\vx) = \sum_{\vk \in \integers^{s}} \tf(\vk) \me^{2 \pi \sqrt{-1} \vk^T \vx}, \qquad \text{where } \tf(\vk) = \int_{[0,1)^{\infty}} f(\vx) \me^{-2 \pi \sqrt{-1} \vk^T \vx}\, \dif \vx
\end{equation}
Define the weights
\begin{equation}
\tr(\vk,\vgamma) = \prod_{j=1}^{s} r(k_{j},\gamma_{j}),
\qquad \text{where } r(k_{j},\gamma_{j})=\begin{cases} 1, &
k_{j}=0, \\ \gamma_{j}^{-1}\abs{k_{j}}, & k_{j} \ne 0.  \end{cases}
\end{equation}
Define a Banach
space of functions:
$$
\cf_{\alpha} = \{ f \in \cl_2[0,1)^{s} :
\norm[\cf{\alpha}]{f} < \infty \}, \qquad
\norm[\cf{\alpha}]{f} := \sup_{\vk \in \integers^{s}}
\left(\tr(\vk,\vgamma)^{\alpha} \abs{\tilde{f}(\vk)} \right).
$$



It then follows that the error in approximating the integral by the sample mean is
\begin{align} \label{eq:wcerrPalpha}
\nonumber
\abs{\int_{[0,1)^{s}} f(\vx) \, \dif \vx - \appxint_n(f)} &
= \abs{\sum_{\vk \in \integers^{s}} \tf(\vk) \left[\int_{[0,1)^{s}} \me^{2 \pi \sqrt{-1} \vk^T \vx} \, \dif \vx - \appxintn(\me^{2 \pi \sqrt{-1} \vk^T \cdot})\right]} \\
\nonumber
& = \abs{\sum_{\vk \in \integers^s \setminus\{\vzero\}} \tf(\vk) \left[ \appxintn(\me^{2 \pi \sqrt{-1} \vk^T \cdot})\right]} \\
& \le \norm[\cf{\alpha}]{f} \underbrace{\sum_{\vk \in \integers^s \setminus\{\vzero\}} \abs{\appxint_n(\me^{2 \pi \sqrt{-1} \vk^T \cdot})}\tr(\vk,\vgamma)^{-\alpha}}_{=: P_\alpha(\vh,\vgamma,n,1:s) = \text{quality of the lattice nodes}},  %\qquad \forall f \in \cf_{\alpha} \\
\end{align}
where $\onetos$ means $\{1, \ldots, s\}$.
This the same as \cite[(4)]{HicNie03a} but we do not assume the first $n$ points are a lattice. 

For $n = b^m$, the quality measure $P_\alpha(\vh,\vgamma,n,\onetos)$ takes on a simple form.  Note that
\begin{align}\label{Eq:DualLattice}
    \MoveEqLeft{\appxint_{b^m}(\me^{2 \pi \sqrt{-1} \vk^T \cdot})} \\
    \nonumber
    &= \frac 1{b^m} \sum_{i=0}^{b^m-1} \me^{2 \pi \sqrt{-1} \vk^T (\vh\phi(i) + \vDelta \bmod{\vone})} \\
    \nonumber
    &= \frac {\me^{2 \pi \sqrt{-1} \vk^T \vDelta}}{b^m} \sum_{i=0}^{b^m-1} \me^{2 \pi \sqrt{-1} \vk^T \vh\phi(i)} \\
    \nonumber
    &= \frac {\me^{2 \pi \sqrt{-1} \vk^T \vDelta}}{b^m} \sum_{i=0}^{b^m-1} \me^{2 \pi \sqrt{-1} \vk^T \vh i/b^m} \qquad \text{by \eqref{eq:phipropone}}\\
    \nonumber
    &= \frac {\me^{2 \pi \sqrt{-1} \vk^T \vDelta}}{b^m} \times
    \begin{cases}
    \frac{\me^{2 \pi \sqrt{-1} \vk^T \vh} - 1}{\me^{2 \pi \sqrt{-1} \vk^T \vh/b^m} - 1} = 0, & \vk^T \vh \pmod{b^m} \ne 0\\
    b^m, & \vk^T \vh \pmod{b^m} = 0
    \end{cases} \\
    \nonumber
    & = \me^{2 \pi \sqrt{-1} \vk^T \vDelta} \bbone_{B(\vh,m,1:s)}(\vk)
\end{align}
Thus it follows that
\begin{equation} \label{eq:Palphadual}
    P_\alpha(\vh,\vgamma,b^m,\onetos) = \sum_{\vk \in B(\vh,m,1:s)} \tr(\vk,\vgamma)^{-\alpha},
\end{equation}
where $B(\vh,m,\onetos) : = \{\vk \in  \integers^s \setminus \{\vzero\} : \vk^T \vh \pmod{b^m} = 0\}$ is called the \emph{dual lattice}.  This corresponds to \cite[Equation (3)]{HicNie03a}, and we know by \cite[Theorem 5]{HicNie03a} that  there exists $\vh$ with
\begin{multline} \label{eq:Niedbd}
    P_{\alpha}(\vh,\vgamma,b^m,\onetos) \le C_{R}(\alpha,\vgamma,\epsilon,s)
    b^{-m\alpha} (\log b^{m})^{\alpha(s+1)} [\log \log (
    b^m+1)]^{\alpha(1+\epsilon)}, \\ m = 1, 2,\ldots, \quad \alpha \ge 1.
\end{multline}

%Note that
This result can be extended to the case when $m = 0$
\begin{align*}
    P_{\alpha}(\vh,\vgamma,1,\onetos) & = \sum_{\vk \ne \vzero} \tr(\vk, \vgamma)^{-\alpha} = -1 + \sum_{\vk \in \integers^s} \tr(\vk,\vgamma)^{-\alpha} \\
    & = -1 + \sum_{k_1 =-\infty}^{\infty} \cdots \sum_{k_s  =-\infty}^{\infty} \frac{1}{\prod_{j=1}^s[\max(1,\gamma_j^{-1}\abs{k_j})]^\alpha} \\
    & = -1 + \sum_{k_1 =-\infty}^{\infty} \frac{1}{[\max(1,\gamma_1^{-1}\abs{k_1})]^\alpha} \cdots \sum_{k_s  =-\infty}^{\infty} \frac{1}{[\max(1,\gamma_s^{-1}\abs{k_s})]^\alpha} \\
    & = -1 + \left[\sum_{k =-\infty}^{\infty} \frac{1}{[\max(1,\gamma^{-1}\abs{k})]^\alpha} \right]^d \\
    & = -1 + \left[1 + 2 \gamma^{\alpha} \sum_{k =1}^{\infty} \frac{1}{k^\alpha} \right]^d \\
    & = -1 + \left[1 + 2 \gamma^{\alpha}\zeta(\alpha) \right]^d,
\end{align*}

where $\zeta(\alpha)$ is the Riemann zeta function. \\

Therefore, the above formula \eqref{eq:Niedbd} can be extended to the case of $m=0$ as follows:
\begin{multline} \label{eq:Palphaextm}
    P_{\alpha}(\vh,\vgamma,b^m,\onetos) \\
    \le C_{R}(\alpha,\vgamma,\epsilon,s)
    b^{-m\alpha}\max(1, (\log b^{m})^{\alpha(s+1)})[\max(1,\log \log (
    b^m+1))]^{\alpha(1+\epsilon)}, \\ m =0, 1, 2,\ldots, \quad \alpha \ge 1.
\end{multline}


% \Subsection{Randomized Error Analysis}
% Another approach is to assume that $f$ in \eqref{eq:fseries} is a random function (stochastic process) where
% \begin{equation}
%     \abs{\tf(\vk)} \IIDsim \cn(0,\tr(\vk)^{-2\alpha}).
% \end{equation}
% Then it follows that
% \begin{align*}
%     \MoveEqLeft{\Ex_f\left[\left(\int_{[0,1)^d} f(\vx) \, \dif \vx - \appxintn(f) \right)^2\right]} \\
%     & =
%     \Ex_f\left[\left(\sum_{\vk \in \integers^s \setminus\{\vzero\}} \tf(\vk) \appxintn(\me^{2 \pi \sqrt{-1} \vk^T \cdot})\right)^2\right] \\
%      & =
%     \Ex_f\left[\sum_{\vk,\vl \in \integers^s \setminus\{\vzero\}} \tf(\vk) \tf^*(\vl) \appxintn(\me^{2 \pi \sqrt{-1} \vk^T \cdot}) \appxintn(\me^{-2 \pi \sqrt{-1} \vl^T \cdot})\right] \\
%     & = \underbrace{\sum_{\vk\in \integers^s \setminus\{\vzero\}} \tr(\vk)^{-2\alpha} \abs{\appxintn(\me^{2 \pi \sqrt{-1} \vk^T \cdot})}^2}_{Q_{2\alpha}(\vh,n) = \text{another measure of quality of the lattice nodes}}  \\
% \end{align*}

% \textbf{Research Question}  May we extend this result to all $n$ with perhaps some extra $\log(n)$ terms?




\Chapter{The figure of merit $P_{\alpha}(\vh,\vgamma,n,\onetos)$ for  $n$ other than $b^m$}






% E.g.,  $n =6 = 0\cdot 1 + 1 \cdot 2 + 1 \cdot 4$, $m = 2$ $\vDelta = \vzero$.

% \begin{align*}
%    \MoveEqLeft{\appxint_{6}(\me^{2 \pi \sqrt{-1} \vk^T \cdot})}\\
%     &= \frac 1{6} \sum_{i=0}^{5} \me^{2 \pi \sqrt{-1} \vk^T \vh\phi(i)} \\
%     &= \frac {1}{6} \left\{ \sum_{i=0}^3 \me^{2 \pi \sqrt{-1} \vk^T \vh\phi(i)} + \sum_{i=4}^5 \me^{2 \pi \sqrt{-1} \vk^T \vh\phi(i)} + \sum_{i=6}^5 \me^{2 \pi \sqrt{-1} \vk^T \vh\phi(i)}  \right\} \\
%     &= \frac {1}{6} \left\{ \sum_{i=0}^3 \me^{2 \pi \sqrt{-1} \vk^T \vh\phi(i)} + \sum_{i=0}^1 \me^{2 \pi \sqrt{-1} \vk^T \vh[\phi(i)+\phi(4)]} + \sum_{i=0}^{-1} \me^{2 \pi \sqrt{-1} \vk^T \vh[\phi(i)+\vphi(6)]}  \right\} \ \ \text{by \eqref{eq:phiproptwo}} \\
%     &= \frac {1}{6} \left\{ 1\cdot\sum_{i=0}^{3}\me^{2 \pi \sqrt{-1} \vk^T \vh\phi(i)} + 1 \cdot \me^{2 \pi \sqrt{-1} \vk^T \phi(4)} \cdot \sum_{i=0}^{1}\me^{2 \pi \sqrt{-1} \vk^T \vh\phi(i)} \right . \\
%     & \qquad \qquad \left . +  0 \cdot
%     \me^{2 \pi \sqrt{-1} \vk^T \vh\phi(6)} \cdot \sum_{i=0}^{0}\me^{2 \pi \sqrt{-1} \vk^T \vh\phi(i)} \right\}
% \end{align*}

%  Then,
% \begin{align*}
%     P_\alpha(\vh,6) & = \sum_{\vk \in \integers^s \setminus\{\vzero\}}
%     \abs{\appxint_6(\me^{2 \pi \sqrt{-1} \vk^T \cdot})}\tr(\vk)^{-\alpha}
%     \\
%     & \le
%     \frac {1}{6} \left\{ 1\cdot\sum_{\vk \in \integers^s \setminus\{\vzero\}} \abs{\sum_{i=0}^{3}\me^{2 \pi \sqrt{-1} \vk^T \vh\phi(i)}} \tr(\vk)^{-\alpha}  \right .
%     \\
%     & \qquad \qquad + 1  \cdot \sum_{\vk \in \integers^s \setminus\{\vzero\}} \abs{\sum_{i=0}^{1}\me^{2 \pi \sqrt{-1} \vk^T \vh\phi(i)}} \tr(\vk)^{-\alpha}  \\
%     & \qquad \qquad \left . +  0
%     \cdot \sum_{\vk \in \integers^s \setminus\{\vzero\}} \abs{\sum_{i=0}^{0}\me^{2 \pi \sqrt{-1} \vk^T \vh\phi(i)}}  \tr(\vk)^{-\alpha} \right\}\\
%     & =
%     \frac {1}{6} \left\{ 1\cdot\sum_{\vk \in \integers^s \setminus\{\vzero\}} 4 \cdot \bbone_{B(\vh,2)}(\vk) \tr(\vk)^{-\alpha}  \right .
%     + 1  \cdot \sum_{\vk \in \integers^s \setminus\{\vzero\}}2 \cdot \bbone_{B(\vh,1)}(\vk) \tr(\vk)^{-\alpha}  \\
%     & \qquad \qquad \left . +  0
%     \cdot \sum_{\vk \in \integers^s \setminus\{\vzero\}} 1 \cdot \bbone_{B(\vh,1)}(\vk)  \tr(\vk)^{-\alpha} \right\} \qquad \text{by \eqref{Eq:DualLattice}}\\
%     & = \frac {1}{6} \left\{ 4 \cdot 1 \cdot P_\alpha(\vh,4)+ 2 \cdot 1 \cdot P_\alpha(\vh,2)+ 1 \cdot 0 \cdot P_\alpha(\vh,1) \right\} \\
% \end{align*}

% We know that and $P_\alpha(\vh,2^2) \le C_{R}(\alpha,\epsilon,s)
%     2^{-2\alpha}$ and $P_\alpha(\vh,2^1) \le C_{R}(\alpha,\epsilon,s)
%     2^{-\alpha} $. Then,
%     \begin{align*}
%         P_\alpha(\vh,6) & \le \frac {1}{6} \left\{ 4 \cdot 1 \cdot P_\alpha(\vh,4) + 2 \cdot 1 \cdot P_\alpha(\vh,2) \right\} \\
%          & \le \frac {1}{6} \left\{C_{R}(\alpha,\epsilon,s)
%          2^{2-2\alpha} + C_{R}(\alpha,\epsilon,s)
%          2^{1-\alpha} \right\}
%     \end{align*}

\Section{$P_{\alpha}(\vh,\vgamma,n,\onetos)$ for  arbitrary $n$}
For any non-negative integer $n$, express as  $n = n_0 + bn_1 + \cdots + b^m n_m$, where the digits $n_0, n_1, \ldots$ are in $\{0, 1, \cdots, b-1\}$. Thus, $m = \lfloor \log_b(n) \rfloor$.  Let $\trun(n,\ell) = n_0 + \cdots + b^{\ell} n_{\ell}$ for $\ell = 0, \ldots, m$ and $\trun(n,-1) = 0$, then

\begin{align*}
    \MoveEqLeft{\appxint_{n}(\me^{2 \pi \sqrt{-1} \vk^T \cdot}) } \\
    &= \frac 1{n} \sum_{i=0}^{n-1} \me^{2 \pi \sqrt{-1} \vk^T (\vh\phi(i) + \vDelta \pmod{\vone})} \\
    \nonumber
    &= \frac {\me^{2 \pi \sqrt{-1} \vk^T \vDelta}}{n} \sum_{\ell = 0}^{m} \,
    \sum_{i=n - \trun(n,m-\ell) }^{n - \trun(n,m-\ell-1)  -1} \me^{2 \pi \sqrt{-1} \vk^T \vh\phi(i)} \\
    &= \frac {\me^{2 \pi \sqrt{-1} \vk^T \vDelta}}{n} \sum_{\ell = 0}^{m} \, n_{m-\ell}
    \sum_{i=0 }^{b^{m-\ell}  -1} \me^{2 \pi \sqrt{-1} \vk^T \vh[i/b^m + \phi(n - \trun(n,m-\ell))]} \\
    & \hspace{30ex} \text{by \eqref{eq:phiproptwo}} \\
    & = \frac {\me^{2 \pi \sqrt{-1} \vk^T \vDelta}}{n} \sum_{\ell = 0}^{m} b^{m-\ell} n_{m - \ell} \bbone_{B(\vh,m-\ell,1:s)}(\vk) \me^{2 \pi \sqrt{-1} \vk^T \vh\phi(n - \trun(n,m-\ell))}.
\end{align*}
This implies that
\begin{equation} \label{eq:ihatarbn}
    \abs{\appxint_{n}(\me^{2 \pi \sqrt{-1} \vk^T \cdot})} \le \frac {1}{n} \sum_{\ell = 0}^{m} b^{m-\ell} n_{m - \ell} \bbone_{B(\vh,m-\ell, 1:s)}(\vk)
\end{equation}

\LarysaNote{First we want to express it as a sum.. } 
Next we compute $P_\alpha(\vh,\vgamma,n,\onetos)$ for this arbitrary $n$:
\begin{align*}
      P_\alpha(\vh,\vgamma,n,\onetos) & = \sum_{\vk \in \integers^s \setminus\{\vzero\}} \abs{\appxint_n(\me^{2 \pi \sqrt{-1} \vk^T \cdot})}\tr(\vk,\vgamma)^{-\alpha} \qquad \text{by \eqref{eq:wcerrPalpha}} \\
      & \le \frac {1}{n} \sum_{\ell = 0}^{m} b^{m-\ell} n_{m - \ell} \sum_{\vk \in B(\vh,m-\ell,\onetos)} \tr(\vk,\vgamma)^{-\alpha}
      \qquad \text{by \eqref{eq:ihatarbn}}\\
      & = \frac {1}{n} \sum_{\ell = 0}^{m} b^{m-\ell} n_{m - \ell} P_\alpha(\vh,\vgamma,b^{m-\ell},\onetos)
      \qquad \text{by \eqref{eq:Palphadual}}\\
\intertext{Writing this sum out explicitly, we get}
      P_\alpha(\vh,\vgamma,n,\onetos) & =  \frac {1}{n} \left\{ b^m n_m P_\alpha(\vh,\vgamma,b^m,\onetos) + b^{m-1} n_{m-1} P_\alpha(\vh,\vgamma,b^{m-1},\onetos) + \cdots + b^0 n_0 P_\alpha(\vh,\vgamma,b^0,\onetos)  \right\} \\
      &\le \frac {(b-1)}{n} \left\{ b^m  P_\alpha(\vh,\vgamma,b^m,\onetos) + b^{m-1}  P_\alpha(\vh,\vgamma,b^{m-1},\onetos) + \cdots + 2^0  P_\alpha(\vh,\vgamma,b^0,\onetos)  \right\} \\
      & \le \frac {(b-1)C_{P}(\alpha,\vgamma,\epsilon,s)}{n} \left\{ b^{m(1-\alpha)}\max(1, (\log b^{m})^{\alpha(s+1)}) [\max(1,\log \log (
    b^m+1))]^{\alpha(1+\epsilon)}\right .\\ 
      & \qquad  + b^{(m-1)(1-\alpha)}\max(1, (\log b^{(m-1)})^{\alpha(s+1)}) [\max(1,\log \log (
    b^{(m-1)}+1))]^{\alpha(1+\epsilon)}\\
      & \qquad + \cdots \\
      & \qquad  \left . + b^{-0}\max(1, (\log b^{0})^{\alpha(s+1)}) [\max(1,\log \log (
    b^{0}+1))]^{\alpha(1+\epsilon)} \right\}\\
      & \qquad \qquad \qquad \alpha \ge 1 \qquad \text{by \eqref{eq:Palphaextm}}\\
      & \le \frac {(b-1)C_{P}(\alpha,\vgamma,\epsilon,s)}{n} \frac{(1 - b^{(m+1)(1-\alpha)})}{1 - b^{1-\alpha}}
      \max(1, (\log b^{m})^{\alpha(s+1)}) \\
      & \qquad \qquad \times [\max(1,\log \log (
    b^m+1))]^{\alpha(1+\epsilon)} \\
      & \le \frac {C_{P}(\alpha,\vgamma,\epsilon,s)}{n} \frac{(b-1)}{1 - b^{1-\alpha}}
  [\max(1, (\log n)^{\alpha(s+1)})
      [\max(1,\log \log (n+1))]^{\alpha(1+\epsilon)} 
\end{align*}

Unfortunately, this only decays like $\Order(n^{-1+\delta})$. \\

% \begin{align*}
%       P_\alpha(\vh,\vgamma,n,u) & = \sum_{\vk \in \integers^s \setminus\{\vzero\}} \abs{\appxint_n(\me^{2 \pi \sqrt{-1} \vk^T \cdot})}\tr(\vk)^{-\alpha} \qquad \text{by \eqref{eq:wcerrPalpha}} \\
%       & \le \frac {1}{n} \sum_{\ell = 0}^{m} 2^{m-\ell} n_{m - \ell} \sum_{\vk \in B(\vh,m-\ell)} \tr(\vk)^{-\alpha}
%       \qquad \text{by \eqref{eq:ihatarbn}}\\
%       & = \frac {1}{n} \sum_{\ell = 0}^{m} 2^{m-\ell} n_{m - \ell} P_\alpha(\vh,\vgamma,2^{m-\ell},u)
%       \qquad \text{by \eqref{eq:Palphadual}}\\
%       & =  \frac {1}{n} \left\{ 2^m n_m P_\alpha(\vh,\vgamma,2^m,u) + 2^{m-1} n_{m-1} P_\alpha(\vh,\vgamma,2^{m-1},u) + \cdots + 2^0 n_0 P_\alpha(\vh,\vgamma,2^0,u)  \right\} \\
%       & \le \frac {C_{R}(\alpha,\epsilon,s)}{n} \left\{ 2^{m(1-\alpha)}\max(1, \log(2^{m\alpha(s+1)})) [\max(1,\log \log (
%     2^m+1))]^{\alpha(1+\epsilon)}\right .\\ 
%       & \qquad  + 2^{(m-1)(1-\alpha)}\max(1, \log(2^{(m-1)\alpha(s+1)})) [\max(1,\log \log (
%     2^{(m-1)}+1))]^{\alpha(1+\epsilon)}\\
%       & \qquad + \cdots \\
%       & \qquad  \left . + 2^{-0}\max(1, \log(2^{0\alpha(s+1)})) [\max(1,\log \log (
%     2^{0}+1))]^{\alpha(1+\epsilon)} \right\}\\
%       & \qquad \qquad \qquad \alpha \ge 1 \qquad \text{by \eqref{eq:Palphaextm}}\\
%       & \le \frac {C_{R}(\alpha,\epsilon,s)}{n} \frac{(1 - 2^{(m+1)(1-\alpha)})}{1 - 2^{1-\alpha}}
%       \max(1, \log(2^{m\alpha(s+1)})) \\
%       & \qquad \qquad \times [\max(1,\log \log (
%     2^m+1))]^{\alpha(1+\epsilon)} \\
%       & \le \frac {\widetilde{C}_{R}(\alpha,\epsilon,s)}{n}
%       \max(1, \log(n^{\alpha(s+1)}))
%       [\max(1,\log \log (
%     n+1))]^{\alpha(1+\epsilon)}
%       % add C_R and the worst log
% \end{align*}
% Unfortunately, this only decays like $\Order(n^{-1+\delta})$.

\Section{$P_{\alpha}(\vh,\vgamma,n,\onetos)$ for $n$ an integer multiple of $b^p$}

Thus, next we try $n = \lambda b^p$, where $\lambda$ is an odd integer. Then, $n = b^pn_p + b^{p+1}n_{p+1} + \cdots + b^m n_m$
% \FredNote{Let's carry on following the above.  WAtch out for your geometric sum}
\begin{align*}
    P_\alpha(\vh,\vgamma,\lambda b^p,\onetos)
    %& \le \frac {1}{\lambda 2^p} \left\{2^p  P_\alpha(\vh,2^p) + \sum_{\ell = 0}^{m-p-1} 2^{m-\ell} n_{m - \ell} P_\alpha(\vh,2^{m-\ell})\right\} \\
    & \le \frac {1}{\lambda b^p} \sum_{\ell = 0}^{m-p} b^{m-\ell} n_{m -\ell} P_\alpha(\vh,\vgamma,b^{m-\ell},\onetos) \\
    &= \frac {1}{\lambda b^p} \left\{b^m n_m P_\alpha(\vh,\vgamma,b^m,\onetos) + b^{m-1} n_{m-1} P_\alpha(\vh,\vgamma,b^{m-1},\onetos) + \cdots + b^p n_p P_\alpha(\vh,\vgamma,b^p,\onetos) 
    \right\} \\ 
    & \le \frac {(b-1)}{\lambda b^p} \left\{b^m  P_\alpha(\vh,\vgamma,b^m,\onetos) + b^{m-1}  P_\alpha(\vh,\vgamma,b^{m-1},\onetos) + \cdots + b^p  P_\alpha(\vh,\vgamma,b^p,\onetos) 
    \right\} \\ 
    & \le \frac {(b-1)}{\lambda b^p} C_{P}(\alpha,\vgamma,\epsilon,s) \left\{ b^{m(1-\alpha)}\max(1, (\log b^{m})^{\alpha(s+1)})) [\max(1,\log \log (
    b^m+1))]^{\alpha(1+\epsilon)}\right .\\ 
    & \qquad  + b^{(m-1)(1-\alpha)}\max(1, (\log b^{(m-1)})^{\alpha(s+1)})) [\max(1,\log \log (
    b^{(m-1)}+1))]^{\alpha(1+\epsilon)}\\
      &  \qquad + \cdots \\
      & \qquad  \left . +   b^{p(1-\alpha)}\max(1,(\log b^{p})^{\alpha(s+1)})) [\max(1,\log \log (b^p+1))]^{\alpha(1+\epsilon))}  \right\}\\
    & \le \frac {(b-1)}{\lambda b^p} C_{P}(\alpha,\vgamma,\epsilon,s) 
    \left\{\frac{b^{p(1-\alpha)} - b^{(m+1)(1- \alpha )}}{1 - b^{1 -\alpha}}\right\} \\
     & \qquad \qquad \times \max(1, (\log b^{m})^{\alpha(s+1)})) [\max(1,\log \log (
    b^m+1))]^{\alpha(1+\epsilon)} \\
    & = \frac { (b-1) C_{P}(\alpha, \vgamma, \epsilon,s)}{\lambda b^p} \left\{ b^{p(1-\alpha)} \cdot \frac{(1 - b^{(m-p+1)(1- \alpha )})}{1 - b^{1- \alpha}}\right\} \\
    & \qquad \qquad \times \max(1, (\log b^{m})^{\alpha(s+1)})) [\max(1,\log \log (
    b^m+1))]^{\alpha(1+\epsilon)} \\
    &=  \frac {(b-1)C_{P}(\alpha, \vgamma, \epsilon,s)}{\lambda b^{p\alpha}}  \frac{1 - b^{(m-p+1)(1- \alpha )}}{1 - b^{1- \alpha}} \max(1, (\log b^{m})^{\alpha(s+1)})) [\max(1,\log \log (
    b^m+1))]^{\alpha(1+\epsilon)} \\
    &\le \frac{\lambda^{\alpha -1}{C}_{P}(\alpha, \vgamma,\epsilon,s)}{n^{\alpha}}  \frac {(b-1)} {1 - b^{1- \alpha}}   \max(1, (\log n )^{\alpha(s+1)})) [\max(1,\log \log (
    n+1))]^{\alpha(1+\epsilon)}
\end{align*}
Which decays  nearly like $\Order(b^{-p \alpha})$ \\
% \begin{align*}
%     P_\alpha(\vh,\lambda 2^p)
%     %& \le \frac {1}{\lambda 2^p} \left\{2^p  P_\alpha(\vh,2^p) + \sum_{\ell = 0}^{m-p-1} 2^{m-\ell} n_{m - \ell} P_\alpha(\vh,2^{m-\ell})\right\} \\
%     & \le \frac {1}{\lambda 2^p} \sum_{\ell = 0}^{m-p} 2^{m-\ell} n_{m -\ell} P_\alpha(\vh,2^{m-\ell}) \\
%     &= \frac {1}{\lambda 2^p} \left\{2^m n_m P_\alpha(\vh,2^m) + 2^{m-1} n_{m-1} P_\alpha(\vh,2^{m-1}) + \cdots + 2^p n_p P_\alpha(\vh,2^p) % no n_p though
%     \right\} \\ % this sum doesnt look right
%     & \le \frac {1}{\lambda 2^p} C_{R}(\alpha,\epsilon,s) \left\{ 2^{m(1-\alpha)}\max(1, \log(2^{m\alpha(s+1)})) [\max(1,\log \log (
%     2^m+1))]^{\alpha(1+\epsilon)}\right .\\ 
%     & \qquad  + 2^{(m-1)(1-\alpha)}\max(1, \log(2^{(m-1)\alpha(s+1)})) [\max(1,\log \log (
%     2^{(m-1)}+1))]^{\alpha(1+\epsilon)}\\
%       &  \qquad + \cdots \\
%       & \qquad  \left . +   2^{p(1-\alpha)}\max(1,\log(2^{p\alpha(s+1)})) [\max(1,\log \log (2^p+1))]^{\alpha(1+\epsilon))}  \right\}\\
%     & \le \frac {1}{\lambda 2^p} C_{R}(\alpha,\epsilon,s) 
%     \left\{\frac{2^{p(1-\alpha)} - 2^{(m+1)(1- \alpha )}}{1 - 2^{1 -\alpha}}\right\} \max(1, \log(2^{m\alpha(s+1)})) [\max(1,\log \log (
%     2^m+1))]^{\alpha(1+\epsilon)} \\
%     & = \frac { C_{R}(\alpha,\epsilon,s)}{\lambda 2^p} \left\{ 2^{p(1-\alpha)} \cdot \frac{(1 - 2^{(m-p+1)(1- \alpha )})}{1 - 2^{1- \alpha}}\right\} 
%      \max(1, \log(2^{m\alpha(s+1)})) [\max(1,\log \log (
%     2^m+1))]^{\alpha(1+\epsilon)} \\
%     &=  \frac {C_{R}(\alpha,\epsilon,s)}{\lambda 2^{p\alpha}}  \frac{1 - 2^{(m-p+1)(1- \alpha )}}{1 - 2^{1- \alpha}} \max(1, \log(2^{m\alpha(s+1)})) [\max(1,\log \log (
%     2^m+1))]^{\alpha(1+\epsilon)} \\
%     &\le \frac{\lambda^{\alpha -1}}{n^{\alpha}} \widetilde{C}_{R}(\alpha,\epsilon,s)  \max(1, \log(n^{\alpha(s+1)})) [\max(1,\log \log (
%     n+1))]^{\alpha(1+\epsilon)} \\ 
% \end{align*}
% Which decays like $\Order(2^p)$ \\

% For small $\lambda$: let $\lambda = 3, \ p = 10,\  b =2 $
% \begin{align*}
%     P_\alpha(\vh,\vgamma,3 \cdot 2^{10},\onetos) 
%     &\leq \frac{3^{\alpha -1}{C}_{P}(\alpha, \vgamma,\epsilon,s)}{(3 \cdot 2^{10})^{\alpha}}  \frac {1} {1 - 2^{1- \alpha}} \\
%     & \qquad \qquad \times
%     \max(1, (\log (3 \cdot 2^{10}) )^{\alpha(s+1)})) [\max(1,\log \log (
%     3 \cdot 2^{10}+1))]^{\alpha(1+\epsilon)}
% \end{align*}

% For large $\lambda$: let $\lambda = 1021, \ p = 1, \ b =2 $

% \begin{align*}
%     P_\alpha(\vh,\vgamma,1021 \cdot 2^{1},\onetos) 
%     &\leq \frac{1021^{\alpha -1}{C}_{P}(\alpha, \vgamma,\epsilon,s)}{(1021 \cdot 2^{1})^{\alpha}}  \frac {1} {1 - 2^{1- \alpha}} \\
%     & \qquad \qquad \times \max(1, (\log (1021 \cdot 2^{1}) )^{\alpha(s+1)})) [\max(1,\log \log (
%     1021 \cdot 2^{1}+1))]^{\alpha(1+\epsilon)}
% \end{align*}


\Chapter{Extensions to other spaces of integrands and error measures}

We can also consider a more general class of integrands, note: 
$$
\norm[\cf{\alpha,q}]{f} := \norm[q]{\tr(\cdot,\vgamma)^{\alpha}\tilde{f}(\cdot)} = \norm[q]{\big\{\tr(\vk,\vgamma)^{\alpha}\tilde{f}(\vk)\bigr\}_{\vk \in \integers^s}} = \left[\sum_{\vk \in \integers^s} \left(\tr(\vk,\vgamma)^{\alpha}\abs{\tilde{f}(\vk)}\right)^q \right]^{\frac{1}{q}} \qquad 1 \leq q \leq \infty, \qquad\frac{1}{q} + \frac{1}{q'} = 1 
$$

$\abs{\int_{[0,1)^{s}} f(\vx) \, \dif \vx - \appxint_n(f)} = \abs{\sum_{\vk \neq 0} \tf(\vx) \, - \appxint_n(f)}  \leq \norm[\cf{\alpha,q}]{f} \norm[q^{'}]{\tr(\cdot,\vgamma)^{-\alpha} \eta(\cdot)}$, \\ 

where $\eta(\vk) = \appxint_n(\me^{2 \pi \sqrt{-1} \vk^T \cdot})$. \\

For $n = b^m,\  \eta(\vk) = \me^{2 \pi \sqrt{-1} \vk^T \vDelta}\bbone_{B(\vh,m,1:s)}(\vk) $

\begin{align*}
    \norm[q'] {\tr(\cdot,\vgamma)^{-\alpha} \eta(\vk)} 
    &= \norm[q'] {\tr(\vk,\vgamma)^{-\alpha} \me^{2 \pi \sqrt{-1} \vk^T \vDelta}\bbone_{B(\vh,m,1:s)}(\vk) } \\
    &=  \left( \sum_{\vk} \abs{\tr(\vk,\vgamma)^{-\alpha} \me^{2 \pi \sqrt{-1} \vk^T \vDelta}\bbone_{B(\vh,m,1:s)}(\vk)}^{q'}\right)^{\frac{1}{q'}} \\
    &= \left(\sum_{\vk \in B(\vh,m,1:s)} \tr(\vk,\vgamma)^{-\alpha q'}\right)^{\frac{1}{q'}} \\
    &= \left( P_{\alpha q'}(\vh,\vgamma,b^m,\onetos) \right)^{\frac{1}{q'}} \\
    &= (\approx n^{-\alpha q'})^{\frac{1}{q'}} \\
    & \approx n^{-\alpha}
\end{align*}



Now, consider arbitrary $n$:
\begin{align*}
    \norm[q'] {\tr(\cdot,\vgamma)^{-\alpha} \appxint_n(\me^{2 \pi \sqrt{-1} \vk^T \cdot})}^{q'} 
    % &= \left(\sum_{\vk}  \abs{\tr(\vk,\vgamma)^{-\alpha} \appxint_n(\me^{2 \pi \sqrt{-1} \vk^T \cdot})}^{q'} \right) \\
    % & \leq \left(\sum_{\vk  }  \abs{\tr(\vk,\vgamma)^{-\alpha} \frac {1}{n} \sum_{\ell = 0}^{m} b^{m-\ell} n_{m-\ell} \bbone_{B(\vh,m-\ell)}  }^{q'}\right) \text{by \eqref{eq:ihatarbn}} \\
    % % &\leq   P_{\alpha}(\vh,\vgamma,n,\onetos)  \textbf{\LarysaNote{??}}
    % & = \left(\sum_{\vk \in B(\vh,m-\ell,1:s)  }  \abs{\tr(\vk,\vgamma)^{-\alpha} \frac {1}{n} \sum_{\ell = 0}^{m} b^{m-\ell} n_{m-\ell}  }^{q'}\right) \\
    % &= {\left( \frac {1}{n} \sum_{\ell = 0}^{m} b^{m-\ell} n_{m-\ell}\right)^{q'}}{\sum_{\vk \in B(\vh,m-\ell,1:s)} \tr(\vk,\vgamma)^{-\alpha q'}} \\
    % &= P_{\alpha q'}(\vh,\vgamma,n,\onetos) \\
    % & \approx n^{-1}
    &\leq  \norm[1] {\tr(\cdot,\vgamma)^{-\alpha} \appxint_n(\me^{2 \pi \sqrt{-1} \vk^T \cdot})} \\ 
    % &= P_{\alpha}(\vh,\vgamma,n,\onetos) \\
    &  \approx \Order(n^{-1+\delta}) %?
\end{align*}
This quantity has the same upper bound as the case with $q^{'} = 1$. Therefore, there does not appear to be an advantage in making $q$ smaller and thus making $q^{'}$ larger. 
% bounded by P_\alpha since q'= 1 is the max 
\Section{Randomized Error Analysis}
Another approach is to assume that $f$ in \eqref{eq:fseries} is a random function (stochastic process) where
\begin{equation}
    \abs{\tf(\vk)} \IIDsim \cn(0,\tr(\vk)^{-2\alpha}).
\end{equation}
Then it follows that
\begin{align*}
    \MoveEqLeft{\Ex_f\left[\left(\int_{[0,1)^d} f(\vx) \, \dif \vx - \appxintn(f) \right)^2\right]} \\
    & =
    \Ex_f\left[\left(\sum_{\vk \in \integers^s \setminus\{\vzero\}} \tf(\vk) \appxintn(\me^{2 \pi \sqrt{-1} \vk^T \cdot})\right)^2\right] \\
     & =
    \Ex_f\left[\sum_{\vk,\vl \in \integers^s \setminus\{\vzero\}} \tf(\vk) \tf^*(\vl) \appxintn(\me^{2 \pi \sqrt{-1} \vk^T \cdot}) \appxintn(\me^{-2 \pi \sqrt{-1} \vl^T \cdot})\right] \\
    & = \underbrace{\sum_{\vk\in \integers^s \setminus\{\vzero\}} \tr(\vk)^{-2\alpha} \abs{\appxintn(\me^{2 \pi \sqrt{-1} \vk^T \cdot})}^2}_{Q_{2\alpha}(\vh,n) = \text{another measure of quality of the lattice nodes}}  \\
\end{align*}

In terms of powers of $n$, we will not be able to improve the upper bound by making $q^{'}$ bigger. 


\Chapter{Lattice Sequences and Toeplitz Matrices}

Consider the lattice sequence defined in \eqref{eq:extlat} and the formula for $P_{\alpha}(\vh,\gamma, n, 1:s)$ given in \eqref{eq:wcerrPalpha}

%\[ \]
Previously, we assumed that all the weights were equal. Now, we want to see if we can improve the upper bound on $P_{\alpha}$ if we have unequal weights. 
Let $\appxintn(f) = \frac{1}{n} \sum_{i=0}^{n-1} w_if(\vz_i) $

Then the error in \eqref{eq:wcerrPalpha} becomes
\begin{align*}
\nonumber
\abs{\int_{[0,1)^{s}} f(\vx) \, \dif \vx - \appxint_n(f)}^2 
& \le \norm[\cf{\alpha},2]{f}^2 \sum_{\vk \in \integers^s \setminus\{\vzero\}} \abs{\appxint_n(\me^{2 \pi \sqrt{-1} \vk^T \cdot})}^2\tr(\vk,\vgamma)^{-2\alpha} \\
\end{align*}





$B_2$ is a Bernoulli polynomial of degree 2 \cite{OlvEtal10a}
% BIBLIOGRAPHY
%
% you have two options: 1) create bibliography manually,
% 2) create bibliography automatically. See BibliographyHelp.pdf file for details.


%\bibliographystyle{plain}
%\bibliographystyle{apacite}
\bibliographystyle{ieeetr} %this package will format in IEEE style.
%\bibliographystyle{elsarticle-harv.bst}
\bibliography{FJH23,FJHown23,LarysaReferences}

\end{document}  % end of document