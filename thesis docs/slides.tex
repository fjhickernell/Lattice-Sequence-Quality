\documentclass{beamer}
\setbeamertemplate{bibliography item}{[\theenumiv]}
\usepackage[english]{babel}
\usepackage{amsfonts}
\usepackage{amsmath}
\usepackage{amssymb}
% \usepackage{graphicx}
\usepackage{array}
\usepackage{xcolor}
\usepackage{setspace}
 \usepackage{movie15}
\usetheme{CambridgeUS}
% \usetheme{Boadilla}
 \usepackage{hyperref}
% \usepackage{mathtools,upref,siunitx,upquote,fancyvrb,bbm,xspace,color,amsmath,amssymb, bm,amsthm }
% \usepackage[hyphens]{url}
% \usepackage[utf8]{inputenc}
% \usepackage{esdiff}
% \usepackage{graphicx}
% \usepackage{xcolor}

\input{FJHDef.tex}
\usepackage{xspace}
\usepackage{algpseudocode}
\usepackage{algorithm, algorithmicx}
\algnewcommand\algorithmicparam{\textbf{Parameters:}}
\algnewcommand\PARAM{\item[\algorithmicparam]}
\algnewcommand\algorithmicinput{\textbf{Input:}}
\algnewcommand\INPUT{\item[\algorithmicinput]}
\algnewcommand\RETURN{\State \textbf{Return }}

\newcommand{\tr}{\widetilde{r}}
\newcommand{\appxintn}{\appxint_n}
\DeclareMathOperator{\appxint}{\hat{I}}
\DeclareMathOperator{\trun}{trunc}
\newcommand{\onetos}{1\!:\!s}

% \newcommand{\FredNote}[1]{{\color{blue}#1}}
% \newcommand{\FredBNote}[1]{{\color{blue}[#1]}}

% \newcommand{\LarysaNote}[1]{{\color{violet}#1}}

% \newtheorem{theorem}{Theorem}
% add more libraries here if you need them


\title[Illinois Institute of Technology]{The Quality of Lattice Sequences}

%\subtitle{\footnotesize Larysa Matiukha \& Dr.\ Sou-Cheng T.\ Choi}

\author[] {Larysa Matiukha \\
Adviser: Fred Hickernell 
}

\institute{Department of Applied Mathematics, Illinois Institute of Technology}

\date{30 April 2025}

\graphicspath{ {./images/} }

% \logo{
% \includegraphics[width=1cm]{thesis docs/slides_images/iit_logo.png}

% }

 % you can change it on choice
% I recommend: CambridgeUS, Berlin, Singapore
% for more choices, visit: https://deic.uab.cat/~iblanes/beamer_gallery/index_by_theme_and_color.html



\begin{document}
{

\begin{frame} 
    \maketitle
\end{frame}
}
%\frame{\titlepage}
\begin{frame}
\frametitle{Contents}
\tableofcontents
\end{frame}


\addtocontents{toc}{\setcounter{tocdepth}{2}}

\section*{Acknowledgements}  
\begin{frame} {Acknowledgements}
The invaluable guidance and mentorship of my adviser, Dr.\ Fred Hickernell, cannot be overstated. It has been an honor and a great pleasure working with him and I feel extremely fortunate to have had him as an adviser. \\
\vspace{0.5em}

 % I would like to thank my thesis committee 
 % Dr.\ Yuhan Ding for her contributions to this work and Dr.\ Igor Cialenco 
I would like to thank Dr.\ Yuhan Ding and and Dr.\ Igor Cialenco for both serving on my
thesis committee and the support they've provided me throughout the completion of my degree. \\
\vspace{0.5em}

 I would like to thank Dr.\ Sou-Cheng Choi for her mentorship during my graduate studies, which she has helped make so much more fulfilling and enjoyable.
\vspace{0.5em}

 Finally, I would like to thank my family and friends for their unconditional support.
\end{frame}

\section{Introduction}
\begin{frame}{Introduction}
        \textbf{Multidimensional integrals}
      \begin{itemize}
          \item  Relevant to quantitative finance \cite{CafMorOwe97}, statistics \cite{Gen92,GenBre99}, and physics \cite{PapTra97}
          \item Often arise as expectations of random variables
          \end{itemize}
          \textbf{Challenge:}
          \begin{itemize}
              \item Many such integrals cannot be computed analytically 
          \end{itemize}
          
          \textbf{Solution:}
          \begin{itemize}
              \item Approximate an integral a sample mean:
              \[
\mathbb{E}[f(\underbrace{\vX}_{\sim \mathcal{U}[0,1]^d})] : = \int_{[0,1)^s} f(\vx) \, \dif \vx \approx \frac{1}{n} \sum_{i=0}^{n-1} f(\vz_i) =:\appxintn(f),
\]
        \item The nodeset $\{\vz_0, \vz_1, \ldots \}$ is:
        \begin{itemize}
            \item Independent and identically distributed (IID): \textbf{Monte Carlo}
            \item More evenly distributed:  \textbf{Quasi-Monte Carlo}
            \end{itemize}
        \end{itemize}
\end{frame}

\begin{frame}{Lattices}
    %(Have bounds and typically use for preferred values but what if something happens and we cant get a full lattice)
    \begin{itemize}
        \item A popular choice of nodes $\{\vz_i\}_{i=0}^{n-1} \in [0,1)^{s \times n}$
        \item Closed under addition modulo one
        \item Can be extended for $n =  b, b^2, \ldots$, where $b$ is a prime
    \end{itemize}
    
\centerline{\includegraphics[width=0.9 \linewidth]{lattice-plot1.png}}

\begin{tabular}{m{0.3\textwidth}m{0.8\textwidth}}
What about  $ n = 47$?
& \includegraphics[width=0.3\linewidth]{thesis docs/plots/lattice_47.png} 
\end{tabular}

\end{frame}

\begin{frame}{Contributions}

    \begin{itemize}
        \item Derive theoretical upper bounds on the figure of merit $P_\alpha$ for arbitrary $n$
        \item Numerically explore the figure of merit $P_{\alpha,2}$
    \end{itemize}
    
\end{frame}
 \section{Integration Lattices}


% \textbf{Lattice points}
%     \begin{itemize}
%         \item  Were initially constructed as sets with fixed cardinality, $n$, 
% \begin{equation} \label{eq:lat}
%     \{\vz_i = i \vh/n \bmod{\vone} : i=0,1, \ldots, n-1 \} \in [0,1)^s,
% \end{equation}
% where $\vh \in \{1, \ldots, n-1\}^s$ is the \emph{generating vector}.
% \item Can include a random shift:
% \begin{equation} \label{eq:shlat}
%     \{\vz_i = i \vh/n + \vDelta \pmod{\vone} : i=0,1, \ldots, n-1 \} \in [0,1)^s.
% \end{equation}
%     \end{itemize}

\begin{frame}{Integration Lattices}
Historically, lattice points were constructed as sets of fixed size $n$:
\[
\{\vz_i = i \vh / n \bmod \vone : i = 0,1, \ldots, n-1\} \subset [0,1)^s
\]
\begin{itemize}
    \item $\vh \in \{1, \ldots, n-1\}^s$ is the \textbf{generating vector}.
    \item Closed under addition modulo $\vone$.
\end{itemize}

\vspace{0.5em}
\textbf{A (random) shift} $\vDelta$, is often added:
\begin{equation} 
    \{\vz_i = i \vh/n + \vDelta \pmod{\vone} : i=0,1, \ldots, n-1 \} \in [0,1)^s,
\end{equation}
where $\vDelta \in [0,1)^s$.

\end{frame}





\begin{frame}{Extensible Lattice Sequences}

\textbf{Extensible lattice sequences} \cite{HicEtal00, Mai81a}:
\[
\{\vz_i = \vh \phi(i) + \vDelta \pmod{\vone} : i = 0,1,\ldots\} \subset [0,1)^s
\]
\begin{itemize}
    \item $\vh$ must be a \textbf{generalized integer} \cite[Section 2]{HicNie03a}.
    \item $\phi(i)$: the \textbf{van der Corput sequence} in base $b$.
\end{itemize}

\vspace{0.5em}
\textbf{Van der Corput sequence:}
\[
\phi((\cdots i_2 i_1 i_0)_b) = {}_b0.i_0 i_1 i_2 \cdots
\]
Example (base $b=2$):
\[
\phi(6) = \phi(110_2) = {}_20.011 = \frac{3}{8}
\]

\end{frame}


\section{Worst Case Error Analysis for Arbitrary $n$}
    \begin{frame}{Worst Case Error Analysis for Arbitrary $n$}
We first consider integrands, $f$, that have an absolutely summable Fourier series:
\begin{equation} 
    f(\vx) = \sum_{\vk \in \integers^{s}} \tf(\vk) \me^{2 \pi \sqrt{-1} \vk^T \vx}, \quad \text{where } \tf(\vk) = \int_{[0,1)^{\infty}} f(\vx) \me^{-2 \pi \sqrt{-1} \vk^T \vx}\, \dif \vx
\end{equation}
Define the weights
\begin{equation}
\tr(\vk,\vgamma) = \prod_{j=1}^{s} r(k_{j},\gamma_{j}),
\quad \text{where } r(k_{j},\gamma_{j})=\begin{cases} 1, &
k_{j}=0, \\ \gamma_{j}^{-1}\abs{k_{j}}, & k_{j} \ne 0.  \end{cases}
\end{equation}

Define a Banach
space of functions:
\begin{equation} 
\cf_{\alpha} = \{ f \in \cl_2[0,1)^{s} :
\norm[\cf{\alpha}]{f} < \infty \}, \quad
\norm[\cf{\alpha}]{f} := \sup_{\vk \in \integers^{s}}
\left(\tr(\vk,\vgamma)^{\alpha} \abs{\tilde{f}(\vk)} \right).
\end{equation}
    \end{frame}

\begin{frame}{Cubature Error for General $n$} %doesn't look good
    The error in approximating the integral by the sample mean is:
    \begin{align*} 
\abs{\int_{[0,1)^{s}} f(\vx) \, \dif \vx - \appxint_n(f)} 
& \le \norm[\cf{\alpha}]{f} \underbrace{\sum_{\vk \in \integers^s \setminus\{\vzero\}} \abs{\appxint_n(\me^{2 \pi \sqrt{-1} \vk^T \cdot})}\tr(\vk,\vgamma)^{-\alpha}}_{=: \textcolor{red}{P_\alpha(\vh,\vgamma,n,1:s)} = \text{quality of the nodes}},  %\qquad \forall f \in \cf_{\alpha} \\
\end{align*}
where $\onetos$ means $\{1, \ldots, s\}$.
\end{frame}

\begin{frame}{Cubature Error for $n = b^m$} %doesn't look good
For $n = b^m$:
\begin{equation*} 
P_\alpha(\vh,\vgamma,b^m,\onetos) = \sum_{\vk \in B(\vh,m,\onetos)} \tr(\vk,\vgamma)^{-\alpha},
\end{equation*}
where 
\[
B(\vh,m,\onetos) := \left\{ \vk \in \mathbb{Z}^s \setminus \{\vzero\} : \vk^T \vh = 0 \pmod{b^m} \right\}
\]
is the \textbf{dual lattice}. By \cite[Theorem 5]{HicNie03a}, there exists $\vh$ with
\begin{multline*} 
P_{\alpha}(\vh,\vgamma,b^m,\onetos) \le C_{P}(\alpha,\vgamma,\epsilon,s) b^{-m\alpha}  (\log b^{m})^{\alpha(s+1)} [\log \log (
b^m+1)]^{\alpha(1+\epsilon)}, \\
\quad m = 1, 2,\ldots, \quad \alpha \ge 1.
\end{multline*}
\end{frame}

% \begin{frame}{Extension to $m = 0$}
% We can extend the above bound for the case of a single point (i.e., $n =1$)
%    \begin{align*}
%        P_{\alpha}(\vh,\vgamma,1,\onetos) & =  -1 + \left[1 + 2 \gamma^{\alpha}\zeta(\alpha) \right]^d,
%    \end{align*}
%    where $\zeta$ is the Riemann zeta function.
% \end{frame}

\section{The figure of merit $P_\alpha$ for $n$ other than $b^m$}

\begin{frame}{$P_\alpha$ for arbitrary $n$ }
    \begin{block}{Theorem 1 (Part 1)}
    \small 
    For fixed integer base $b \ge 2$, a fixed smoothness parameter $\alpha \ge 1$, a fixed vector of coordinate weights $\vgamma \in [0,\infty)^\infty$, and a fixed $\epsilon > 0$, there exists a generating vector for the lattice, $\vh$, for which the figure of merit, $P_\alpha$, has the following upper bound for arbitrary positive integer $n$:
    \begin{multline*} 
        P_\alpha(\vh,\vgamma,n,\onetos) 
        \le \frac {C_{P}(\alpha,\vgamma,\epsilon,s)}{n} \frac{(b-1)}{1 - b^{1-\alpha}} \\
        \times [\max(1, (\log n)^{\alpha(s+1)})]
      [\max(1,\log \log (n+1))]^{\alpha(1+\epsilon)} \\
      \forall n , s \in \naturals.
    \end{multline*}
    \end{block}
\end{frame}



\begin{frame}{$P_\alpha$ for $n = \lambda b^p$}
    \begin{block}{Theorem 1 (Part 2)}
    \small
    However, if $n$ is a (small) integer multiple of a power of the base, then $P_\alpha$ has an upper bound that is similar to the case of $n=b^m$:
    \begin{multline*} 
         P_\alpha(\vh,\vgamma,\lambda b^m,\onetos) 
         \le \frac{\lambda^{\alpha -1}{C}_{P}(\alpha, \vgamma,\epsilon,s)}{n^{\alpha}}  \frac {(b-1)} {1 - b^{1- \alpha}} \\
         \times \max(1, (\log n )^{\alpha(s+1)}) [\max(1,\log \log (
    n+1))]^{\alpha(1+\epsilon)} \\
    \forall m \in \natzero, \ \lambda, s \in \naturals.
    \end{multline*}
    % Here, the upper bound is similar to that in \eqref{eq:Palphaextm} but contains an extra factor of $\lambda^{\alpha -1}(b-1)/(1 - b^{1-\alpha})$.
    \end{block}
\end{frame}


\begin{frame}{$P_{\alpha}$ when the coordinate weights satisfy summability condition}
\begin{block} {Theorem 2}
 \small   
Suppose $\alpha > 1$.
 If $\sum_{j = 1}^{\infty} \gamma_j ^{\alpha} < \infty$, then for any fixed $\delta > 0$ :
\[
  P_\alpha(\vh,\vgamma,n,\onetos)
        \leq \frac{\tilde{C}_P(\alpha, \vgamma, \delta)}{n^{1-\delta}} \frac{b^{\delta}(b - 1) }{b^{\delta} - 1} 
\] If $\sum_{j = 1}^{\infty} \gamma_j < \infty$, then for any fixed $\delta $ such that $0 < \delta < \alpha - 1  $:
\[
  P_\alpha(\vh,\vgamma,\lambda b^p,\onetos)
        \leq \frac{\lambda^{\alpha - 1 - \delta }\tilde{C}_P(\alpha, \vgamma, \delta)}{n^{\alpha - \delta}}\frac{(b-1)}{1 - b^{1 - \alpha + \delta}}.
\]

\end{block}
\end{frame}

\begin{frame}{Idea of the Proofs}
    \begin{itemize}
        \item The first $n = n_0 + bn_1 + \cdots + b^m n_m$ points of a lattice sequence is the union of a lattice with $n_0$ points and a lattice with $b n_1$ points and \ldots
        \item $P_\alpha$ for $n$ points can be bounded in terms of the $P_\alpha$ values for those lattices
        
    \end{itemize}
\end{frame}

\section{Numerical experiments}
 \begin{frame}{The decay of $P_2$ for all positive integer $n \leq 2^{10}$ }
     \begin{figure}
         \centering
         \includegraphics[width=0.6 \linewidth]{thesis docs/plots/p_alpha_O.png}
         \caption{The decay of $P_2(\vh,[0.1, 0.05],n,1:2)$ for $ 1 \leq n \leq 2^{10}$ for the default $\vh$ in QMCPy. For values $n$ that are powers of $2$, the decay is nearly $\Order(n^{-2})$, while the decay of $P_2$ for arbitrary $n$ is like $\Order(n^{-1})$.}
         \label{fig:enter-label}
     \end{figure}
 \end{frame}

\begin{frame}{The decay for $n = \lambda 2^p$ for different small $\lambda$'s}
    \begin{figure}
        \centering
        \includegraphics[width=0.6\linewidth]{thesis docs/plots/p_alpha_lambda_O.png}
        \caption{The decay of $P_2(\vh,[0.1, 0.05],\lambda 2^p,1:2)$ for $ 1 \leq \lambda 2^p \leq 2^{10}$ for the default $\vh$ in QMCPy. Note the $\Order(n^{-2})$ decay in all cases.  The values of $P_2$ are generally smaller for smaller $\lambda$.}
    \label{fig:enter-label}
        \label{fig:enter-label}
    \end{figure}
\end{frame}


\section{$P_{\alpha,2}$ with optimal sample weights}


\begin{frame}{$P_{\alpha,2}$ with Optimal Sample Weights}
    \textbf{Goal:} Explore how can the upper bound on the cubature error be improved
    by having unequal sample weights versus equal sample weights. 
    
    
  
\end{frame}

\begin{frame}{A Hilbert Space Related to $\cf_{\alpha}$}
      %Define a Hilbert space of functions: % related to the Banach space defined in \eqref{eq:Banachspace}: 
$$
\cf_{\alpha,2} = \{ f \in \cl_2[0,1)^{s} :
\norm[\cf{\alpha},2]{f} < \infty \}, 
$$
$$
\norm[\cf{\alpha,2}]{f} := \left[\sum_{\vk \in \integers^s} \left(\tr(\vk,\vgamma)^{\alpha}\abs{\tilde{f}(\vk)}\right)^2 \right]^{\frac{1}{2}} .
$$

\bigskip
\[
P_{\alpha,2}^2 = P_{2\alpha} \qquad \text{for }n = b^m \text{ and equal sample weights}
\]


\end{frame}

\begin{frame}{Cubature Error for $\cf_{\alpha,2}$ }
    Let 
\[
\appxintn(f) =  \sum_{i=0}^{n-1} w_if(\vz_i),  
\]
\vspace{0.5em}
for $\vw = [w_1, w_2, \cdots, w_n]$, such that $\sum_j^{n} w_j = 1$
\vspace{0.5em}

The tight bound on the error becomes:
\begin{align*}
\abs{\int_{[0,1)^{s}} f(\vx) \, \dif \vx - \appxint_n(f)}^2 &\leq \norm[\cf{\alpha},2]{f}^2 \underbrace{ (\vw^T \mK \vw),}_{=: \textcolor{red}{ P_{\alpha,2}^2(\vh,\tilde{\vgamma},n,1:s)}} \\
\text{where } \mK  & = \biggl( \prod_{\ell = 1}^s [1 + \tilde{\gamma}_{\ell}^{2 \alpha}B_{2\alpha}(x_{i,\ell} - x_{j, \ell} \bmod 1 )]\biggr)_{i,j=0}^{n-1} , \\
\vw & = \bigl( w_i \bigr)_{i=0}^{n-1},
\end{align*}
\end{frame}

\begin{frame}{Finding Optimal Weights}
    To find the optimal sample weights $\vw$, we solve the following minimization problem:
    
\vspace{0.5em}
\vspace{0.5em}

\[ \min_{\vw} \vw^T \mK \vw \quad \text{s.t.} \quad \vw^T \vone = 1. \]
\vspace{0.5em}
Lagrange multipliers give the following optimal weights: 
\[
\vw = \frac{\mK^{-1}\vone}{\vone^T \mK^{-1}\vone }.
\]
\end{frame}


\begin{frame}
    \begin{figure}
    \centering
    \includegraphics[width=0.6\linewidth]{thesis docs/plots/ssdisc_vs_ssdiscopt.png}
    \caption{The decay of $P_{1,2}(\vh, [1, 2^{-2}, \cdots, 6^{-2}], n, 1:6)$ with equal sample weights vs unequal sample weights for $ 1 \leq n \leq 2^{12}$ for the default $\vh$ in QMCPy. The quantity $P_{1,2}$ decays nearly like $\Order(n^{-1})$, and for optimal weights it is non-increasing.}
    \label{fig:enter-label}
\end{figure}

\end{frame}


\begin{frame}
    \begin{figure}
    \centering
    \includegraphics[width=0.6\linewidth]{thesis docs/plots/ratio_floor.png}
    \caption{The ratio $P_{1,2}(\vh,[1, 1/4],n,1:2)/P_{1,2} (\vh,[1, 1/4],2^{\lfloor \log_2(n) \rfloor},1:2)$ for $ 1 \leq n \leq 2^7$. This shows how much better (smaller than one) or worse (greater than one) the figure of merit is compared to the largest sample size that is a power of $2$ and no greater than $n$. }
    \label{fig:enter-label}
\end{figure}
\end{frame}

\begin{frame}
    \begin{figure}
    \centering
    \includegraphics[width=0.6\linewidth]{thesis docs/plots/ratio_ceil.png}
    \caption{The ratio $P_{1,2}(\vh,[1, 1/4],n,1:2)/P_{1,2} (\vh,[1, 1/4],2^{\lceil \log_2(n) \rceil},1:2)$ for $ 1 \leq n \leq 2^7$. This shows how much worse (greater than one) the figure of merit is compared to the smallest sample size that is a power of $2$ and no less than $n$.}
    \label{fig:enter-label}
\end{figure}
\end{frame}




    
\section{Discussion and Future Work}
    \begin{frame}{Discussion and Future Work}
	       \begin{itemize}
	           \item Derived upper bound on the figure of merit $P_\alpha$ for extensible lattices sequences for arbitrary $n$.
               \item For a general $n$, the error decays nearly like $\Order(n^{-1 + \delta})$.
               \item For $n = \lambda b^p$, we get the decay of $\Order(n^{-\alpha})$.
                \item 
               \item \textbf{Future Work:} Evaluate performance of lattices with arbitrary $n$ in concrete numerical test cases.
	       \end{itemize}
    \end{frame}
    


\section*{References}
    \begin{frame}{References}

	   {\fontsize{6}{6}\selectfont 
	
    \bibliographystyle{plain}
    \bibliography{FJH23,FJHown23,LarysaReferences}
  }
    \end{frame}

\section*{}  
\begin{frame}
\begin{center}
\textcolor {red} {\Huge Thank you!}
\end{center}
\end{frame}



\end{document}
