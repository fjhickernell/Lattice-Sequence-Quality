\documentclass{beamer}
\setbeamertemplate{bibliography item}{[\theenumiv]}
\usepackage[english]{babel}
\usepackage{amsfonts}
\usepackage{amsmath}
\usepackage{amssymb}
% \usepackage{graphicx}
\usepackage{array}
\usepackage{xcolor}
\usepackage{setspace}
 \usepackage{movie15}
\usetheme{CambridgeUS}
% \usetheme{Boadilla}
 \usepackage{hyperref}

\usepackage[authoryear]{natbib}
% \usepackage{mathtools,upref,siunitx,upquote,fancyvrb,bbm,xspace,color,amsmath,amssymb, bm,amsthm }
% \usepackage[hyphens]{url}
% \usepackage[utf8]{inputenc}
% \usepackage{esdiff}
% \usepackage{graphicx}
% \usepackage{xcolor}

\input{FJHDef.tex}
\usepackage{xspace}
\usepackage{algpseudocode}
\usepackage{algorithm, algorithmicx}
\algnewcommand\algorithmicparam{\textbf{Parameters:}}
\algnewcommand\PARAM{\item[\algorithmicparam]}
\algnewcommand\algorithmicinput{\textbf{Input:}}
\algnewcommand\INPUT{\item[\algorithmicinput]}
\algnewcommand\RETURN{\State \textbf{Return }}

\newcommand{\tr}{\widetilde{r}}
\newcommand{\appxintn}{\appxint_n}
\DeclareMathOperator{\appxint}{\hat{I}}
\DeclareMathOperator{\trun}{trunc}
\newcommand{\onetos}{1\!:\!s}

% \newcommand{\FredNote}[1]{{\color{blue}#1}}
% \newcommand{\FredBNote}[1]{{\color{blue}[#1]}}

% \newcommand{\LarysaNote}[1]{{\color{violet}#1}}

% \newtheorem{theorem}{Theorem}

\setbeamertemplate{footline}{
  \leavevmode%
  \hbox{%
    \begin{beamercolorbox}[wd=.5\paperwidth,ht=2.5ex,dp=1ex,left]{author in head/foot}%
      \hspace*{1em}%
    \end{beamercolorbox}%
    \begin{beamercolorbox}[wd=.5\paperwidth,ht=2.5ex,dp=1ex,right]{date in head/foot}%
      \insertframenumber{} / \inserttotalframenumber\hspace*{1em}
    \end{beamercolorbox}%
  }
}


\title[Illinois Institute of Technology]{The Quality of Lattice Sequences}
% \subtitle{\footnotesize This research is supported in part by U.S.\  NSF  Grant \#2316011}
%\subtitle{\footnotesize Larysa Matiukha \& Dr.\ Sou-Cheng T.\ Choi}

\author[] {Larysa Matiukha, Yuhan Ding, Fred J Hickernell
}

\institute{Department of Applied Mathematics, Illinois Institute of Technology}


\date{28 July 2025}

% \setbeamertemplate{title page}{
%   \vbox{}
%   \vfill
%   \begin{centering}
%     {\usebeamerfont{title}\inserttitle\par}
%     \vskip0.5em
%     {\usebeamerfont{author}\insertauthor\par}
%     \vskip0.5em
%     {\usebeamerfont{institute}\insertinstitute\par}
%     \vskip1em
%     {\footnotesize This research is supported in part by U.S.\  NSF  Grant \#2316011.\par}
%     \vskip1em
%     {\usebeamerfont{date}\insertdate\par}
%   \end{centering}
%   \vfill
% }

%{{This research is supported in part by U.s. NSF  Grant #2316011}}
\graphicspath{ {./images/} }

% \logo{
% \includegraphics[width=1cm]{thesis docs/slides_images/iit_logo.png}

% }

 % you can change it on choice
% I recommend: CambridgeUS, Berlin, Singapore
% for more choices, visit: https://deic.uab.cat/~iblanes/beamer_gallery/index_by_theme_and_color.html



\begin{document}
{

\begin{frame} 
    \maketitle
\end{frame}
}
%\frame{\titlepage}
\begin{frame}
\frametitle{Contents}
\tableofcontents
\end{frame}


\addtocontents{toc}{\setcounter{tocdepth}{2}}


\section{Motivation}
% \begin{frame}{Motivation}
% % \textbf{High-dimensional integrals}:
% % \begin{itemize}
% %     \item Appear in quantitative finance \cite{CafMorOwe97}, statistics \cite{Gen92,GenBre99}, and physics \cite{PapTra97}
% %     \item Often arise as expectations of random variables: 
% %     \[
% %     \mathbb{E}[f(\vX)] = \int_{[0,1)^s} f(\vx) \, \dif \vx
% %     \]
% % \end{itemize}

% % \vspace{0.5em}
% % \textbf{Challenge:} Analytical solutions are often infeasible.

% % \vspace{0.5em}
% % \textbf{Solution:} Approximate by a sample mean:
% % \[
% % \appxintn(f) = \frac{1}{n} \sum_{i=0}^{n-1} f(\vz_i)
% % \]
% % \begin{itemize}
% %     \item \textbf{Monte Carlo (MC):} $\vz_i$ are IID
% %     \item \textbf{Quasi-Monte Carlo (QMC):} $\vz_i$ more evenly spaced
% % \end{itemize}



% \vspace{0.5em}
% \textbf{Lattices}
% \begin{itemize}
%   %  \item A popular choice of nodes for approximating multidimensional integrals by a sample mean
%   \item Node sets $\{\vz_i\}_{i=0}^{n-1} \in [0,1)^{s \times n}$ that fill the space evenly
%     \item Can be defined extensibly for $n = b, b^2, \ldots$, where $b$ is a prime \cite{HicEtal00, Mai81a} :
%     % \item (something on faster convergence compared to MC?)
% %     \[
% % \{\vz_i = \vh \phi(i) + \vDelta \pmod{\vone} : i = 0,1,\ldots\} \subset [0,1)^s
% % \]
% % \begin{itemize}
% %     \item $\vh$ must be a \textbf{generalized integer} \cite[Section 2]{HicNie03a}.
% %     \item $\phi(i)$: the \textbf{van der Corput sequence} in base $b$.
% % \end{itemize}
% \end{itemize}


% \begin{figure}
%     \centering
%     \includegraphics[width=0.8\linewidth]{thesis docs/lattice-plot1.png}
%     \label{fig:enter-label}
% \end{figure}



% \begin{tabular}{m{0.3\textwidth}m{0.7\textwidth}}
% What about  $ n = 47$?
% & \includegraphics[width=0.3\linewidth]{thesis docs/plots/lattice_47.png} 
% \end{tabular}
% \end{frame}
% \begin{frame}{Extensible Lattices and Contributions}

% \textbf{Extensible lattice sequences:} \cite{HicEtal00, Mai81a}
% \[
% \{\vz_i = \vh \phi(i) + \vDelta \pmod{\vone} : i = 0,1,\ldots\} \subset [0,1)^s
% \]
% \begin{itemize}
%     \item $\vh \in \mathbb{Z}^s$: generating vector
%     \item $\vDelta \in [0,1)^s$: random shift
%     \item $\phi(i)$: the \textbf{van der Corput sequence} in base $b$.
    
% \end{itemize}

% \vspace{0.5em}
% \textbf{Limitation:} Lattice sequences defined for sample size $n = b^m$

% \vspace{0.5em}
% \textbf{Our contribution:}
% \begin{itemize}
%     \item Derive theoretical upper bounds on the figure of merit $P_\alpha$ for arbitrary $n$
%     \item Explore behavior of $P_{\alpha,2}$ numerically
% \end{itemize}
% \end{frame}


\begin{frame}{What About Partial Lattices?}

\textbf{Extensible Lattice Sequences} \cite{HicEtal00, Mai81a}:
\small
\[
\{\vz_i = \vh \phi(i) + \vDelta \pmod{\vone} : i = 0,1,\ldots\} \subset [0,1)^s
\]

\begin{itemize}
    \item $\vh$: a \textbf{generating vector} with \textbf{generalized integer} coordinates
    \item $\phi(i)$: van der Corput sequence in prime base $b$ 
    %\item $\vh \in \{1, \ldots, n-1\}^s$ is the \textbf{generating vector}, must be a \textbf{generalized integer} \cite[Section 2]{HicNie03a}.
    \item For $n = b^m$, $m \in \naturals$ \cite[Theorem 5]{HicNie03a} shows there exists $\vh$ with:
\end{itemize}

\vspace{-1em}
\small
\begin{equation*}
P_\alpha(\vh, \vgamma, n, \onetos) \le C_{P}(\alpha,\vgamma,\epsilon,s)\, n^{-\alpha} (\log n)^{\alpha(s+1)} [\log \log( n +1 )]^{\alpha(1+\epsilon)}
\end{equation*}
\normalsize

\vspace{-0.5em}
\begin{figure}
    \centering
    \includegraphics[width=0.82\linewidth]{lattice-plot1.png}
\end{figure}

\vspace{-0.8em}
\textbf{Question:} How much worse is $P_\alpha$ when $n$ is \emph{not} a power of the base $b$?
%\textbf{Question:} How much worse is the figure of merit $P_\alpha$ when $n$ is \emph{not} a power of the base $b$?
\end{frame}

% \begin{tabular}{m{0.3\textwidth}m{0.7\textwidth}}
% What about  $ n = 47$?
% & \includegraphics[width=0.3\linewidth]{thesis docs/plots/lattice_47.png} 
% \end{tabular}
% \begin{frame}

% \begin{itemize}
%     \item Extend the upper bound in \cite[Theorem 5]{HicNie03a} to all positive $n$
%     \item 
% \end{itemize}
    
% \end{frame}
\begin{frame}{Examples of Partial Lattices}
  \begin{columns}[c] 

  \column{0.31\textwidth}
  \includegraphics[width=\linewidth]{plots/lattice_40.png}
  \centering \small 

  \column{0.31\textwidth}
  \includegraphics[width=\linewidth]{plots/lattice_47.png}
  \centering \small 

  \column{0.31\textwidth}
  \includegraphics[width=\linewidth]{plots/lattice_48.png}
  \centering \small 

\end{columns}
\end{frame}

 \section{Worst Case Error Analysis for Arbitrary $n$}

\begin{frame}{Worst-Case Error Analysis for arbitrary $n$}

\textbf{Goal:} Bound the integration error 
\small
\[
\left| \int_{[0,1)^s} f(\vx) \, \dif \vx - \appxint_n(f) \right|,
\quad \appxintn(f) := \frac{1}{n} \sum_{i=0}^{n-1} f(\vz_i) \]
\textbf{Set-up:}
\begin{itemize} 
    \item Assume $f$ has an absolutely summable Fourier series %\quad (with Fourier coefficients $\tilde{f}(\vk)$)
    \item Coordinate weights:
    \small
    \[ \tr(\vk,\vgamma) = \prod_{j=1}^{s} r(k_{j},\gamma_{j}),
\quad  r(k_{j},\gamma_{j})=\begin{cases} 1, &
k_{j}=0, \\ \gamma_{j}^{-1}\abs{k_{j}}, & k_{j} \ne 0.  \end{cases}\]
    \item  Banach space of functions:
    \small
\[
\cf_{\alpha} = \{ f \in \cl_2[0,1)^{s} :
\norm[\cf{\alpha}]{f} < \infty \}, \quad
\norm[\cf{\alpha}]{f} := \sup_{\vk \in \integers^{s}}
\left(\tr(\vk,\vgamma)^{\alpha} \abs{\tilde{f}(\vk)} \right),
\]
\item[] where \(\tilde{f}(\vk)\) are the Fourier coefficients.
    % \item The cubature error can be bounded by:
    % \small
    % \[
    % |\text{Error}| \le \norm[\cf{\alpha}]{f} \cdot P_\alpha(\vh,\vgamma,n,1:s)
    % \]
\end{itemize}
\end{frame}

% \begin{frame}{Examples of Partial Lattices}
    
% \end{frame}



\begin{frame}{Cubature Error} 
The cubature error can be bounded by:
    \small
    \[
    |\text{Error}| \le \norm[\cf{\alpha}]{f} \cdot \underbrace{P_\alpha(\vh,\vgamma,n,1:s)}_{ \text{quality of the nodes}}
    \]
\begin{itemize}
\item For arbitrary $n$: 
\small
\[ P_\alpha(\vh,\vgamma,n,1:s) = \sum_{\vk \in \integers^s \setminus\{\vzero\}} \abs{\appxint_n(\me^{2 \pi \sqrt{-1} \vk^T \cdot})}\tr(\vk,\vgamma)^{-\alpha}\]
\item For $n = b^m$: \small
\[
P_\alpha(\vh,\vgamma,b^m,\onetos) = \sum_{\vk \in B(\vh,m,1:s )} \tr(\vk,\vgamma)^{-\alpha}
\]
\item[] where 
\vspace{-1ex} \[ B(\vh,m,\onetos) := \left\{ \vk \in \mathbb{Z}^s \setminus \{\vzero\} : \vk^T \vh \equiv 0 \pmod{b^m} \right\}
\]
% \[
% \text{where } B(\vh,m,\onetos) := \left\{ \vk \in \mathbb{Z}^s \setminus \{\vzero\} : \vk^T \vh \equiv 0 \pmod{b^m} \right\}
% \]
\normalsize
\end{itemize}
% where 
% \[
% B(\vh,m,\onetos) := \left\{ \vk \in \mathbb{Z}^s \setminus \{\vzero\} : \vk^T \vh = 0 \pmod{b^m} \right\}
% \]
%\textbf{$P_\alpha$}: quality of the nodes
\end{frame}


\section{The figure of merit $P_\alpha$ for $n$ other than $b^m$}

%\begin{frame}{Theoretical Bounds for Arbitrary $n$}
\begin{frame}{$P_\alpha$ for arbitrary $n$}
\begin{block}{Theorem 1}
    % For fixed integer base $b \ge 2$, a fixed smoothness parameter $\alpha \ge 1$, a fixed vector of coordinate weights $\vgamma \in [0,\infty)^\infty$, and a fixed $\epsilon > 0$, there exists a generating vector for the lattice, $\vh$, for which the figure of merit, $P_\alpha$, has the following upper bound for arbitrary positive integer $n$:
    \begin{multline*} 
        P_\alpha(\vh,\vgamma,n,\onetos) 
        \le \frac {C_{P}(\alpha,\vgamma,\epsilon,s)}{\textcolor {red}{n}} \frac{(b-1)}{1 - b^{1-\alpha}} \\
        \times [\max(1, (\log n)^{\alpha(s+1)})]
      [\max(1,\log \log (n+1))]^{\alpha(1+\epsilon)} \\
      \forall n , s \in \naturals, \ \alpha > 1.
    \end{multline*}
        \begin{multline*} 
         P_\alpha(\vh,\vgamma,\lambda b^p,\onetos) 
         \le \frac{\textcolor{red}{\lambda^{\alpha -1}}{C}_{P}(\alpha, \vgamma,\epsilon,s)}{\textcolor {red}{n^{\alpha}}}  \frac {(b-1)} {1 - b^{1- \alpha}} \\
         \times \max(1, (\log n )^{\alpha(s+1)}) [\max(1,\log \log (
    n+1))]^{\alpha(1+\epsilon)} \\
    \forall p \in \natzero, \ \lambda, s \in \naturals, \ \alpha > 1.
    \end{multline*}
    \end{block}
    
\end{frame}
%\begin{frame}{$P_\alpha$ for $n = \lambda b^p$}


%     \begin{block}{Theorem 1 (Part 2)}
%     % This is the same power of logarithm as for $n = b^{m}$, but the power of $n$ is fixed at $-1$.

%     % However, if $n$ is a (small) integer multiple of a power of the base, then $P_\alpha$ has an upper bound that is similar to the case of $n=b^m$:
%     \begin{multline*} 
%          P_\alpha(\vh,\vgamma,\lambda b^m,\onetos) 
%          \le \frac{\lambda^{\alpha -1}{C}_{P}(\alpha, \vgamma,\epsilon,s)}{\textcolor {red}{n^{\alpha}}}  \frac {(b-1)} {1 - b^{1- \alpha}} \\
%          \times \max(1, (\log n )^{\alpha(s+1)}) [\max(1,\log \log (
%     n+1))]^{\alpha(1+\epsilon)} \\
%     \forall m \in \natzero, \ \lambda, s \in \naturals.
%     \end{multline*}
%     \end{block}
% \end{frame}

\begin{frame}{$P_{\alpha}$ when the coordinate weights satisfy summability condition}
\begin{block} {Theorem 2}
   
Suppose $\alpha > 1$.
 If $\sum_{j = 1}^{\infty} \gamma_j ^{\alpha} < \infty$, then for any fixed $\delta > 0$ :
\[
  P_\alpha(\vh,\vgamma,n,\onetos)
        \leq \frac{\tilde{C}_P(\alpha, \vgamma, \delta)}{n^{1-\delta}} \frac{b^{\delta}(b - 1) }{b^{\delta} - 1} 
\] If $\sum_{j = 1}^{\infty} \gamma_j < \infty$, then for any fixed $\delta $ such that $0 < \delta < \alpha - 1  $:
\[
  P_\alpha(\vh,\vgamma,\lambda b^p,\onetos)
        \leq \frac{\lambda^{\alpha - 1 - \delta }\tilde{C}_P(\alpha, \vgamma, \delta)}{n^{\alpha - \delta}}\frac{(b-1)}{1 - b^{1 - \alpha + \delta}}.
\]
\end{block}
% (add citation for \cite[Theorem 2, case (iii)]{HicNie03a}) 
\end{frame}



\begin{frame}{Decay of the upper bound on $P_2$ for Different Sample Sizes}

\begin{columns}
\column{0.5\textwidth}
\includegraphics[width=\linewidth]{thesis docs/plots/p_alpha_O.png}

\vspace{0.5em}
\small Upper bound on $P_2$ for $1 \leq n \leq 2^{10}$

\begin{itemize}
  \item Nearly $\Order(n^{-2})$ when $n = 2^m$
  \item Slower $\Order(n^{-1})$ decay for general $n$
\end{itemize}

\column{0.5\textwidth}
\includegraphics[width=\linewidth]{thesis docs/plots/p_alpha_lambda_O.png}

\vspace{0.5em}
\small Upper bound on $P_2$ for $n = \lambda 2^p$ for various small $\lambda$'s

\begin{itemize}
  \item Still achieves $\Order(n^{-2})$ decay
  \item Smaller $\lambda$ tends to give smaller $P_2$
\end{itemize}
\end{columns}
\end{frame}


\begin{frame}{Keister Example with $d = 6, \  1 \leq n \leq 2^{15}$}
\[ f(\vx) = \pi^{d/2}\cos(\norm{\vx})\]
    \begin{columns}
        \column{0.5\textwidth} \includegraphics[width=\linewidth]{plots/keister_n.png}
         % \centering \footnotesize Absolute error vs $n$
         
        \column{0.5\textwidth} \includegraphics[width=\linewidth]{plots/keister_lambdas.png}
        % \centering \footnotesize $\lambda \in \{1,3,5,7\}$: error vs $n = \lambda 2^p$
    \end{columns}
    \vspace{0.5em}
\centering
\footnotesize %some observation maybe
\end{frame}
\section{$P_{\alpha,2}$ with optimal sample weights}
\begin{frame}{Improving Cubature Error with Unequal Weights}

\textbf{Standard approach:} Use equal weights $1/n$ for each sample point.

\vspace{0.5em}
\textbf{Question:} Can we improve the upper bound on the cubature error by assigning optimal sample weights?

\vspace{0.5em}
\textbf{Set-up:}
\begin{itemize}
    %\item We define a Hilbert space $\cf_{\alpha,2}$ of functions, with norm:
    \item Hilbert space of functions:
    \small
    \[ \cf_{\alpha,2} = \{ f \in \cl_2[0,1)^{s} :
\norm[\cf{\alpha},2]{f} < \infty \}, \quad
    \norm[\cf{\alpha,2}]{f}^2 := \sum_{\vk} \left(\tr(\vk,\vgamma)^\alpha |\tilde{f}(\vk)|\right)^2
    \]
    % $$
% \cf_{\alpha,2} = \{ f \in \cl_2[0,1)^{s} :
% \norm[\cf{\alpha},2]{f} < \infty \}, \qquad
% \norm[\cf{\alpha,2}]{f} := \left[\sum_{\vk \in \integers^s} \left(\tr(\vk,\vgamma)^{\alpha}\abs{\tilde{f}(\vk)}\right)^2 \right]^{\frac{1}{2}} .
% $$
     \vspace{-0.8em}
    \item Let $\appxintn(f) = \sum_{i=0}^{n-1} w_i f(\vz_i)$
    \vspace{0.5em}
    \item Worst-case error becomes:
    
    \[
     |\text{Error}|^2 \le \norm[\cf{\alpha,2}]{f}^2 \underbrace{ [ -1 + (\vw^T \mK \vw) ],}_{=: \textcolor{red}{ P_{\alpha,2}^2(\vh,\tilde{\vgamma},n,1:s)}}
    \]
    %   \[ 
    % |\text{Error}|^2 \le \norm[\cf{\alpha,2}]{f}^2 \cdot P_{\alpha,2}^2(\vh,\tilde{\vgamma},n,1:s), \quad P_{\alpha,2}^2(\vh,\tilde{\vgamma},n,1:s) := [ -1 + (\vw^T \mK \vw) ]  \]
    
    % [ -1 + (\vw^T \mK \vw) ] =: \norm[\cf{\alpha,2}]{f}^2 \cdot P_{\alpha,2}^2
   
    % \[
    % |\text{Error}|^2 \le \norm[\cf{\alpha,2}]{f}^2 [ -1 + (\vw^T \mK \vw) ] =: \norm[\cf{\alpha,2}]{f}^2 \cdot P_{\alpha,2}^2
    % \] 
    
   \item[] where
    \vspace{-1em}
    \small
    \[\mK   = \biggl( \prod_{\ell = 1}^s [1 + \tilde{\gamma}_{\ell}^{2 \alpha}B_{2\alpha}(x_{i,\ell} - x_{j, \ell} \bmod 1 )]\biggr)_{i,j=0}^{n-1} \]
\end{itemize}
\end{frame}

\begin{frame}{Computing Optimal Weights}
% \[ \min_{\vw} \vw^T \mK \vw \quad \text{s.t.} \quad \vw^T \vone = 1. \]
\textbf{Goal:} Minimize $\vw^T \mK \vw$

\vspace{0.5em}
\textbf{Subject to:} $\sum_{j=0}^{n-1} w_j = 1$

\vspace{1em}
\textbf{Solution (via Lagrange multipliers):}
\[
\vw = \frac{\mK^{-1} \vone}{\vone^T \mK^{-1} \vone}
\]

\textbf{Result:}
\begin{itemize}
    \item This choice of weights minimizes the cubature error bound
    \item Unequal weights can improve $P_{\alpha,2}$ for arbitrary $n$
    \item \textbf{Cost:} Can be reduced to $\Order(sn) + \Order(n^2)$ since $\mK$ is Toeplitz
\end{itemize}
\end{frame}

% The cost of solving for the weights $\vw$ is $\Order(sn^2)$ to evaluate the kernel and $\Order(n^3)$ to invert the Gram matrix.  (We assume that evaluating one element of the Gram matrix is proportional to the dimension.)  However, since $\mK$ is Toeplitz, we can reduce the cost of solving for optimal sample weights to $\Order(sn) + \Order(n^2)$.

\begin{frame}{$P_{\alpha,2}$ with Optimal Weights vs Equal Weights}

% \begin{columns}
% \column{0.5\textwidth}
% \includegraphics[width=\linewidth]{thesis docs/plots/ssdisc_vs_ssdiscopt.png}

% \vspace{0.2em}
% \scriptsize
% \textbf{$P_{1,2}$ with equal weight vs optimal weights}  
% Optimal weights lead to smaller $P_{1,2}$, and non-increasing behavior with $n$.

% \column{0.5\textwidth}
% \includegraphics[width=\linewidth]{thesis docs/plots/ratio_floor.png}
% \scriptsize
% \textbf{The ratio $P_{1,2}(\vh,[1, 1/4],n,1:2)/P_{1,2} (\vh,[1, 1/4],2^{\lfloor \log_2(n) \rfloor},1:2)$} for $ 1 \leq n \leq 2^7$. This shows how much better (smaller than one) or worse (greater than one) the figure of merit is compared to the largest sample size that is a power of $2$ and no greater than $n$.

% \column{0.5\textwidth}
% \includegraphics[width=\linewidth]{thesis docs/plots/ratio_ceil.png}
% \scriptsize

%\end{columns}
\begin{figure}
    \centering
    \includegraphics[width=0.5\linewidth]{thesis docs/plots/ssdisc_vs_ssdiscopt.png}

\end{figure}
\textbf{$P_{1,2}$ with equal weight vs optimal weights} for $1 \le n \le 2^{12}$. Optimal weights lead to consistently smaller and non-increasing values. 
\end{frame}

\begin{frame}%{ }
\begin{columns}
\column{0.5\textwidth}
\includegraphics[width=\linewidth]{thesis docs/plots/ratio_floor.png}
\scriptsize
\textbf{The ratio $P_{1,2}(n)/P_{1,2} (2^{\lfloor \log_2(n) \rfloor})$} for $ 1 \leq n \leq 2^7$. This shows how much better or worse the figure of merit is compared to the largest sample size that is a power of $2$ and no greater than $n$.

\column{0.5\textwidth}
\includegraphics[width=\linewidth]{thesis docs/plots/ratio_ceil.png}
\scriptsize
\textbf{The ratio $P_{1,2}(n)/P_{1,2} (2^{\lceil \log_2(n) \rceil})$} for $ 1 \leq n \leq 2^7$. This shows how much worse the figure of merit is compared to the smallest sample size that is a power of $2$ and no less than $n$.
\end{columns}
\end{frame}


\section{Future Work}    
\begin{frame}{Future Work}
    \begin{itemize}
        \item Establish a theoretical upper bound on the figure of merit $P_{\alpha,2}$ for all $n$
        \item Numerically study the decay of the upper bound on $P_\alpha$ and decay of $P_{\alpha,2}$ for a wider range of $\alpha$ values. 
        \item Evaluate the performance of lattices with arbitrary $n$ in practical examples such as option pricing.
    \end{itemize}
\end{frame}

\section*{References}
    \begin{frame}{References}

	   {\fontsize{6}{6}\selectfont 
	
    %\bibliographystyle{plain}
    \bibliographystyle{plainnat} 
    \bibliography{FJH23,FJHown23,LarysaReferences}
  }
    \end{frame}

% \section*{}  
% \begin{frame}
% \begin{center}
% \textcolor {red} {\Huge Thank you!}
% \end{center}
% \end{frame}



\end{document}
