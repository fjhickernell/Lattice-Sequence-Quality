\documentclass{amsart}
\usepackage{mathtools,upref,siunitx,upquote,fancyvrb,bbm,xspace,color}
\usepackage[hyphens]{url}
\usepackage[utf8]{inputenc}
\usepackage{esdiff}
\usepackage{graphicx}
\usepackage{xcolor}

\input{FJHDef.tex}


\usepackage{algpseudocode}
\usepackage{algorithm, algorithmicx}
\algnewcommand\algorithmicparam{\textbf{Parameters:}}
\algnewcommand\PARAM{\item[\algorithmicparam]}
\algnewcommand\algorithmicinput{\textbf{Input:}}
\algnewcommand\INPUT{\item[\algorithmicinput]}
\algnewcommand\RETURN{\State \textbf{Return }}

\newcommand{\tr}{\widetilde{r}}
\newcommand{\appxintn}{\appxint_n}
\DeclareMathOperator{\appxint}{\hat{I}}
\DeclareMathOperator{\trun}{trunc}

\newcommand{\FredNote}[1]{{\color{blue}#1}}

\newcommand{\LarysaNote}[1]{{\color{violet}#1}}



\begin{document}
\title{The Quality of Lattice Sequences}
\author{Larysa Matiukha}
\author{Yuhan Ding}
\author{Fred J. Hickernell}
\begin{abstract}This project is where all of the files and commands go that are needed elsewhere
\end{abstract}

\maketitle

\section{Introduction}
Lattices are a popular choice of nodes $\{\vz_i\}_{i=0}^{n-1} \in [0,1)^s$ for approximating multidimensional integrals by a sample mean,
\[
\int_{[0,1)^s} f(\vx) \, \dif \vx \approx \frac{1}{n} \sum_{i=0}^{n-1} f(\vz_i) =:\appxint(f),
\]
\cite{DicEtal22a,Nie92,SloJoe94}.
Historically, lattice points were initially constructed as sets with fixed cardinality, $n$, and took the form
\begin{equation} \label{eq:lat}
    \{\vz_i = i \vh/n \pmod{\vone} : i=0,1, \ldots, n-1 \} \in [0,1)^s,
\end{equation}
where $\vh \in \{1, \ldots, n-1\}^s$ is the \emph{generating vector}.  A (random) shift, $\vDelta \in [0,1)^s$, is often added:
\begin{equation} \label{eq:shlat}
    \{\vz_i = i \vh/n + \vDelta \pmod{\vone} : i=0,1, \ldots, n-1 \} \in [0,1)^s.
\end{equation}

\begin{figure}[h]
\includegraphics[width=5cm,trim={0 0 0 7.5mm},clip]{shifted-lattice}
\caption{Two-dimensional Shifted Lattice}
\label{fig:enter-label}
\end{figure}



Extensible lattice sequences were proposed by \cite{HicEtal00,Mai81a} and take the form
\begin{equation} \label{eq:extlat}
    \{\vz_i = \vh\phi(i)+ \vDelta \pmod{\vone} : i=0,1, \ldots \} \in [0,1)^s.
\end{equation}
where $\{\phi(\cdot)\}_{i=0}^\infty$ is the van der Corput sequence (in base 2 for now).  In this case $\vh$ must be a generalized integer as defined in \cite[Section 2]{HicNie03a}.

For the van der Corput sequence
\[
\phi((\cdots i_2 i_1 i_0)_2) = {}_20.i_0 i_1 i_2 \cdots.
\]
For example,
\[
\phi(6) = \phi(110_2) = {}_20.011 = \frac 38.
\]
Note that
\begin{equation} \label{eq:phipropone}
\{ \phi(i) : i = 0, \ldots, 2^m-1 \} = \{0, 2^{-m}, 2\times 2^{-m}, \ldots, 1 - 2^{-m} \}.
\end{equation}
Also note that
\begin{multline} \label{eq:phiproptwo}
\{ \phi(i) : i = \lambda \times 2^m , \ldots, (\lambda+1)2^m-1 \} \\
= \{\phi(\lambda \times 2^m) + 0, \phi(\lambda \times 2^m) + 2^{-m}, \ldots, \phi(\lambda \times 2^m) + 1 - 2^{-m} \} , \\
\lambda \in \natzero.
\end{multline}


Hickernell and Niederreiter \cite{HicNie03a} established the existence of extensible lattices with good generating vectors for the preferred values $n = 2, 4, 8, \ldots$.  The purpose here is to extend these results to all $n>1$, recognizing that the upper bounds on the figures of merit will be somewhat worse.

We will fix $s$ and ignore $\vgamma$ for now.

\subsection{Worst case error analysis}
We first consider integrands, $f$, that have an absolutely summable Fourier series:
\begin{equation} \label{eq:fseries}
    f(\vx) = \sum_{\vk \in \integers^s} \tf(\vk) \me^{2 \pi \sqrt{-1} \vk^T \vx}, \qquad \tf(\vk) = \int_{[0,1)^s} f(\vx) \me^{-2 \pi \sqrt{-1} \vk^T \vx}\, \dif \vx
\end{equation}
Define the weights
\begin{equation}
\tr(\vk) = \prod_{j=1}^{s} r(k_{j}),
\qquad \text{where} \qquad r(k_{j})=\begin{cases} 1, &
k_{j}=0, \\ \abs{k_{j}}, & k_{j} \ne 0.  \end{cases}
\end{equation}
Define a Banach
space of functions:
$$
\cf_{\alpha} = \{ f \in \cl_2[0,1)^s :
\norm[\cf{\alpha}]{f} < \infty \}, \qquad
\norm[\cf{\alpha}]{f} := \sup_{\vk \in \integers^{s}}
\left(\tr(\vk)^{\alpha} \abs{\tilde{f}(\vk)} \right).
$$
It then follows that the error in approximating the integral by the sample mean is
\begin{align} \label{eq:wcerrPalpha}
\abs{\int_{[0,1)^s} f(\vx) \, \dif \vx - \appxint_n(f)} &
= \abs{\sum_{\vk \in \integers^s} \tf(\vk) \left[\int_{[0,1)^s} \me^{2 \pi \sqrt{-1} \vk^T \vx} \, \dif \vx - \appxintn(\me^{2 \pi \sqrt{-1} \vk^T \cdot})\right]} \\
\nonumber
& = \abs{\sum_{\vk \in \integers^s \setminus\{\vzero\}} \tf(\vk) \left[ \appxintn(\me^{2 \pi \sqrt{-1} \vk^T \cdot})\right]} \\
\nonumber
& \le \norm[\cf{\alpha}]{f} \underbrace{\sum_{\vk \in \integers^s \setminus\{\vzero\}} \abs{\appxint_n(\me^{2 \pi \sqrt{-1} \vk^T \cdot})}\tr(\vk)^{-\alpha}}_{=: P_\alpha(\vh,n) = \text{quality of the lattice nodes}} \\
\nonumber
\end{align}

For $n = 2^m$, the quality measure $P_\alpha(\vh,n)$ takes on a simple form.  Note that
\begin{align}\label{Eq:DualLattice}
    \MoveEqLeft{\appxint_{2^m}(\me^{2 \pi \sqrt{-1} \vk^T \cdot})} \\
    \nonumber
    &= \frac 1{2^m} \sum_{i=0}^{2^m-1} \me^{2 \pi \sqrt{-1} \vk^T (\vh\phi(i) + \vDelta \pmod{\vone})} \\
    \nonumber
    &= \frac {\me^{2 \pi \sqrt{-1} \vk^T \vDelta}}{2^m} \sum_{i=0}^{2^m-1} \me^{2 \pi \sqrt{-1} \vk^T \vh\phi(i)} \\
    \nonumber
    &= \frac {\me^{2 \pi \sqrt{-1} \vk^T \vDelta}}{2^m} \sum_{i=0}^{2^m-1} \me^{2 \pi \sqrt{-1} \vk^T \vh i/2^m} \qquad \text{by \eqref{eq:phipropone}}\\
    \nonumber
    &= \frac {\me^{2 \pi \sqrt{-1} \vk^T \vDelta}}{2^m} \times
    \begin{cases}
    \frac{\me^{2 \pi \sqrt{-1} \vk^T \vh} - 1}{\me^{2 \pi \sqrt{-1} \vk^T \vh/2^m} - 1} = 0, & \vk^T \vh \pmod{2^m} \ne 0\\
    2^m, & \vk^T \vh \pmod{2^m} = 0
    \end{cases} \\
    \nonumber
    & = \me^{2 \pi \sqrt{-1} \vk^T \vDelta} \bbone_{B(\vh,m)}(\vk)
\end{align}
Thus it follows that
\begin{equation} \label{eq:Palphadual}
    P_\alpha(\vh,2^m) = \sum_{\vk \in B(\vh,m)} \tr(\vk)^{-\alpha},
\end{equation}
where $B(\vh,m) : = \{\vk \in \integers^s \setminus \{\vzero\} : \vk^T \vh \pmod{2^m} = 0\}$ is called the \emph{dual lattice}.  This corresponds to \cite[Equation (3)]{HicNie03a}, and we know by \cite[Theorem 5]{HicNie03a} that  there exists $\vh$ with
\begin{multline} \label{eq:Niedbd}
    P_{\alpha}(\vh,2^m) \le C_{R}(\alpha,\epsilon,s)
    2^{-m\alpha} \log(2^{m\alpha(s+1)}) [\log \log (
    2^m+1)]^{\alpha(1+\epsilon)}, \\ m = 1, 2,\ldots, \quad \alpha \ge 1.
\end{multline}


Note that
\begin{align*}
    P_{\alpha}(\vh,1) & = \sum_{\vk \ne \vzero} \tr(\vk)^{-\alpha} = -1 + \sum_{\vk \in \integers^s} \tr(\vk)^{-\alpha} \\
    & = -1 + \sum_{k_1 =-\infty}^{\infty} \cdots \sum_{k_s  =-\infty}^{\infty} \frac{1}{\prod_{j=1}^s[\max(1,\abs{k_j})]^\alpha} \\
    & = -1 + \sum_{k_1 =-\infty}^{\infty} \frac{1}{[\max(1,\abs{k_1})]^\alpha} \cdots \sum_{k_s  =-\infty}^{\infty} \frac{1}{[\max(1,\abs{k_s})]^\alpha} \\
    & = -1 + \left[\sum_{k =-\infty}^{\infty} \frac{1}{[\max(1,\abs{k})]^\alpha} \right]^d \\
    & = -1 + \left[1 + 2 \sum_{k =1}^{\infty} \frac{1}{k^\alpha} \right]^d \\
    & = -1 + \left[1 + 2 \zeta(\alpha) \right]^d
\end{align*}

Therefore, the above formula \eqref{eq:Niedbd} can be extended to the case of $m=0$:

\begin{multline} \label{eq:Palphaextm}
    P_{\alpha}(\vh,2^m) \\
    \le C_{R}(\alpha,\epsilon,s)
    2^{-m\alpha}\max(1, \log(2^{m\alpha(s+1)})) [\max(1,\log \log (
    2^m+1))]^{\alpha(1+\epsilon)}, \\ m =0, 1, 2,\ldots, \quad \alpha \ge 1.
\end{multline}

\subsection{Randomized Error Analysis}
Another approach is to assume that $f$ in \eqref{eq:fseries} is a random function (stochastic process) where
\begin{equation}
    \abs{\tf(\vk)} \IIDsim \cn(0,\tr(\vk)^{-2\alpha}).
\end{equation}
Then it follows that
\begin{align*}
    \MoveEqLeft{\Ex_f\left[\left(\int_{[0,1)^d} f(\vx) \, \dif \vx - \appxintn(f) \right)^2\right]} \\
    & =
    \Ex_f\left[\left(\sum_{\vk \in \integers^s \setminus\{\vzero\}} \tf(\vk) \appxintn(\me^{2 \pi \sqrt{-1} \vk^T \cdot})\right)^2\right] \\
     & =
    \Ex_f\left[\sum_{\vk,\vl \in \integers^s \setminus\{\vzero\}} \tf(\vk) \tf^*(\vl) \appxintn(\me^{2 \pi \sqrt{-1} \vk^T \cdot}) \appxintn(\me^{-2 \pi \sqrt{-1} \vl^T \cdot})\right] \\
    & = \underbrace{\sum_{\vk\in \integers^s \setminus\{\vzero\}} \tr(\vk)^{-2\alpha} \abs{\appxintn(\me^{2 \pi \sqrt{-1} \vk^T \cdot})}^2}_{Q_{2\alpha}(\vh,n) = \text{another measure of quality of the lattice nodes}}  \\
\end{align*}

\textbf{Research Question}  May we extend this result to all $n$ with perhaps some extra $\log(n)$ terms?

E.g.,  $n =6 = 0\cdot 1 + 1 \cdot 2 + 1 \cdot 4$, $m = 2$ $\vDelta = \vzero$.

\begin{align*}
   \MoveEqLeft{\appxint_{6}(\me^{2 \pi \sqrt{-1} \vk^T \cdot})}\\
    &= \frac 1{6} \sum_{i=0}^{5} \me^{2 \pi \sqrt{-1} \vk^T \vh\phi(i)} \\
    &= \frac {1}{6} \left\{ \sum_{i=0}^3 \me^{2 \pi \sqrt{-1} \vk^T \vh\phi(i)} + \sum_{i=4}^5 \me^{2 \pi \sqrt{-1} \vk^T \vh\phi(i)} + \sum_{i=6}^5 \me^{2 \pi \sqrt{-1} \vk^T \vh\phi(i)}  \right\} \\
    &= \frac {1}{6} \left\{ \sum_{i=0}^3 \me^{2 \pi \sqrt{-1} \vk^T \vh\phi(i)} + \sum_{i=0}^1 \me^{2 \pi \sqrt{-1} \vk^T \vh[\phi(i)+\phi(4)]} + \sum_{i=0}^{-1} \me^{2 \pi \sqrt{-1} \vk^T \vh[\phi(i)+\vphi(6)]}  \right\} \ \ \text{by \eqref{eq:phiproptwo}} \\
    &= \frac {1}{6} \left\{ 1\cdot\sum_{i=0}^{3}\me^{2 \pi \sqrt{-1} \vk^T \vh\phi(i)} + 1 \cdot \me^{2 \pi \sqrt{-1} \vk^T \phi(4)} \cdot \sum_{i=0}^{1}\me^{2 \pi \sqrt{-1} \vk^T \vh\phi(i)} \right . \\
    & \qquad \qquad \left . +  0 \cdot
    \me^{2 \pi \sqrt{-1} \vk^T \vh\phi(6)} \cdot \sum_{i=0}^{0}\me^{2 \pi \sqrt{-1} \vk^T \vh\phi(i)} \right\}
\end{align*}

 Then,
\begin{align*}
    P_\alpha(\vh,6) & = \sum_{\vk \in \integers^s \setminus\{\vzero\}}
    \abs{\appxint_6(\me^{2 \pi \sqrt{-1} \vk^T \cdot})}\tr(\vk)^{-\alpha}
    \\
    & \le
    \frac {1}{6} \left\{ 1\cdot\sum_{\vk \in \integers^s \setminus\{\vzero\}} \abs{\sum_{i=0}^{3}\me^{2 \pi \sqrt{-1} \vk^T \vh\phi(i)}} \tr(\vk)^{-\alpha}  \right .
    \\
    & \qquad \qquad + 1  \cdot \sum_{\vk \in \integers^s \setminus\{\vzero\}} \abs{\sum_{i=0}^{1}\me^{2 \pi \sqrt{-1} \vk^T \vh\phi(i)}} \tr(\vk)^{-\alpha}  \\
    & \qquad \qquad \left . +  0
    \cdot \sum_{\vk \in \integers^s \setminus\{\vzero\}} \abs{\sum_{i=0}^{0}\me^{2 \pi \sqrt{-1} \vk^T \vh\phi(i)}}  \tr(\vk)^{-\alpha} \right\}\\
    & =
    \frac {1}{6} \left\{ 1\cdot\sum_{\vk \in \integers^s \setminus\{\vzero\}} 4 \cdot \bbone_{B(\vh,2)}(\vk) \tr(\vk)^{-\alpha}  \right .
    + 1  \cdot \sum_{\vk \in \integers^s \setminus\{\vzero\}}2 \cdot \bbone_{B(\vh,1)}(\vk) \tr(\vk)^{-\alpha}  \\
    & \qquad \qquad \left . +  0
    \cdot \sum_{\vk \in \integers^s \setminus\{\vzero\}} 1 \cdot \bbone_{B(\vh,1)}(\vk)  \tr(\vk)^{-\alpha} \right\} \qquad \text{by \eqref{Eq:DualLattice}}\\
    & = \frac {1}{6} \left\{ 4 \cdot 1 \cdot P_\alpha(\vh,4)+ 2 \cdot 1 \cdot P_\alpha(\vh,2)+ 1 \cdot 0 \cdot P_\alpha(\vh,1) \right\} \\
\end{align*}

We know that and $P_\alpha(\vh,2^2) \le C_{R}(\alpha,\epsilon,s)
    2^{-2\alpha}$ and $P_\alpha(\vh,2^1) \le C_{R}(\alpha,\epsilon,s)
    2^{-\alpha} $. Then,
    \begin{align*}
        P_\alpha(\vh,6) & \le \frac {1}{6} \left\{ 4 \cdot 1 \cdot P_\alpha(\vh,4) + 2 \cdot 1 \cdot P_\alpha(\vh,2) \right\} \\
         & \le \frac {1}{6} \left\{C_{R}(\alpha,\epsilon,s)
         2^{2-2\alpha} + C_{R}(\alpha,\epsilon,s)
         2^{1-\alpha} \right\}
    \end{align*}


For any non-negative integer $n$, express as  $n = n_0 + 2n_1 + \cdots + 2^m n_m$, where $n_0, n_1, \ldots$ are the binary digits of $n$.  Thus, $m = \lfloor \log_2(n) \rfloor$.  Let $\trun(n,\ell) = n_0 + \cdots + 2^{\ell} n_{\ell}$ for $\ell = 0, \ldots, m$ and $\trun(n,-1) = 0$, then

\begin{align*}
    \MoveEqLeft{\appxint_{n}(\me^{2 \pi \sqrt{-1} \vk^T \cdot}) } \\
    &= \frac 1{n} \sum_{i=0}^{n-1} \me^{2 \pi \sqrt{-1} \vk^T (\vh\phi(i) + \vDelta \pmod{\vone})} \\
    \nonumber
    &= \frac {\me^{2 \pi \sqrt{-1} \vk^T \vDelta}}{n} \sum_{\ell = 0}^{m} \,
    \sum_{i=n - \trun(n,m-\ell) }^{n - \trun(n,m-\ell-1)  -1} \me^{2 \pi \sqrt{-1} \vk^T \vh\phi(i)} \\
    &= \frac {\me^{2 \pi \sqrt{-1} \vk^T \vDelta}}{n} \sum_{\ell = 0}^{m} \, n_{m-\ell}
    \sum_{i=0 }^{2^{m-\ell}  -1} \me^{2 \pi \sqrt{-1} \vk^T \vh[i/2^m + \phi(n - \trun(n,m-\ell))]} \\
    & \hspace{30ex} \text{by \eqref{eq:phiproptwo}} \\
    & = \frac {\me^{2 \pi \sqrt{-1} \vk^T \vDelta}}{n} \sum_{\ell = 0}^{m} 2^{m-\ell} n_\ell \bbone_{B(\vh,m-\ell)}(\vk) \me^{2 \pi \sqrt{-1} \vk^T \vh\phi(n - \trun(n,m-\ell))}.
\end{align*}
This implies that
\begin{equation} \label{eq:ihatarbn}
    \abs{\appxint_{n}(\me^{2 \pi \sqrt{-1} \vk^T \cdot})} \le \frac {1}{n} \sum_{\ell = 0}^{m} 2^{m-\ell} n_\ell \bbone_{B(\vh,m-\ell)}(\vk)
\end{equation}

Next we compute $P_\alpha(\vh,n)$ for this arbitrary $n$:
\begin{align*}
      P_\alpha(\vh,n) & = \sum_{\vk \in \integers^s \setminus\{\vzero\}} \abs{\appxint_n(\me^{2 \pi \sqrt{-1} \vk^T \cdot})}\tr(\vk)^{-\alpha} \qquad \text{by \eqref{eq:wcerrPalpha}} \\
      & \le \frac {1}{n} \sum_{\ell = 0}^{m} 2^{m-\ell} n_{m - \ell} \sum_{\vk \in B(\vh,m-\ell)} \tr(\vk)^{-\alpha}
      \qquad \text{by \eqref{eq:ihatarbn}}\\
      & = \frac {1}{n} \sum_{\ell = 0}^{m} 2^{m-\ell} n_{m - \ell} P_\alpha(\vh,2^{m-\ell})
      \qquad \text{by \eqref{eq:Palphadual}}\\
      & =  \frac {1}{n} \left\{ 2^m n_m P_\alpha(\vh,2^m) + 2^{m-1} n_{m-1} P_\alpha(\vh,2^{m-1}) + \cdots + 2^0 n_0 P_\alpha(\vh,2^0)  \right\} \\
      & \le \frac {C_{R}(\alpha,\epsilon,s)}{n} \left\{ 2^{-m\alpha}\max(1, \log(2^{m\alpha(s+1)})) [\max(1,\log \log (
    2^m+1))]^{\alpha(1+\epsilon)}\right .\\
      & \qquad  + 2^{-(m-1)\alpha}\max(1, \log(2^{(m-1)\alpha(s+1)})) [\max(1,\log \log (
    2^{(m-1)}+1))]^{\alpha(1+\epsilon)}\\
      & \qquad + \cdots \\
      & \qquad  \left . + 2^{-0}\max(1, \log(2^{0\alpha(s+1)})) [\max(1,\log \log (
    2^{0}+1))]^{\alpha(1+\epsilon)} \right\}\\
      & \qquad \qquad \qquad \alpha \ge 1 \qquad \text{by \eqref{eq:Palphaextm}}\\
      & \le \frac {C_{R}(\alpha,\epsilon,s)}{n} \frac{(1 - 2^{(m+1)(1-\alpha)})}{1 - 2^{1-\alpha}}
      \max(1, \log(2^{m\alpha(s+1)})) \\
      & \qquad \qquad \times [\max(1,\log \log (
    2^m+1))]^{\alpha(1+\epsilon)} \\
      & \le \frac {\widetilde{C}_{R}(\alpha,\epsilon,s)}{n}
      \max(1, \log(n^{\alpha(s+1)}))
      [\max(1,\log \log (
    n+1))]^{\alpha(1+\epsilon)}
      % add C_R and the worst log
\end{align*}
Unfortunately, this only decays like $\Order(n^{-1+\delta})$.


Thus, next we try $n = \lambda 2^p$, where $\lambda$ is an odd integer. Then, $n = 2^p + 2^{p+1}n_{p+1} + \cdots + 2^m n_m$
\FredNote{Let's carry on following the above.  WAtch out for your geometric sum} \\

\begin{align*}
    P_\alpha(\vh,\lambda 2^p)
    %& \le \frac {1}{\lambda 2^p} \left\{2^p  P_\alpha(\vh,2^p) + \sum_{\ell = 0}^{m-p-1} 2^{m-\ell} n_{m - \ell} P_\alpha(\vh,2^{m-\ell})\right\} \\
    & \le \frac {1}{\lambda 2^p} \sum_{\ell = 0}^{m-p} 2^{m-\ell} n_{m -\ell} P_\alpha(\vh,2^{m-\ell}) \\
    & \le \frac {1}{\lambda 2^p} \left\{2^m n_m P_\alpha(\vh,2^m) + 2^{m-1} n_{m-1} P_\alpha(\vh,2^{m-1}) + \cdots + 2^p n_p P_\alpha(\vh,2^p) % no n_p though
    \right\} \\ % this sum doesnt look right
    & \le \frac {1}{\lambda 2^p} \left\{C_{R}(\alpha,\epsilon,s) 2^{m(1-\alpha)} + C_{R}(\alpha,\epsilon,s) 2^{(m-1)(1-\alpha)} + \cdots +  C_{R}(\alpha,\epsilon,s) 2^{p(1-\alpha)}\right\} \\
    & = \frac {1}{\lambda 2^p} C_{R}(\alpha,\epsilon,s) \left\{ 2^{m(1-\alpha)} \cdot \frac{(1 - 2^{(m-p+1)(\alpha - 1)})}{1 - 2^{\alpha-1}}\right\} \\
    & = \frac {1}{\lambda 2^p} C_{R}(\alpha,\epsilon,s) \left\{ 2^{m(1-\alpha)} \cdot \frac{(2^{(m-p+1)(\alpha - 1)}-1)}{2^{\alpha-1}-1}\right\} \\
    & = \frac {1}{\lambda 2^p} C_{R}(\alpha,\epsilon,s) \frac{2^{(-p+1)(\alpha -1)} - 2^{m(1-\alpha)})} {2^{\alpha-1}-1}\\
    & = \frac {1}{\lambda 2^{p\alpha}} C_{R}(\alpha,\epsilon,s) \frac{2^{(\alpha -1)} - 2^{(m-p)(1-\alpha)})} {2^{\alpha-1}-1}\\
\end{align*}


% \left\{\right\}
\bibliographystyle{amsplain}
\bibliography{FJH23,FJHown23}

\end{document}