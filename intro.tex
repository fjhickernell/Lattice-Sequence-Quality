\documentclass[authoryear]{elsarticle}
\usepackage{mathtools,upref,siunitx,upquote,fancyvrb,bbm,xspace,color,amsmath,amssymb, bm}
\usepackage[hyphens]{url}
\usepackage[utf8]{inputenc}
\usepackage{esdiff}
\usepackage{graphicx}
\usepackage{xcolor}

\input{FJHDef.tex}


\usepackage{algpseudocode}
\usepackage{algorithm, algorithmicx}
\algnewcommand\algorithmicparam{\textbf{Parameters:}}
\algnewcommand\PARAM{\item[\algorithmicparam]}
\algnewcommand\algorithmicinput{\textbf{Input:}}
\algnewcommand\INPUT{\item[\algorithmicinput]}
\algnewcommand\RETURN{\State \textbf{Return }}

\newcommand{\tr}{\widetilde{r}}
\newcommand{\appxintn}{\appxint_n}
\DeclareMathOperator{\appxint}{\hat{I}}
\DeclareMathOperator{\trun}{trunc}
\newcommand{\onetos}{1\!:\!s}

\newcommand{\FredNote}[1]{{\color{blue}#1}}

\newcommand{\LarysaNote}[1]{{\color{violet}#1}}
\begin{document}

% \maketitle
\section{Introduction}



Multidimensional integrals arise in various fields and practical applications \LarysaNote{such as quantitative finance, statistics , and physics } \FredNote{Often these arise as expectations of random variables.}  However, many such integrals cannot be computed analytically and, thus, numerical methods are often utilized. \FredNote{A common approach to approximating an integral that represents a population mean is by a sample mean:} 
\[
\mathbb{E}[f(\underbrace{\vX}_{\sim \mathcal{U}[0,1]^d})] : = \int_{[0,1)^s} f(\vx) \, \dif \vx \approx \frac{1}{n} \sum_{i=0}^{n-1} f(\vz_i) =:\appxintn(f),
\]
\citep{DicEtal22a,Nie92,SloJoe94}.
The nodeset, $\{\vz_0, \vz_1, \ldots \}$, is chosen as independent and identically distributed (IID), which corresponds to simple Monte Carlo. However, the nodeset may be chosen to be more evenly distributed, which corresponds to quasi-Monte Carlo methods. Lattices are a popular choice of nodes $\{\vz_i\}_{i=0}^{\infty} \in [0,1)^s$ for such approximations. \\

% what we are doing 
% \LarysaNote{In this work, we study how poor lattices perform if they have \emph{not} preferred cardinality. Our main results are presented in Theorem 1, where we consider the case when $n$ is  any positive integer, and also when $n$ is a integer multiple of a power of the base. We found that the error decays relatively slowly for any $n$, but when $n$ is a small multiple of, the decay is close to that of the preferred values. }
%In this work 

%we derive the upper bounds on the figure of merit $P_{\alpha}$ for extensible lattices with arbitrary cardinality $n$. We consider a case of an arbitrary positive $n$ and the case of $n = \lambda b^p$, where $\lambda$ is a odd integer. For arbitrary n, we found that $P_{\alpha}$ decays like $\Order(n^{-1})$, and in the case of $n = \lambda b^p$, $P_{\alpha}$ decays like $\Order(b^{-p \alpha})$, which can be faster when $\lambda$ is small. \\
\LarysaNote{Typically, for integral approximations lattices with preferred number of points $b^m$ are used ($b$ is a prime, $m$ is a positive integer). In this work, we study how big approximation error can be if we use lattices with any number of points $n$. We found that the error decays relatively slow for any $n$, but when $n$ is a small multiple of a power of the base, the decay is close to that of the preferred values.}

\cite{HicNie03a} established the existence of extensible lattices with good generating vectors for the preferred values $n = b, b^2, \ldots$.  Our purpose is to extend these results to all positive $n$, recognizing that the upper bounds on the figures of merit will be somewhat worse for general $n$ than for powers of the base. \\



   
In the next section, we give more background on extensible lattices. In Section 3 we go over the figures of merit and the worst case error analysis for arbitrary $n$ for integrands that have absolutely summable Fourier series. In Section 4 we derive upper bounds on the figure of merit % P_alpha 
for $n$ other than $b^m$, and in Section 5 we generalize the results to other spaces of integrands and error measures. 

\begin{figure}[h]
\centering
\includegraphics[width=5cm,clip]{lattice-plot.png}
\caption{Two-dimensional Shifted Lattice}
\label{fig:enter-label}
\end{figure}
\section{Background}

Historically, lattice points were initially constructed as sets with fixed cardinality, $n$, and took the form
\begin{equation} \label{eq:lat}
    \{\vz_i = i \vh/n \bmod{\vone} : i=0,1, \ldots, n-1 \} \in [0,1)^s,
\end{equation}
where $\vh \in \{1, \ldots, n-1\}^s$ is the \emph{generating vector}.  A (random) shift, $\vDelta \in [0,1)^s$, is often added:
\begin{equation} \label{eq:shlat}
    \{\vz_i = i \vh/n + \vDelta \pmod{\vone} : i=0,1, \ldots, n-1 \} \in [0,1)^s.
\end{equation}



% \cite{HicNie03a} established the existence of extensible lattices with good generating vectors for the preferred values $n = b, b^2, \ldots$.  The purpose of this thesis is to extend these results to all positive $n$, recognizing that the upper bounds on the figures of merit will be somewhat worse for general $n$ than for powers of the base. 

% In the next section, we give more background on extensible lattices. In Section 2 we go over the figures of merit and the worst case error analysis for arbitrary $n$ for integrands that have absolutely summable Fourier series. In Section 3 we derive upper bounds on the figure of merit % P_alpha 
% for $n$ other than $b^m$, and in Section 4 we generalize the results to other spaces of integrands and error measures. 














% Lattices are a popular choice of nodes $\{\vz_i\}_{i=0}^{\infty} \in [0,1)^s$ for approximating multidimensional integrals by a sample mean,
% \[
% \int_{[0,1)^s} f(\vx) \, \dif \vx \approx \frac{1}{n} \sum_{i=0}^{n-1} f(\vz_i) =:\appxintn(f),
% \]
% \citep{DicEtal22a,Nie92,SloJoe94}.



% \cite{HicNie03a} established the existence of extensible lattices with good generating vectors for the preferred values $n = b, b^2, \ldots$.  The purpose of this thesis is to extend these results to all positive $n$, recognizing that the upper bounds on the figures of merit will be somewhat worse for general $n$ than for powers of the base. 

% \begin{figure}[h]
% \centering
% \includegraphics[width=5cm,trim={0 0 0 7.5mm},clip]{shifted-lattice}
% \caption{Two-dimensional Shifted Lattice}
% \label{fig:enter-label}
% \end{figure}

% Historically, lattice points were initially constructed as sets with fixed cardinality, $n$, and took the form
% \begin{equation} \label{eq:lat}
%     \{\vz_i = i \vh/n \bmod{\vone} : i=0,1, \ldots, n-1 \} \in [0,1)^s,
% \end{equation}
% where $\vh \in \{1, \ldots, n-1\}^s$ is the \emph{generating vector}.  A (random) shift, $\vDelta \in [0,1)^s$, is often added:
% \begin{equation} \label{eq:shlat}
%     \{\vz_i = i \vh/n + \vDelta \pmod{\vone} : i=0,1, \ldots, n-1 \} \in [0,1)^s.
% \end{equation}




Extensible lattice sequences were proposed by \cite{HicEtal00,Mai81a} and take the form
\begin{equation} \label{eq:extlat}
    \{\vz_i = \vh\phi(i)+ \vDelta \pmod{\vone} : i=0,1, \ldots \} \in [0,1)^s.
\end{equation}
where $\{\phi(\cdot)\}_{i=0}^\infty$ is the van der Corput sequence in base $b$.  In this case $\vh$ must be a generalized integer as defined in \cite[Section 2]{HicNie03a}.

The van der Corput sequence is defined as: 
\[
\phi((\cdots i_2 i_1 i_0)_b) = {}_b0.i_0 i_1 i_2 \cdots.
\]
For example, for $b=2$,
\[
\phi(6) = \phi(110_2) = {}_20.011 = \frac 38.
\]
Note that the first ${b^m}$ points of the van der Corput sequence are just equally spaced points reordered. 
\begin{equation} \label{eq:phipropone}
\{ \phi(i) : i = 0, \ldots, b^m-1 \} = \{0, b^{-m}, 2\times b^{-m}, \ldots, 1 - b^{-m} \}.
\end{equation}
Also note that
\begin{multline} \label{eq:phiproptwo}
\{ \phi(i) : i = \lambda \times b^m , \ldots, (\lambda+1)b^m-1 \} \\
= \{\phi(\lambda \times b^m) + 0, \phi(\lambda \times b^m) + b^{-m}, \ldots, \phi(\lambda \times b^m) + 1 - b^{-m} \} , \\
\lambda \in \natzero.
\end{multline}

\cite{HicEtal10a} established 






\section{Proof of \eqref{eq:mainresulttwo}}
Next, suppose that $n$ is an integer multiple of the base. Let $n = \lambda b^p$, where $\lambda$ is an odd integer. Then,  $b$-ary expansion of $n$ takes the form:  $n = b^pn_p + b^{p+1}n_{p+1} + \cdots + b^m n_m$.



% suppose that $n = \lambda b^p$, where $\lambda$ is an odd integer. Then, the $n = b^pn_p + b^{p+1}n_{p+1} + \cdots + b^m n_m$
% % \FredNote{Let's carry on following the above.  WAtch out for your geometric sum}

Earlier we showed 
% Thus the figure of merit $P_\alpha(\vh,\vgamma,\lambda b^p,\onetos)$ is also bounded above by a sum of of $P_\alpha$for powers of the base.
It follows that $P_\alpha$ for $n = \lambda b^p$ can also be bounded above by a sum of $P_\alpha$ for powers of the base. 
\begin{align*}
    P_\alpha(\vh,\vgamma,\lambda b^p,\onetos)
    & = \sum_{\vk \in \integers^s \setminus\{\vzero\}} \abs{\appxint_{\lambda b^p}(\me^{2 \pi \sqrt{-1} \vk^T \cdot})}\tr(\vk,\vgamma)^{-\alpha} \qquad \text{by \eqref{eq:wcerrPalpha}} \\
    & \le \frac {1}{\lambda b^p} \sum_{\ell = 0}^{m-p} b^{m-\ell} n_{m -\ell} P_\alpha(\vh,\vgamma,b^{m-\ell},\onetos) \qquad \text{by \eqref{eq:Palphadual}}
\end{align*}
Similarly to the proof for arbitrary $n$, we substitute in the upper bounds for $P_\alpha$  
\begin{align*}
   P_\alpha(\vh,\vgamma,\lambda b^p,\onetos) 
    &\le \frac {1}{\lambda b^p} \left\{ b^m n_m P_\alpha(\vh,\vgamma,b^m,\onetos) + b^{m-1} n_{m-1} P_\alpha(\vh,\vgamma,b^{m-1},\onetos) \right. \\
    &\qquad + \cdots + b^p n_p P_\alpha(\vh,\vgamma,b^p,\onetos) 
    \bigg\} \\ 
    & \le \frac {(b-1)}{\lambda b^p}  
    \left\{
    b^m  P_\alpha(\vh,\vgamma,b^m,\onetos) + b^{m-1}  P_\alpha(\vh,\vgamma,b^{m-1},\onetos) \right. \\
    &\qquad + \cdots + b^p  P_\alpha(\vh,\vgamma,b^p,\onetos) 
    \bigg\} \\
    & \le \frac {(b-1)}{\lambda b^p} C_{P}(\alpha,\vgamma,\epsilon,s) \\
    &\qquad \times \left\{ b^{m(1-\alpha)}\max\left(1, (\log b^m)^{\alpha(s+1)}\right) \max\left(1,\log \log (b^m+1)\right)^{\alpha(1+\epsilon)} \right.\\ 
    & \qquad  + b^{(m-1)(1-\alpha)}\max\left(1, (\log b^{(m-1)})^{\alpha(s+1)}\right) \max\left(1,\log \log (b^{(m-1)}+1)\right)^{\alpha(1+\epsilon)}\\
    &  \qquad + \cdots \\
    & \qquad  \left . +   b^{p(1-\alpha)}\max\left(1,(\log b^p)^{\alpha(s+1)}\right) \max\left(1,\log \log (b^p+1)\right)^{\alpha(1+\epsilon)}  \right\}.
\end{align*}

Next, we factor out the largest logarithmic term and $b^{p(1-\alpha)}$
and compute the sum of the geometric series 

\begin{align*}
     P_\alpha(\vh,\vgamma,\lambda b^p,\onetos)
     & \le \frac {(b-1)}{\lambda b^p} C_{P}
    (\alpha,\vgamma,\epsilon,s) 
    \left\{\frac{b^{p(1-\alpha)} - b^{(m+1)(1- \alpha )}}{1 - b^{1 -\alpha}}\right\} \\
     & \qquad \qquad \times \max(1, (\log b^{m})^{\alpha(s+1)})) [\max(1,\log \log (
    b^m+1))]^{\alpha(1+\epsilon)} \\
    & = \frac { (b-1) C_{P}(\alpha, \vgamma, \epsilon,s)}{\lambda b^p} \left\{ b^{p(1-\alpha)} \cdot \frac{(1 - b^{(m-p+1)(1- \alpha )})}{1 - b^{1- \alpha}}\right\} \\
    & \qquad \qquad \times \max(1, (\log b^{m})^{\alpha(s+1)})) [\max(1,\log \log (
    b^m+1))]^{\alpha(1+\epsilon)}
\end{align*}

Finally, we simplify the sum %to arrive(??) to the final
\begin{align*}
    P_\alpha(\vh,\vgamma,\lambda b^p,\onetos)
    &\leq  \frac {(b-1)C_{P}(\alpha, \vgamma, \epsilon,s)}{\lambda b^{p\alpha}}  \frac{1 - b^{(m-p+1)(1- \alpha )}}{1 - b^{1- \alpha}} \\
    &\qquad \times \max(1, (\log b^{m})^{\alpha(s+1)})) [\max(1,\log \log (
    b^m+1))]^{\alpha(1+\epsilon)} \\
    &\le \frac{\lambda^{\alpha -1}{C}_{P}(\alpha, \vgamma,\epsilon,s)}{n^{\alpha}}  \frac {(b-1)} {1 - b^{1- \alpha}}   \max(1, (\log n )^{\alpha(s+1)})) [\max(1,\log \log (
    n+1))]^{\alpha(1+\epsilon)}   
\end{align*}
This completes the proof of \eqref{eq:mainresulttwo}. The upper bound for this case is 



% Which decays  nearly like $\Order(b^{-p \alpha})$ \\
% For small $\lambda$ the decay is closer to $\Order(n^{-\alpha})$, and for large $\lambda$ the decay is close to $\Order(n^{-1})$

In this case the upper bound decays almost like $\Order(b^{-p \alpha})$
which is an improvement since for small $\lambda$ the decay is close to that for preferred values of $n$. 
However, for large $\lambda$, the decay is closer to (4.1)

\bibliographystyle{elsarticle-harv.bst}
\bibliography{FJH23,FJHown23}
 \end{document}