\documentclass[authoryear]{elsarticle}
\usepackage{mathtools,upref,siunitx,upquote,fancyvrb,bbm,xspace,color,amsmath,amssymb, bm}
\usepackage[hyphens]{url}
\usepackage[utf8]{inputenc}
\usepackage{esdiff}
\usepackage{graphicx}
\usepackage{xcolor}

\input{FJHDef.tex}


\usepackage{algpseudocode}
\usepackage{algorithm, algorithmicx}
\algnewcommand\algorithmicparam{\textbf{Parameters:}}
\algnewcommand\PARAM{\item[\algorithmicparam]}
\algnewcommand\algorithmicinput{\textbf{Input:}}
\algnewcommand\INPUT{\item[\algorithmicinput]}
\algnewcommand\RETURN{\State \textbf{Return }}

\newcommand{\tr}{\widetilde{r}}
\newcommand{\appxintn}{\appxint_n}
\DeclareMathOperator{\appxint}{\hat{I}}
\DeclareMathOperator{\trun}{trunc}
\newcommand{\onetos}{1\!:\!s}

\newcommand{\FredNote}[1]{{\color{blue}#1}}

\newcommand{\LarysaNote}[1]{{\color{violet}#1}}
\begin{document}

% \maketitle
\section{Introduction}



Multidimensional integrals arise in various fields and practical applications \LarysaNote{such as quantitative finance, statistics , and physics } \FredNote{Often these arise as expectations of random variables.}  However, many such integrals cannot be computed analytically and, thus, numerical methods are often utilized. \FredNote{A common approach to approximating an integral that represents a population mean is by a sample mean:} 
\[
\mathbb{E}[f(\underbrace{\vX}_{\sim \mathcal{U}[0,1]^d})] : = \int_{[0,1)^s} f(\vx) \, \dif \vx \approx \frac{1}{n} \sum_{i=0}^{n-1} f(\vz_i) =:\appxintn(f),
\]
\citep{DicEtal22a,Nie92,SloJoe94}.
The nodeset, $\{\vz_0, \vz_1, \ldots \}$, is chosen as independent and identically distributed (IID), which corresponds to simple Monte Carlo. However, the nodeset may be chosen to be more evenly distributed, which corresponds to quasi-Monte Carlo methods. Lattices are a popular choice of nodes $\{\vz_i\}_{i=0}^{\infty} \in [0,1)^s$ for such approximations. \\

% what we are doing 

\LarysaNote{In this work we establish the upper bounds on the figures of merit for extensible lattices with arbitrary cardinality $n$. }
\cite{HicNie03a} established the existence of extensible lattices with good generating vectors for the preferred values $n = b, b^2, \ldots$.  Our purpose is to extend these results to all positive $n$, recognizing that the upper bounds on the figures of merit will be somewhat worse for general $n$ than for powers of the base. \\

\LarysaNote{We consider a case of an arbitrary positive $n$ and the case of $n = \lambda b^p$, where $\lambda$ is a small odd integer. We found $P_{\alpha}$  } \\

   
In the next section, we give more background on extensible lattices. In Section 2 we go over the figures of merit and the worst case error analysis for arbitrary $n$ for integrands that have absolutely summable Fourier series. In Section 3 we derive upper bounds on the figure of merit % P_alpha 
for $n$ other than $b^m$, and in Section 4 we generalize the results to other spaces of integrands and error measures. 

\begin{figure}[h]
\centering
\includegraphics[width=5cm,clip]{lattice-plot.png}
\caption{Two-dimensional Shifted Lattice}
\label{fig:enter-label}
\end{figure}
\section{Background}

Historically, lattice points were initially constructed as sets with fixed cardinality, $n$, and took the form
\begin{equation} \label{eq:lat}
    \{\vz_i = i \vh/n \bmod{\vone} : i=0,1, \ldots, n-1 \} \in [0,1)^s,
\end{equation}
where $\vh \in \{1, \ldots, n-1\}^s$ is the \emph{generating vector}.  A (random) shift, $\vDelta \in [0,1)^s$, is often added:
\begin{equation} \label{eq:shlat}
    \{\vz_i = i \vh/n + \vDelta \pmod{\vone} : i=0,1, \ldots, n-1 \} \in [0,1)^s.
\end{equation}



% \cite{HicNie03a} established the existence of extensible lattices with good generating vectors for the preferred values $n = b, b^2, \ldots$.  The purpose of this thesis is to extend these results to all positive $n$, recognizing that the upper bounds on the figures of merit will be somewhat worse for general $n$ than for powers of the base. 

% In the next section, we give more background on extensible lattices. In Section 2 we go over the figures of merit and the worst case error analysis for arbitrary $n$ for integrands that have absolutely summable Fourier series. In Section 3 we derive upper bounds on the figure of merit % P_alpha 
% for $n$ other than $b^m$, and in Section 4 we generalize the results to other spaces of integrands and error measures. 














% Lattices are a popular choice of nodes $\{\vz_i\}_{i=0}^{\infty} \in [0,1)^s$ for approximating multidimensional integrals by a sample mean,
% \[
% \int_{[0,1)^s} f(\vx) \, \dif \vx \approx \frac{1}{n} \sum_{i=0}^{n-1} f(\vz_i) =:\appxintn(f),
% \]
% \citep{DicEtal22a,Nie92,SloJoe94}.



% \cite{HicNie03a} established the existence of extensible lattices with good generating vectors for the preferred values $n = b, b^2, \ldots$.  The purpose of this thesis is to extend these results to all positive $n$, recognizing that the upper bounds on the figures of merit will be somewhat worse for general $n$ than for powers of the base. 

% \begin{figure}[h]
% \centering
% \includegraphics[width=5cm,trim={0 0 0 7.5mm},clip]{shifted-lattice}
% \caption{Two-dimensional Shifted Lattice}
% \label{fig:enter-label}
% \end{figure}

% Historically, lattice points were initially constructed as sets with fixed cardinality, $n$, and took the form
% \begin{equation} \label{eq:lat}
%     \{\vz_i = i \vh/n \bmod{\vone} : i=0,1, \ldots, n-1 \} \in [0,1)^s,
% \end{equation}
% where $\vh \in \{1, \ldots, n-1\}^s$ is the \emph{generating vector}.  A (random) shift, $\vDelta \in [0,1)^s$, is often added:
% \begin{equation} \label{eq:shlat}
%     \{\vz_i = i \vh/n + \vDelta \pmod{\vone} : i=0,1, \ldots, n-1 \} \in [0,1)^s.
% \end{equation}




Extensible lattice sequences were proposed by \cite{HicEtal00,Mai81a} and take the form
\begin{equation} \label{eq:extlat}
    \{\vz_i = \vh\phi(i)+ \vDelta \pmod{\vone} : i=0,1, \ldots \} \in [0,1)^s.
\end{equation}
where $\{\phi(\cdot)\}_{i=0}^\infty$ is the van der Corput sequence in base $b$.  In this case $\vh$ must be a generalized integer as defined in \cite[Section 2]{HicNie03a}.

The van der Corput sequence is defined as: 
\[
\phi((\cdots i_2 i_1 i_0)_b) = {}_b0.i_0 i_1 i_2 \cdots.
\]
For example, for $b=2$,
\[
\phi(6) = \phi(110_2) = {}_20.011 = \frac 38.
\]
Note that the first ${b^m}$ points of the van der Corput sequence are just equally spaced points reordered. 
\begin{equation} \label{eq:phipropone}
\{ \phi(i) : i = 0, \ldots, b^m-1 \} = \{0, b^{-m}, 2\times b^{-m}, \ldots, 1 - b^{-m} \}.
\end{equation}
Also note that
\begin{multline} \label{eq:phiproptwo}
\{ \phi(i) : i = \lambda \times b^m , \ldots, (\lambda+1)b^m-1 \} \\
= \{\phi(\lambda \times b^m) + 0, \phi(\lambda \times b^m) + b^{-m}, \ldots, \phi(\lambda \times b^m) + 1 - b^{-m} \} , \\
\lambda \in \natzero.
\end{multline}



\bibliographystyle{elsarticle-harv.bst}
\bibliography{FJH23,FJHown23}
 \end{document}